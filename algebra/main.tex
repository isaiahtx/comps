\documentclass[openany, amssymb]{amsart} \usepackage{mathrsfs,comment}
\usepackage[margin=1.3333333333in]{geometry}
\usepackage[usenames,dvipsnames]{color}
\usepackage[normalem]{ulem}
\usepackage{url}
\usepackage[all,arc,2cell]{xy}
\usepackage{enumerate}

%%%% ADDED

\usepackage{thmtools}
\usepackage{amsmath, amsthm, amssymb}
\usepackage{tikz}
\usepackage{tikz-cd}
\usepackage{todonotes}
\usepackage{tikz-3dplot}
\usepackage{quiver}
\usepackage{eucal}[mathcal]
\usepackage{enumitem}
\usepackage{subfiles}
\usepackage[backend=biber]{biblatex}
\addbibresource{refs.bib}
%\usepackage{parskip}

%%%

%%% hyperref stuff is taken from AGT style file
\usepackage{hyperref}  
\hypersetup{%
	colorlinks,%
	linkcolor={red!60!black},%
	citecolor={red!60!black},%
	urlcolor={red!60!black}%
}
  

\newenvironment{solution}
{\renewcommand\qedsymbol{}\begin{proof}[Solution]}
{\end{proof}}
%\let\fullref\autoref
%%
%%  \autoref is very crude.  It uses counters to distinguish environments
%%  so that if say {lemma} uses the {theorem} counter, then autrorefs
%%  which should come out Lemma X.Y in fact come out Theorem X.Y.  To
%%  correct this give each its own counter eg:
%%                 \newtheorem{theorem}{Theorem}[section]
%%                 \newtheorem{lemma}{Lemma}[section]
%%  and then equate the counters by commands like:
%%                 \makeatletter
%%                   \let\c@lemma\c@theorem
%%                  \makeatother
%%
%%  To work correctly the environment name must have a corrresponding 
%%  \XXXautorefname defined.  The following command does the job:
%%
%\def\makeautorefname#1#2{\expandafter\def\csname#1autorefname\endcsname{#2}}
%%
%%  Some standard autorefnames.  If the environment name for an autoref 
%%  you need is not listed below, add a similar line to your TeX file:
%%  
%%\makeautorefname{equation}{Equation}%
%\def\equationautorefname~#1\null{(#1)\null}
%\makeautorefname{footnote}{footnote}%
%\makeautorefname{item}{item}%
%\makeautorefname{figure}{Figure}%
%\makeautorefname{table}{Table}%
%\makeautorefname{part}{Part}%
%\makeautorefname{appendix}{Appendix}%
%\makeautorefname{chapter}{Chapter}%
%\makeautorefname{section}{Section}%
%\makeautorefname{subsection}{Section}%
%\makeautorefname{subsubsection}{Section}%
%\makeautorefname{theorem}{Theorem}%
%\makeautorefname{thm}{Theorem}%
%\makeautorefname{cor}{Corollary}%
%\makeautorefname{lem}{Lemma}%
%\makeautorefname{prop}{Proposition}%
%\makeautorefname{pro}{Property}
%\makeautorefname{conj}{Conjecture}%
%\makeautorefname{defn}{Definition}%
%\makeautorefname{notn}{Notation}
%\makeautorefname{notns}{Notations}
%\makeautorefname{rem}{Remark}%
%\makeautorefname{quest}{Question}%
%\makeautorefname{exmp}{Example}%
%\makeautorefname{ax}{Axiom}%
%\makeautorefname{claim}{Claim}%
%\makeautorefname{ass}{Assumption}%
%\makeautorefname{asss}{Assumptions}%
%\makeautorefname{con}{Construction}%
%\makeautorefname{prob}{Problem}%
%\makeautorefname{warn}{Warning}%
%\makeautorefname{obs}{Observation}%
%\makeautorefname{conv}{Convention}%
%
%
%%
%%                  *** End of hyperref stuff ***
%
%%theoremstyle{plain} --- default
%\newtheorem{thm}{Theorem}[section]
%\newtheorem{cor}{Corollary}[section]
%\newtheorem{prop}{Proposition}[section]
%\newtheorem{lem}{Lemma}[section]
%\newtheorem{prob}{Problem}[section]
%\newtheorem{conj}{Conjecture}[section]
%%\newtheorem{ass}{Assumption}[section]
%%\newtheorem{asses}{Assumptions}[section]
%
%\theoremstyle{definition}
%\newtheorem{defn}{Definition}[section]
%\newtheorem{ass}{Assumption}[section]
%\newtheorem{asss}{Assumptions}[section]
%\newtheorem{ax}{Axiom}[section]
%\newtheorem{con}{Construction}[section]
%\newtheorem{exmp}{Example}[section]
%\newtheorem{notn}{Notation}[section]
%\newtheorem{notns}{Notations}[section]
%\newtheorem{pro}{Property}[section]
%\newtheorem{quest}{Question}[section]
%\newtheorem{rem}{Remark}[section]
%\newtheorem{warn}{Warning}[section]
%\newtheorem{sch}{Scholium}[section]
%\newtheorem{obs}{Observation}[section]
%\newtheorem{conv}{Convention}[section]
%
%%%%% hack to get fullref working correctly
%\makeatletter
%\let\c@obs=\c@thm
%\let\c@cor=\c@thm
%\let\c@prop=\c@thm
%\let\c@lem=\c@thm
%\let\c@prob=\c@thm
%\let\c@con=\c@thm
%\let\c@conj=\c@thm
%\let\c@defn=\c@thm
%\let\c@notn=\c@thm
%\let\c@notns=\c@thm
%\let\c@exmp=\c@thm
%\let\c@ax=\c@thm
%\let\c@pro=\c@thm
%\let\c@ass=\c@thm
%\let\c@warn=\c@thm
%\let\c@rem=\c@thm
%\let\c@sch=\c@thm
%\let\c@equation\c@thm
%\numberwithin{equation}{section}
%\makeatother

%% theorems in usual style --- italicised text, bold header
\theoremstyle{plain}
\newtheorem{theorem}{Theorem}[section]
\newtheorem{corollary}[theorem]{Corollary}
\newtheorem{proposition}[theorem]{Proposition}
\newtheorem{lemma}[theorem]{Lemma}
\newtheorem*{thm*}{Theorem}
\newcommand{\n}{\mbf{n}}
\newcommand{\m}{\mbf{m}}

%% theorems in `definition' style --- regular text, bold header
\theoremstyle{definition}
\newtheorem{claim}[theorem]{Claim}
\newtheorem{remark}[theorem]{Remark}
\newtheorem{warning}[theorem]{Warning}
\newtheorem{definition}[theorem]{Definition}
\newtheorem{exercise}[theorem]{Exercise}
\newtheorem{condition}[theorem]{Condition}
\newtheorem{discussion}[theorem]{Discussion}
\newtheorem{notation}[theorem]{Notation}
\newtheorem{convention}[theorem]{Convention}
\newtheorem{conjecture}[theorem]{Conjecture}
\newtheorem{example}[theorem]{Example}

\usepackage[nameinlink]{cleveref}

\tikzcdset{scale cd/.style={every label/.append style={scale=#1},
    cells={nodes={scale=#1}}}}
\newcommand{\ol}{\overline}
\newcommand{\bDelta}{\mathbf{\Delta}}
\newcommand{\abs}[1]{\left\lvert #1\right\rvert}
\newcommand{\wh}{\widehat}

%% scale a tikz diagram with [sep=small]

\DeclareMathOperator*{\Rlim}{Rlim}
\DeclareMathOperator{\rank}{rank}
\DeclareMathOperator{\Ann}{Ann}
\DeclareMathOperator{\lcm}{lcm}
\DeclareMathOperator{\conj}{conj}
\DeclareMathOperator{\Stab}{Stab}
\DeclareMathOperator{\Orb}{Orb}
\DeclareMathOperator{\Cl}{Cl}

%% indent subsections in table of contents
\makeatletter 
%%\def\l@subsection{\@tocline{2}{0pt}{1pc}{5pc}{}} 
\def\l@subsection{\@tocline{2}{0pt}{2pc}{6pc}{}} \makeatother
\DeclareMathOperator{\Ch}{Ch}
\DeclareMathOperator{\Pic}{Pic}
\DeclareMathOperator{\Spec}{Spec}
\DeclareMathOperator{\imm}{im}
\DeclareMathOperator{\coker}{coker}
\DeclareMathOperator{\chr}{char}
\newcommand{\Mod}{\mbf{Mod}}
\newcommand{\acts}{\curvearrowright}
\newcommand{\Cell}{\mbf{Cell}}
\newcommand{\CoMod}{\mbf{CoMod}}
\newcommand{\Mon}{\mbf{Mon}}
\newcommand{\CMon}{\mbf{CMon}}
\newcommand{\Gr}{\mbf{Gr}}
\newcommand{\GrMod}{\mbf{GrMod}}
\newcommand{\cSH}{\mathcal{SH}}
\newcommand{\Top}{\mbf{Top}}
\newcommand{\Hom}{\mathrm{Hom}}
\newcommand{\aast}{{\ast\ast}}
\newcommand{\acast}{{\ast,\ast}}
\newcommand{\Ext}{\mathrm{Ext}}
\newcommand{\pt}{\mathrm{pt}}
\newcommand{\Ab}{\mbf{Ab}}
\newcommand{\GL}{\mathrm{GL}}
\newcommand{\SL}{\mathrm{SL}}
\newcommand{\bgCRing}{\mbf{bgCRing}}
\newcommand{\CRing}{\mbf{CRing}}
\newcommand{\AGrCRing}{A\mbf{GrCRing}}
\newcommand{\GrCAlg}{\mbf{GrCAlg}}
\newcommand{\GCA}{\mbf{GCA}}
\newcommand{\AGrCAlg}{A\mbf{GrCAlg}}
\newcommand{\GrCRing}{\mbf{GrCRing}}
\newcommand{\CStabRing}[1]{\mbf{CStabRing}_{#1}}
\newcommand{\GrAb}{\mbf{GrAb}}
\newcommand{\hoTop}{\mbf{hoTop}}
\newcommand{\hoCW}{\mbf{hoCW}}
\newcommand{\CW}{\mbf{CW}}
\newcommand{\SSet}{\mbf{SSet}}
\newcommand{\xr}{\xrightarrow}
\newcommand{\Sp}{\mbf{Sp}}
\newcommand{\hoSp}{\mbf{hoSp}}
\newcommand{\ho}{\mbf{ho}}
\newcommand{\SH}[1]{\mbf{SH}_{#1}}
\newcommand{\Ha}[1]{\mbf{H}_\ast\boldsymbol{(}{#1}\boldsymbol{)}}
\newcommand{\Sms}[1]{{\mbf{Sm}\boldsymbol{/}{#1}}}
\newcommand{\Spt}[1]{\mbf{Spt}_T\boldsymbol{(}{#1}\boldsymbol{)}}
\newcommand{\SptSig}[1]{\mbf{Spt}^\Sigma_T\boldsymbol{(}{#1}\boldsymbol{)}}
\newcommand{\Spc}[1]{{\mbf{Spc}\boldsymbol{(}{#1}\boldsymbol{)}}}
\newcommand{\Spca}[1]{\mbf{Spc}_\ast\boldsymbol{(}{#1}\boldsymbol{)}}
\newcommand{\Set}{\mbf{Set}}
\newcommand{\Map}{\mathrm{Map}}
\newcommand{\Sing}{\mathrm{Sing}}
\newcommand{\Grp}{\mbf{Grp}}
\newcommand{\Ord}{\mbf{Ord}}
\newcommand{\RMod}{R-\mbf{Mod}}
\newcommand{\op}{\mathrm{op}}
\newcommand{\alg}{\mathrm{alg}}
\newcommand{\tors}{\mathrm{tors}}

\makeatletter
\newcommand\xleftrightarrow[2][]{%
  \ext@arrow 9999{\longleftrightarrowfill@}{#1}{#2}}
\newcommand\longleftrightarrowfill@{%
  \arrowfill@\leftarrow\relbar\rightarrow}
\makeatother

\makeatletter
\newcommand{\xRightarrow}[2][]{\ext@arrow 0359\Rightarrowfill@{#1}{#2}}
\makeatother

\newcommand{\from}{\colon}
\newcommand{\sseq}{\subseteq}
\newcommand{\wt}{\widetilde}
\newcommand{\spseq}{\supseteq}
\newcommand{\brn}{\mathbb R^n}
\newcommand{\bRn}{\mathbb R^n}
\newcommand{\bP}{\mathbb P}
\newcommand{\bS}{\mathbb S}
\newcommand{\bA}{\mathbb A}
\newcommand{\bG}{\mathbb G}
\newcommand{\0}{\mathbf{0}}
\newcommand{\bR}{\mathbb{R}}
\newcommand{\cA}{\mathcal A}
\newcommand{\cB}{\mathcal B}
\newcommand{\cC}{\mathcal C}
\newcommand{\sk}{\mathrm{sk}}
\newcommand{\cD}{\mathcal D}
\newcommand{\id}{\mathrm{id}}
\newcommand{\Id}{\mathrm{Id}}
\newcommand{\cE}{\mathcal E}
\newcommand{\cF}{\mathcal F}
\newcommand{\cG}{\mathcal G}
\newcommand{\cH}{\mathcal H}
\newcommand{\cI}{\mathcal I}
\newcommand{\p}{{_\perp}}
\newcommand{\cJ}{\mathcal J}
\newcommand{\cK}{\mathcal K}
\newcommand{\cL}{\mathcal L}
\newcommand{\cM}{\mathcal M}
\newcommand{\cN}{\mathcal N}
\newcommand{\cO}{\mathcal O}
\newcommand{\cP}{\mathcal P}
\newcommand{\cQ}{\mathcal Q}
\newcommand{\into}{\hookrightarrow}
\newcommand{\onto}{\twoheadrightarrow}
\newcommand{\mono}{\rightarrowtail}
\newcommand{\cR}{\mathcal R}
\newcommand{\cS}{\mathcal S}
\newcommand{\cT}{\mathcal T}
\newcommand{\cU}{\mathcal U}
\newcommand{\cV}{\mathcal V}
\newcommand{\cW}{\mathcal W}
\newcommand{\cX}{\mathcal X}
\newcommand{\cY}{\mathcal Y}
\newcommand{\cZ}{\mathcal Z}
\newcommand{\mbf}[1]{\mathbf{#1}}
\renewcommand{\ol}{\overline}
\newcommand{\ul}{\underline}
\newcommand{\bZ}{\mathbb{Z}}
\newcommand{\dx}{\,\mathrm dx}
\newcommand{\dt}{\,\mathrm dt}
\newcommand{\bC}{\mathbb{C}}
\newcommand{\scS}{\mathscr{S}}
\newcommand{\scX}{\mathscr{X}}
\newcommand{\scU}{\mathscr{U}}
\newcommand{\bF}{\mathbb{F}}
\newcommand{\bN}{\mathbb{N}}
\newcommand{\bQ}{\mathbb{Q}}
\newcommand{\vare}{\varepsilon}
\renewcommand{\(}{\left(}
\renewcommand{\)}{\right)}
\newcommand\defeq{\mathrel{\overset{\makebox[0pt]{\mbox{\normalfont\tiny def}}}{=}}}
\newcommand{\phantomreplace}[2]{\makebox[0pt][l]{#1}\hphantom{#2}}
\newcommand{\phantommathreplace}[2]{\makebox[0pt][l]{$\displaystyle #1$}\hphantom{#2}}
\makeatletter
\newcommand{\skipitems}[1]{%
  \addtocounter{\@enumctr}{#1}%
}
\makeatother
\newcommand{\Cof}{\mathcal C\mathrm{of}}
\newcommand{\Fib}{\mathcal F\mathrm{ib}}
\newcommand{\W}{\mathcal W}
\newcommand{\inj}{\text-\mathrm{inj}}
\newcommand{\proj}{\text-\mathrm{proj}}
\newcommand{\fib}{\text-\mathrm{fib}}
\newcommand{\cell}{\text-\mathrm{cell}}
\renewcommand{\1}{\mbf{1}}
\newcommand{\cof}{\text-\mathrm{cof}}
\DeclareMathOperator*{\colim}{colim}
\DeclareMathOperator{\Aut}{Aut}
\DeclareMathOperator{\Emb}{Emb}
\DeclareMathOperator{\Irred}{Irred}
\DeclareMathOperator{\Prim}{Prim}
\DeclareMathOperator{\Frac}{Frac}
\DeclareMathOperator{\End}{End}
\DeclareMathOperator{\Syl}{Syl}
\DeclareMathOperator{\Inn}{Inn}
\DeclareMathOperator{\Out}{Out}
\DeclareMathOperator{\Sym}{Sym}
\DeclareMathOperator*{\hocolim}{hocolim}
\DeclareMathOperator*{\holim}{holim}
\DeclareMathOperator{\Mor}{Mor}

%\bibliographystyle{plain}

%--------Meta Data: Fill in your info------
\title{Algebra}

\author{Isaiah Dailey}

\date{\today}

\begin{document}

\maketitle

\tableofcontents

\section{Groups}

\begin{definition}
  A \emph{semigroup} is a set with an associative operation. A \emph{monoid} is
  a semigroup with an identity element. A \emph{group} is a monoid with
  inverses. An \emph{abelian} group is a commutative group.
\end{definition}

\begin{definition}
  For $n\geq 3$, write $D_{2n}$ for the dihedral group of order $2n$ with
  presentation $\left\langle r,s\mid r^n,s^2,rsrs^{-1}\right\rangle$. The
  elements of $D_{2n}$ are $e,r,\ldots,r^{n-1},s,sr,\ldots,sr^{n-1}$.
\end{definition}

\begin{definition}
  The quarternion group $Q_8$ has elements $\pm 1,\pm i,\pm j,\pm k$ with group structure given by
  \[
    -1\cdot x=-x\ \forall x\in Q_8,
    \quad
    (-1)^2=1,
    \quad
    ij=k,
    \quad
    jk=i,
    \quad
    ki=j.
  \]
\end{definition}

\begin{definition}
  A subset $H$ of a group $G$ is a \emph{subgroup} if $H$ is nonemtpy and
  $xy^{-1}\in H$ whenever $x,y\in H$.
\end{definition}

\begin{definition}
  The \emph{special linear group} of a field $F$ is the subgroup
  $\SL_n(F)\subseteq\GL_n(F)$ of matrices $A$ with $\det A=1$.
\end{definition}

\begin{definition}
  The \emph{alternating group} in $n$ elements is the subgroup $A_n\leq S_n$
  consisting of even permutations.\footnote{
    A permutation $\sigma\in S_n$ is said to be \emph{even} if $\sigma$ can be
    written as a composition of an even number of two-element swaps.
  }
  $A_n$ has order $n!/2$.
\end{definition}

\begin{definition}
  Let $H\leq G$ be a subgroup, then we write $G/H$ (resp.\ $H\backslash G$) for
  the set of left (resp.\ right) cosets of $H$ in $G$.
\end{definition}

\begin{proposition}
  Let $H\leq G$.
  \begin{enumerate}[listparindent=\parindent,parsep=5pt,label={(\arabic*)}]
    \item For any $x,y\in G$, there is a bijection $xH\to yH$ given by
    $xh\mapsto yh$.
    \item For any $x\in G$, there is a bijection $xH\to Hx^{-1}$ defined by
    $xh\mapsto h^{-1}x^{-1}$.
    \item There is a bijection $G/H\to H\backslash G$ given by $xH\mapsto
    Hx^{-1}$.
  \end{enumerate}
\end{proposition}

\begin{definition}
  Given a subgroup $H$ of a group $G$, we define the index of $H$ in $G$ to be
  the quantity $|G:H|:=|G/H|=|H\backslash G|$.
\end{definition}

\begin{proposition}
  Given a subgroup $H\leq G$, we have $|G|=|G:H|\cdot|H|$. More generally, if
  $K\leq H\leq G$, we have $|G:K|=|G:H|\cdot|H:K|$.
\end{proposition}

\begin{theorem}[Lagrange's Theorem]
  If $G$ is a finite group and $H\leq G$, then $|H|$ and $|G:H|$ divide
  $|G|$. In particular, $|g|:=|\langle g\rangle|$ divides $|G|$ for all $g\in
  G$.
\end{theorem}

As a consequence of Lagrange's theorem, if $|G|$ is prime then $G$ is cyclic.

\begin{example}
  $S_3$ and $D_6$ are isomorphic, given by $\phi:D_6\to S_3$ given by
  $\phi(r)=(1\,2\,3)$ and $\phi(s)=(1\,2)$.
\end{example}

\begin{definition}
  A subgroup $H\leq G$ is said to be \emph{normal} if $xHx^{-1}=H$ for all
  $x\in G$, equivalently, if $xH=Hx$ for all $x\in G$. We write $H\unlhd G$ to
  mean $H$ is a normal subgroup of $G$.
\end{definition}

\begin{warning}
  The relation $\unlhd$ is NOT a transitive relation on subgroups!
\end{warning}

\begin{definition}
  If $H\unlhd G$, then $G/H$ be comes a group by the operation $xH\cdot
  yH=xyH$.
\end{definition}

\begin{proposition}
  A subgroup $H\leq G$ is normal iff it is the kernel of some homomorphism.
\end{proposition}

\begin{proposition}
  Let $G$ be a group with subgroups $A,B\leq G$, then their intersection $A\cap
  B$ is also a subgroup.
\end{proposition}

\begin{definition}
  Let $G$ be a group with subgroups $A,B\leq G$, then define
  \[
    AB:=\left\{ab\in G\mid a\in A,\,b\in B\right\}.
  \]
  The set $AB$ is \emph{not} generally a subgroup.
\end{definition}

\begin{example}
  Consider $G=D_6$ generated by $\{r,s\}$ with $r^3=s^2=(sr)^2=1$. Let
  $A=\langle s\rangle$ and $B=\langle sr\rangle$, both subgroups of order
  $2$. Then $AB=\{e,s,sr,r\}$, which is not a subgroup since $r^2\notin AB$.
\end{example}

\begin{exercise}
  Show that $AB$ is a subgroup of $G$ iff $AB=BA$.
\end{exercise}

\begin{definition}
  Given a subset $S\subseteq G$, we write $N_G(S)$ for the \emph{normalizer} of $S$ in $G$, that is,
  \[
    N_G(S):=\{g\in G\mid gSg^{-1}=S\}.
  \]
\end{definition}

\begin{proposition}
  Let $S\subseteq G$, then
  \begin{itemize}
    \item $N_G(S)$ is a subgroup of $G$.
    \item If $H\leq G$ is a subgroup, then $H\unlhd N_G(H)$.
    \item $N_G(H)$ is the ``largest'' subgroup of $G$ that $H$ is normal inside of.
    \item $N_G(H)=G$ iff $H\unlhd G$.
  \end{itemize}
\end{proposition}

\begin{theorem}[The Second (``Diamond'') Isomorphism Theorem]
  Suppose $A,B\leq G$ and $A\leq N_G(B)$. Then
  \begin{enumerate}[listparindent=\parindent,parsep=5pt,label={(\arabic*)}]
    \item $AB$ is a subgroup of $G$ (equivalently, $AB=BA$).
    \item $B\unlhd AB$,
    \item $A\cap B\unlhd A$,
    \item $A/(A\cap B)\cong AB/B$.
  \end{enumerate}
  % https://q.uiver.app/#q=WzAsOCxbMCwzLCJBL0FcXGNhcCBCIl0sWzAsMSwiQUIvQiJdLFsyLDQsIkFcXGNhcCBCIl0sWzEsMywiQSJdLFsyLDIsIkFCIl0sWzIsMSwiTl9HKEIpIl0sWzMsMywiQiJdLFsyLDAsIkciXSxbMCwxLCJcXGNvbmciXSxbMiwzLCJcXHVubGhkIiwyLHsic3R5bGUiOnsidGFpbCI6eyJuYW1lIjoiaG9vayIsInNpZGUiOiJ0b3AifX19XSxbMyw0LCIiLDAseyJzdHlsZSI6eyJ0YWlsIjp7Im5hbWUiOiJob29rIiwic2lkZSI6InRvcCJ9fX1dLFs0LDUsIiIsMCx7InN0eWxlIjp7InRhaWwiOnsibmFtZSI6Imhvb2siLCJzaWRlIjoidG9wIn19fV0sWzIsNiwiIiwyLHsic3R5bGUiOnsidGFpbCI6eyJuYW1lIjoiaG9vayIsInNpZGUiOiJ0b3AifX19XSxbNiw0LCJcXHVubGhkIiwyLHsic3R5bGUiOnsidGFpbCI6eyJuYW1lIjoiaG9vayIsInNpZGUiOiJ0b3AifX19XSxbNiw1LCJcXHVubGhkIiwyLHsiY3VydmUiOjIsInN0eWxlIjp7InRhaWwiOnsibmFtZSI6Imhvb2siLCJzaWRlIjoidG9wIn19fV0sWzMsNSwiIiwxLHsiY3VydmUiOi0yLCJzdHlsZSI6eyJ0YWlsIjp7Im5hbWUiOiJob29rIiwic2lkZSI6InRvcCJ9fX1dLFs1LDcsIiIsMCx7InN0eWxlIjp7InRhaWwiOnsibmFtZSI6Imhvb2siLCJzaWRlIjoidG9wIn19fV1d
  \[\begin{tikzcd}
    && G \\
    {AB/B} && {N_G(B)} \\
    && AB \\
    {A/A\cap B} & A && B \\
    && {A\cap B}
    \arrow[hook, from=2-3, to=1-3]
    \arrow[hook, from=3-3, to=2-3]
    \arrow["\cong", from=4-1, to=2-1]
    \arrow[curve={height=-12pt}, hook, from=4-2, to=2-3]
    \arrow[hook, from=4-2, to=3-3]
    \arrow["\unlhd"', curve={height=12pt}, hook, from=4-4, to=2-3]
    \arrow["\unlhd"', hook, from=4-4, to=3-3]
    \arrow["\unlhd"', hook, from=5-3, to=4-2]
    \arrow[hook, from=5-3, to=4-4]
  \end{tikzcd}\]
\end{theorem}

\begin{corollary}
  If $A\leq G$ and $B\unlhd G$, then $AB$ is a subgroup of $G$.
\end{corollary}

\begin{theorem}[The Third Isomorphism Theorem]
  Let $H,K\unlhd G$ with $H\leq K$. Then
  \begin{enumerate}[listparindent=\parindent,parsep=5pt,label={(\arabic*)}]
    \item $K/H\unlhd G/H$, and
    \item $G/K\cong(G/H)/(K/H)$ via the assignment $xK\mapsto(xH)\overline{K}$
    (where $\overline{K}=K/H\subseteq G/H$).
  \end{enumerate}
  % https://q.uiver.app/#q=WzAsNyxbMCwyLCJIIl0sWzEsMiwiSyJdLFsyLDIsIksvSCJdLFsyLDEsIkcvSCJdLFsyLDAsIihHL0gpLyhHL0spIl0sWzEsMSwiRyJdLFsxLDAsIkcvSyJdLFswLDEsIlxcdW5saGQiLDAseyJzdHlsZSI6eyJ0YWlsIjp7Im5hbWUiOiJob29rIiwic2lkZSI6InRvcCJ9fX1dLFsxLDIsIiIsMCx7InN0eWxlIjp7ImhlYWQiOnsibmFtZSI6ImVwaSJ9fX1dLFsyLDMsIlxcdW5saGQiLDIseyJzdHlsZSI6eyJ0YWlsIjp7Im5hbWUiOiJob29rIiwic2lkZSI6InRvcCJ9fX1dLFszLDQsIiIsMCx7InN0eWxlIjp7ImhlYWQiOnsibmFtZSI6ImVwaSJ9fX1dLFswLDUsIlxcdW5saGQiLDAseyJzdHlsZSI6eyJ0YWlsIjp7Im5hbWUiOiJob29rIiwic2lkZSI6InRvcCJ9fX1dLFs1LDMsIiIsMCx7InN0eWxlIjp7ImhlYWQiOnsibmFtZSI6ImVwaSJ9fX1dLFs1LDYsIiIsMix7InN0eWxlIjp7ImhlYWQiOnsibmFtZSI6ImVwaSJ9fX1dLFs2LDQsIlxcY29uZyJdLFsxLDUsIlxcdW5saGQiLDIseyJzdHlsZSI6eyJ0YWlsIjp7Im5hbWUiOiJob29rIiwic2lkZSI6InRvcCJ9fX1dXQ==
  \[\begin{tikzcd}
    & {G/K} & {(G/H)/(G/K)} \\
    & G & {G/H} \\
    H & K & {K/H}
    \arrow["\cong", from=1-2, to=1-3]
    \arrow[two heads, from=2-2, to=1-2]
    \arrow[two heads, from=2-2, to=2-3]
    \arrow[two heads, from=2-3, to=1-3]
    \arrow["\unlhd", hook, from=3-1, to=2-2]
    \arrow["\unlhd", hook, from=3-1, to=3-2]
    \arrow["\unlhd"', hook, from=3-2, to=2-2]
    \arrow[two heads, from=3-2, to=3-3]
    \arrow["\unlhd"', hook, from=3-3, to=2-3]
  \end{tikzcd}\]
\end{theorem}

Intuitively, the following theorem says the following: Let $N\unlhd G$ be a
normal subgroup, then the quotient map $\pi:G\twoheadrightarrow G/N$ induces a
lattice isomorphism (an inclusion-preserving bijection) between the set of
subgroups of $G$ containing $N$, and the set of subgroups of $G/N$. Moreover,
this isomorphism restricts to an isomorphism on the normal subgroups, and given
subgroups $A,B\leq G$ with $N\leq A\cap B$, we have $(A\cap
B)/N=(A/N)\cap(B/N)$.

\begin{theorem}[The Fourth (``Lattice'') Isomorphism Theorem]
  Let $N\unlhd G$ be a normal subgroup. Then we have inverse bijections
  % https://q.uiver.app/#q=WzAsNixbMCwwLCJcXHtBXFxsZXEgR1xcbWlkIE5cXGxlcSBBXFx9Il0sWzIsMCwiXFx7XFxvdmVybGluZXtBfVxcbGVxIEcvTlxcfSJdLFswLDEsIkEiXSxbMiwxLCJBL04iXSxbMCwyLCJcXHBpXnstMX1cXG92ZXJsaW5lIEEiXSxbMiwyLCJcXG92ZXJsaW5lIEEiXSxbMCwxLCJcXHNpbSIsMCx7InN0eWxlIjp7InRhaWwiOnsibmFtZSI6ImFycm93aGVhZCJ9fX1dLFsyLDMsIiIsMCx7InN0eWxlIjp7InRhaWwiOnsibmFtZSI6Im1hcHMgdG8ifX19XSxbNSw0LCIiLDAseyJzdHlsZSI6eyJ0YWlsIjp7Im5hbWUiOiJtYXBzIHRvIn19fV1d
  \[\begin{tikzcd}
    {\{A\leq G\mid N\leq A\}} && {\{\overline{A}\leq G/N\}} \\
    A && {A/N} \\
    {\pi^{-1}\overline A} && {\overline A}
    \arrow["\sim", tail reversed, from=1-1, to=1-3]
    \arrow[maps to, from=2-1, to=2-3]
    \arrow[maps to, from=3-3, to=3-1]
  \end{tikzcd}\]
  where $\pi^{-1}\overline{A}=\{g\in G\mid\pi(g)\in\overline{A}\}$. Furthermore, for $A,B\leq G$ with $N\leq A\cap B$, we have
  \begin{enumerate}[listparindent=\parindent,parsep=5pt,label={(\arabic*)}]
    \item $A\leq B$ iff $A/N\leq B/N$.
    \item If $A\leq B$ then $|B:A|=|B/N:A/N|$.
    \item $(A\cap B)/N=(A/N)\cap(B/N)$.
    \item $A\unlhd G$ iff $A/N\unlhd G/N$.
  \end{enumerate}
\end{theorem}

\begin{definition}
  A \textbf{group presentation} is a pair $(S,R)$ consisting of a set $S$ and a subset $R\subseteq F(S)$ (where $F(S)$ denotes the free group on $S$). The group \emph{presented} by this data is defined to be
  \[
    \langle S\mid R\rangle:=F(S)/N,
  \]
  where $N$ is the \emph{normal closure} of $R$ in $F(S)$, that is, the
  smallest normal subgroup of $F(S)$ containing $R$. 

  Given a group $G$, we say that $(S,R)$ is a \emph{presentation} of $G$ if
  there exists an isomorphism $G\cong\langle S\mid R\rangle$ of groups. We say that $G$ is \emph{finitely presentable} if it has a presentation $(S,R)$, where $S$ and $R$ are both finite.
\end{definition}

\begin{example}
  The dihredral group $D_{2n}$ of order $2n$ has presentation $\langle r,s\mid
  r^n,s^2,srsr\rangle$.
\end{example}

\begin{proposition}
  For $n\geq 1$, we have
  \[
    S_n\cong\langle s_1,\ldots,s_{n-1}\mid R\rangle,
  \]
  where $R$ consists of the relations
  \[
    \begin{split}
      s_i^2&=1, \\
      (s_is_j)^2&=1, \\
      (s_is_{i+1})^3&=1,
    \end{split}\qquad\qquad\begin{split}
      \text{for }i=1,\ldots,n-1, \\
      \text{when }|i-j|\geq 2, \\
      \text{for }i=1,\ldots,n-1.
    \end{split}
  \]
\end{proposition}

\begin{definition}
  Given an object $X$ in a category $\mathcal{C}$ and a group $G$, a \emph{left
  group action} of $G$ on $X$ is a group homomorphism $\phi:G\to\Aut(X)$,
  denoted by $G\acts X$. A right group action is a group homomorphism
  $\phi:G^\op\to\Aut(X)$.
\end{definition}

\begin{definition}
  A set equipped with a $G$-action is called a \emph{$G$-set}.
\end{definition}

\begin{example}
  For any object $X$ and group $G$, the trivial map $G\to\Aut(X)$ yields the
  \emph{trivial action} of $G$ on $X$, in which $G$ simply acts via identities
  on $X$.
\end{example}

\begin{example}
  Given $H\leq G$, the set $G/H$ of left cosets of $H$ admits a natural $G$ action by
  \[
    g\cdot xH:=gxH.
  \]
  Similarly, the set $H\backslash G$ of right cosets of $H$ admits a natural
  $G$ action by the rule
  \[
    g\cdot Hx:=Hxg^{-1}.
  \]
\end{example}

\begin{example}
  Every group acts on itself by conjugation via the map $\conj:G\to\Aut(G)$
  defined by
  \[
    \conj_g(x):=gxg^{-1}.
  \]
\end{example}

\begin{definition}
  Let $\phi:G\to\Aut(X)$ be a left $G$-action on an object $X$.
  \begin{enumerate}[listparindent=\parindent,parsep=5pt,label={(\arabic*)}]
    \item The \emph{kernel} of the action is the kernel of the homomorphism
    $\phi$, i.e., it is the set $\{g\in G\mid \phi_g=\id_X\}$.
    \item The action is \emph{faithful} if the kernel is trivial.
  \end{enumerate}
  If $X$ is a set, then we have the following further definitions.
  \begin{enumerate}[listparindent=\parindent,parsep=5pt,label={(\arabic*)}]
    \item Given $x\in X$, the \emph{stabilizer} of $x$ (denoted by $\Stab(x)$
    or just $G_x$) is the set $\{g\in G\mid g\cdot x=x\}$.
    \item The action is \emph{free} if all the stabilizers $G_x$ are trivial.
  \end{enumerate}
\end{definition}

\begin{proposition}
  Suppose $X$ is a $G$-set, and $x,y\in X$ satisfying $y=g\cdot x$ for some $g\in G$. Then 
  \[
    G_y=gG_xg^{-1}.
  \]
\end{proposition}

\begin{example}
  Consider the tautological action of $G=S_n$ on $X=\{1,\ldots,n\}$, so the
  corresponding homomorphism $G\to\Sym(X)$ is the identity. We have that:
  \begin{itemize}
    \item The kernel of the action is trivial, so it is a faithful action.
    \item The action is free iff $n\geq 3$.
    \item If $n>1$, each $G_x$ is isomorphic to $S_{n-1}$, but each is a
    \emph{distinct} subgroup of $S_n$.
    \item The $G_x$ are conjugate to each other: if $\sigma\in S_n$ such that
    $\sigma(x)=y$, then $G_y=\sigma G_x\sigma^{-1}$.
  \end{itemize}
\end{example}

\begin{theorem}[Cayley's Theorem]
  Every group is isomorphic to a subgroup of some permutation group $\Sym(X)$.
\end{theorem}
\begin{proof}
  Given $G$, it suffices to provide a faithful action on some set $X$, so that
  the induced homomorphism $\phi:G\to\Sym(X)$ is injective, and therefore
  identifies $G$ with a subgroup of $\Sym(X)$. This is easy: equip $X=G$ with
  the natural left $G$ action given by $g\cdot x:=gx$. Then this action is
  faithful, since $gx=x$ for all $x\in X$ certainly implies $g=e$.
\end{proof}

\begin{proposition}\label{index_subgroup_normal_prime}
  If $G$ is a finite group and $p$ is the smallest prime dividing $|G|$, then
  any subgroup of index $p$ is normal. In particular, index $2$ subgroups of
  finite groups are always normal.
\end{proposition}
\begin{proof}
  Let $H\leq G$ be a subgroup of index $p$, and consider the left action of $G$
  on $X=G/H$, which gives a homomorphism $\phi:G\to\Sym(G/H)\cong S_p$. Let
  $K=\ker\phi$ of this action. We know $K$ is normal, since it is a kernel, so
  it suffices to show that $K=H$. Note that clearly $K\leq H$, so it further
  suffices to show that $|H:K|=1$. By the first isomorphism theorem, $G/K$ is
  isomorphic to a subgroup of $S_p$, so that $|G:K|$ divides $|S_p|=p!$, by
  Lagrange's theorem. We have that $|G:K|=|G:H||H:K|=p|H:K|$, so $|H:K|$
  divides $p!/p=(p-1)(p-2)\cdots 2\cdot 1$. However, since $|H:K|$ divides
  $|G|$, we  know that no prime smaller than $p$ divides $|H:K|$. Thus
  $|H:K|=1$, as desired.
\end{proof}

\begin{definition}
  Consider a group action $G\acts X$, where $X$ is a set. Define a relation $\sim$ on $X$ by
  \[
    x\sim y\iff\exists g\in G,\ g\cdot x=y.
  \]
  This is an equivalence relation on $X$, and the equivalence classes of this
  relations are called \emph{orbits}. We write $\Orb(x)$, $Gx$, or $G\cdot x$
  for the orbit which contains $x$, so that $\Orb(x)=\{g\cdot x\mid g\in G\}$.

  An action is \emph{transitive} if it has exactly one orbit.
\end{definition}

\begin{example}
  $G$ acts transitively on $G/H$.
\end{example}

\begin{theorem}[The Orbit/Stabilizer Theorem]
  Suppose $X$ is a $G$-set, and $x\in X$. Then there is a bijection
  \[
    G/\Stab(x)\xrightarrow{\sim}\Orb(x),
    \qquad
    g\Stab(x)\mapsto g\cdot x.
  \]
  Thus for an orbit $\mathcal{O}$, we have $|\mathcal{O}|=|G:\Stab(x)|$ for any
  $x\in\mathcal{O}$.
\end{theorem}

\begin{corollary}
  Let $G$ act on a finite set $X$. Then we have
  \[
    |X|=\sum_{k=1}^{r}|G:\Stab(x_k)|,
  \]
  where $x_1,\ldots,x_r\in X$ are representatives of the orbits of the action (that is, $\Orb(x_i)\cap\Orb(x_j)=\emptyset$ when $i\neq j$, and $\bigcup_{k=1}^r\Orb(x_k)=X$),
\end{corollary}

\begin{theorem}[Cauchy's Theorem]
  Let $G$ be a finite group. If a prime $p$ divides $|G|$, then $G$ has an
  element of order $p$.
\end{theorem}

\begin{definition}
  A group $G$ is \emph{simple} if its only normal subgroups are $\{e\}$ and
  $G$. By convention the trivial group is \emph{not} simple.
\end{definition}

\begin{example}
  Let $p$ be a prime. Then the cyclic group $G=C_p$ of order $p$ is simple.
\end{example}

\begin{proposition}
  The alternating group $A_n$ on $n$ elements is simple for $n\geq 5$.
\end{proposition}
\begin{proof}[Proof sketch]
  Elements of $A_n$ are the even permutations, and it is straightforward to
  check that $A_n$ is also generated by its subset of $3$-cycles. Then one
  checks that any normal subgroup $N$ of $A_n$ which contains some $3$-cycle
  contains every $3$-cycle, and therefore satisfies $N=A_n$.

  Thus, in order to prove $A_n$ is simple, it suffices to show that if $N\unlhd
  A_n$ is a non-trivial normal subgroup, it must contain at least one
  $3$-cycle. This is where the assumption that $n\geq 5$ is needed.
\end{proof}

\begin{example}
  The group $A_4$ is not simple: the subgroup
  $N=\{e,(1\,2)(3\,4),(1\,3)(2\,4),(1\,4)(2\,3)\}$ generated by the products of
  disjoint $2$-cycles is normal.
\end{example}

\begin{definition}
  Consider the conjugation action of $G$ on itself: $\cong_g(x)=gxg^{-1}$.
  \begin{itemize}
    \item The \emph{orbits} for the conjugation action are the conjugacy
    classes; we denote the conjugacy class of an element $x\in G$ by
    $\Cl(x):=\{gxg^{-1}:g\in G\}$.
    \item The \emph{stabilizer} of $x\in G$ under the conjugation action is the
    \emph{centralizer subgroup} of $x$:
    \[
      C_G(x)
      :=\{g\in G\mid gxg^{-1}=x\}
      =\{g\in G\mid gx=xg\}.
    \]
    \item The kernel of the conugation action is precisely the \emph{center}
    \[
      Z_G:=\{g\in G\mid gx=xg\ \forall x\in G\}.
    \]
    \item Note that $\Cl(e)=\{e\}$ and $C_G(e)=G$, so that the conjugation
    action is neither free nor transitive (unless $G=\{e\}$).
  \end{itemize}
\end{definition}

\begin{theorem}[The Class Equation]
  For a finite group $G$, we have
  \[
    |G|=|Z_G|+\sum_{k=1}^{r}|G:C_G(g_k)|,
  \]
  where $g_1,\ldots,g_r$ are representatives of the distinct conjugacy classes of $G$ not contained in the center $Z_G$.

  Moreover, each term on the right divides $|G|$.
\end{theorem}

\begin{definition}
  Let $p$ be a prime. A \emph{$p$-group} is a non-trivial finite group whose
  order is a power of $p$.
\end{definition}

\begin{proposition}
  Every $p$-group has a non-trivial center.
\end{proposition}
\begin{proof}
  The class equation for $G$ gives
  \[
    p^d=|Z_G|+\sum_{k=1}^{r}|G:C_G(g_k)|.
  \]
  Since $C_G(g_k)\neq G$, we have that $p$ divides each
  $|G:C_G(g_k)|$. Therefore $p$ divides $|Z_G|$. Since $|Z_G|\geq 1$ we may
  conclude that $p$ divides $|Z_G|$.
\end{proof}

\begin{corollary}\label{p^2_order_group_is_abelian}
  If $|G|=p^2$ for some prime $p$ then $G$ is abelian.
\end{corollary}
\begin{proof}
  First we note a general fact: If $G/Z_G$ is cyclic, then $G$ is abelian. To
  see this, pick $g\in G$ which projects to a generator of $G/Z_G$. Then every
  element in $G$ can be written as $g^kx$ for some $k\in\mathbb{Z}$ and $x\in
  Z_G$. Then every element in $G$ can be written as $g^kx$ for some
  $k\in\mathbb{Z}$ and $x\in Z_G$. Since $(g^ix)(g^jy)=g^{i+j}xy$ whenever
  $x,y\in Z_G$, we see that $G$ is abelian.

  If $|G|=p^2$, then by the previous result $|Z_G|\in\{p,p^2\}$, whence
  $|G/Z_G|\in\{1,p\}$ and thus is cyclic.
\end{proof}

\begin{definition}
  Given a group $G$, the image of the homomorphism $\conj:G\to\Aut(G)$ is the
  group
  \[
    \Inn(G):=\{\conj_g\mid g\in G\}\leq\Aut(G),
  \]
  and its elements are called \emph{inner automorphisms} of $G$. The first isomorphism theorem then gives an isomorphism
  \[
    G/Z_G\cong\Inn(G).
  \]
\end{definition}

\begin{proposition}
  $\Inn(G)$ is a normal subgroup of $\Aut(G)$.
\end{proposition}

\begin{definition}
  The group of \emph{outer automorphisms} of $G$ is given by the quotient
  \[
    \Out(G):=\Aut(G)/\Inn(G).
  \]
\end{definition}

\begin{definition}
  Given a subgroup $H\leq G$, its \emph{centralizer} is the subgroup
  $C_G(H):=\{g\in G\mid gh=hg\ \forall h\in H\}$.
\end{definition}

\begin{proposition}
  Let $N\unlhd G$ be a normal subgroup, then the conjugation action of $G$ on
  $N$ yields a group homomorphism $\kappa:G\to\Aut(N)$. Then
  \[
    \kappa^{-1}(\Inn(N))=C_G(N)N,
  \]
  which is a normal subgroup of $G$.
\end{proposition}

\begin{remark}
  In the language of the above proposition, we know that $\kappa$ induces an injective homomorphism
  \[
    \overline{\kappa}:G/C_G(N)N\hookrightarrow\Out(N),
  \]
  so any elements of $G\setminus C_G(N)N$ give rise to non-inner automorphisms
  of $N$.
\end{remark}

\begin{proposition}
  $|\Aut(C_n)|=\phi(n)$, where $\phi$ is the \emph{Euler $\phi$ function} for
  which $\phi(n)$ is the number of integers in $\{1,\ldots,n\}$ which are
  relatively prime to $n$.
\end{proposition}

\begin{definition}
  A \emph{$p$-Sylow subgroup} of a finite group $G$ is a subgroup $P\leq G$
  which is a $p$-group, and is such that $|G:P|$ is prime to $p$. Equivalently,
  if $G=p^am$ with $(p,m)=1$ and $a\geq 1$, then a $p$-Sylow subgroup is a
  subgroup of order $p^a$.

  \textbf{Note:} With this convention, the trivial subgroup is not $p$-Sylow
  for any prime $p$.

  Write $\Syl_p(G)$ for the set of $p$-Sylow subgroups of $G$, and write
  $n_p(G):=|\Syl_p(G)|$. Note that $G$ acts on $\Syl_p(G)$ by conjugation: if
  $P\leq G$ is a $p$-Sylow subgroup, so is $gPg^{-1}$ for any $g\in G$.
\end{definition}

In the following three theorems, $p$ will be a chosen prime, and $G$ will be a
finite group of order $p^am$, where $a\geq 1$ and $p\nmid m$.

\begin{theorem}[Sylow 1]
  The group $G$ has a $p$-Sylow subgroup, i.e., $\Syl_p(G)\neq\emptyset$.
\end{theorem}

\begin{theorem}[Sylow 2]
  Any two $p$-Sylow subgroups of $G$ are conjugate, i.e., $G$ acts transitively
  on $\Syl_p(G)$ by conjugation.
\end{theorem}

\begin{theorem}[Sylow 3]
  If $P$ is any $p$-Sylow subgroup of $G$, then $n_p=|G:N_G(P)|$. Furthermore,
  $n_p|m$ and $n_p\equiv 1\bmod p$.
\end{theorem}

\begin{lemma}\label{product_subgroups_normal_prime_products_cyclic}
  Let $P,Q$ be subgroups of a group $G$ with $|P|=p$ and $|Q|=q$ prime and
  distinct. Further suppose that $PQ$ is a subgroup of $G$ (for example, if
  $P\subseteq N_G(Q)$ or $Q\subseteq N_G(P)$) and $ab=ba$ for all $a\in P$ and
  $b\in Q$. Then $PQ$ is isomorphic to the cyclic group $C_{pq}$ of order $pq$.
\end{lemma}
\begin{proof}
  Since $P$ and $Q$ have prime order, we can write $P=\langle x\rangle$ and
  $Q=\langle y\rangle$ where $|x|=p$ and $|y|=q$. Set $z=xy$. If $z^k=e$, then
  $x^k=y^{-k}$ because $x$ and $y$ commute, so that $x^k\in P\cap Q$, which is
  trivial, since $|P\cap Q|$ has to divide both $p$ and $q$, which are distinct
  primes. Hence we must have $x^k=e=y^k$, meaning $|z|=pq$, and we see that
  $PQ$ is cyclic.
\end{proof}

\begin{proposition}\label{product_of_primes_order_cyclic}
  If $p<q$ are primes and $q\not\equiv 1\bmod p$, then every group of order
  $pq$ is cyclic.
\end{proposition}
\begin{proof}
  By Sylow 3, $n_q|p$ and $n_q\equiv 1\bmod q$. If $n_q>q$, then $n_q>p$, a
  contradiction of the fact that $n_q|p$. Hence we must have $n_q=1$. Let
  $Q\leq G$ be the unique $q$-Sylow subgroup of $G$. Note that since $n_q=1$
  and any conjugate of $Q$ is also a $q$-Sylow subgroup, we have that $Q$ is a
  normal subgroup of $G$. Since $|Q|=q$ is prime, we can write $Q=\langle
  y\rangle$, where $y$ has order $q$. Pick any subgroup $P\leq G$ of order $p$,
  and write $P=\langle x\rangle$. $P$ acts on $Q$ via conjugation, yielding a
  map $\kappa:P\to\Aut(Q)$; the order of $\kappa(P)$ must divide both $|P|=p$
  and $|\Aut(Q)|=q-1$ by Lagrange's, and clearly $|\kappa(P)|\leq|P|=p$, so
  that $|\kappa(P)|\in\{1,p\}$. Since $q\not\equiv 1\bmod p$, $p$ does not
  divide $q-1$, so that we must have $|\kappa(P)|=1$, meaning
  $\kappa(P)=\{e\}$. Therefore $ab=ba$ for all $a\in P$ and $b\in Q$. It then
  follows by \autoref{index_subgroup_normal_prime} and
  \autoref{product_subgroups_normal_prime_products_cyclic} that $PQ$ is a
  subgroup of $G$ which is isomorphic to $C_{pq}$. Since $|G|=pq$, it follows
  that $G=PQ\cong C_{pq}$, as desired.
\end{proof}

\begin{proposition}
  If $|G|=30$, $G$ has unique $3$- and $5$-Sylow subgroups and contains a
  normal subgroup isomorphic to $C_{15}$.
\end{proposition}
\begin{proof}
  By the Sylow theorems, $n_3|10$, $n_3\equiv 1\bmod 3$, $n_5|6$, and
  $n_5\equiv 1\bmod 5$. Thus $n_3\in\{1,10\}$ and $n_5\in\{1,6\}$. If $n_3=10$
  and $n_5=6$, then since each subgroup in $\Syl_3$ and $\Syl_5$ are cyclic of
  prime order, there would be at least $2\cdot 10=20$ distinct order $3$
  elements in $G$, and $4\cdot 6=24$ distinct order $5$ elements in $G$, an
  impossibility since $|G|=30<44$. Thus, one of $n_3$ and $n_5$ is $1$. Let
  $P\in\Syl_3$ and $Q\in\Syl_5$, so that since $n_3=1$ or $n_5=1$, at least one
  of $P$ or $Q$ is normal in $G$, so that by the second isomorphism theorem we
  know that $PQ$ is a subgroup of $G$. Moreover, $|PQ|\leq 15$ and $3$ and $5$
  divide $|PQ|$, so we must have $|PQ|=15$. Thus $PQ$ is an index $2$ subgroup
  of $G$, so $PQ$ is normal in $G$. By
  \autoref{product_of_primes_order_cyclic}, since $|PQ|=15=3\cdot 5$ and
  $5\not\equiv 1\bmod 3$, we have that $PQ$ is cyclic, as desired. Sylow 3
  directly gives that $n_3(PQ)=n_5(PQ)=1$, so that $P$ and $Q$ are the unique
  $5$- and $3$-Sylow subgroups in $PQ$. We claim this implies
  $n_3(G)=n_5(G)=1$. If we had $n_3(G)=10$, then $G$ would have at least
  $2\cdot 10=20$ distinct order $3$ elements, so that in particular $PQ$ would
  have to contain at least $5$ elements of order $3$, meaning $PQ$ would have
  to contain at least $\lceil 5/2\rceil=3$ subgroups of order $3$, a
  contradiction of the fact that $n_3(PQ)=1$. A similar argument yields that
  $n_5(G)=1$, as desired.
\end{proof}

\begin{proposition}
  Let $G$ be a group of order $12$. If $G$ does not have a normal $3$-Sylow
  subgroup, then $G\cong A_4$.
\end{proposition}
\begin{proof}
  We have that $n_3|4$ and $n_3\equiv 1\bmod 3$ by the third Sylow theorem, so
  either $n_3=1$ or $n_3=4$. Since $G$ does not have a normal $3$-Sylow
  subgroup, we must have that $n_3=4$. The group $G$ acts on $\Syl_3(G)$ by
  conjugation, yielding a homomorphism
  \[
    \phi:G\to\Sym(\Syl_3(G))\cong S_4.
  \]
  First we aim to show this map is injective, so that $G$ is isomorphic to a
  subgroup of order $12$ of $S_4$.

  If $P\in\Syl_3(G)$, then $|G:N_G(P)|=n_3=4$, meaning $|N_G(P)|=3$, so that
  $P=N_G(P)$. The kernel of $\phi$ consists of the elements of $g$ which
  normalize all $3$-Sylow subgroups of $G$, and so are in the intersection of
  all $3$-Sylow subgroups. This implies $\ker\phi$ is trivial, so $\phi$ is
  injective, as desired.

  It remains to show that $\phi(G)=A_4$, which can be done in a number of
  ways. For instance, $G$ must contain exactly $8$ elements of order $3$, while
  there are exactly $8$ elements of order $3$ in $S_4$, and they generate
  $A_4$.
\end{proof}

\begin{proposition}
  Suppose $|G|=60$ and $n_5(G)>1$. Then $G$ is simple.
\end{proposition}
\begin{proof}
  By the third Sylow theorem, we have $n_5\in\{1,6\}$, so $n_5=6$ by
  assumption. Now, let $H$ be a non-trivial proper normal subgroup of $G$. We
  split into cases.

  \textbf{Case 1.} If $5$ divides $|H|$, then $H$ contains a $5$-Sylow
  subgroup; being normal, it must contain every $5$-Sylow subgroup of $G$. Thus
  $|H|\geq 1+4\cdot 6=25$, so $|H|=30$. But we have shown above that every group of order $30$ has a unique $5$-Sylow subgroup, so this is not possible.

  \textbf{Case 2.} If $5$ does not divide $|H|$, then $|H|$ divides $12$. Now
  we claim that $G$ must contain a normal subgroup of order $3$ or $4$. If $H$
  itself is not of one of these orders, then $|H|=6$ or $|H|=12$. If $|H|=6$
  (resp.\ $|H|=12$), then Sylow 3 yields $n_3(H)=1$ (resp.\ $n_4(H)=1$), so $H$
  admits a normal $3$-Sylow subgroup (resp.\ a normal $4$-Sylow
  subgroup). Since $H$ is normal, it follows that $n_3(G)=1$ (resp.\
  $n_4(G)=1$) as well, so indeed $G$ contains a normal subgroup of order $3$ or
  $4$, call it $K$.

  Now $G/K$ has order $15$ or $20$, and in each case Sylow 3 yields
  $n_5(G/K)=1$, so $G/K$ has a normal $5$-Sylow subgroup. By the fourth
  (lattice) isomorphism theorem, the preimage of such a $5$-Sylow subgroup with
  be a normal subgroup of $G$ with order divisible by $5$, contradicting the
  above.
\end{proof}

In general, subgroups of f.g.\ groups are not f.g.!

\begin{example}
  Let $G=F(a,b)$ be the free group on two generators. Write $x_n:=a^nba^{-n}\in
  G$, and let $H=\langle x_n,n\in\mathbb{Z}\rangle$. Then $H$ is not finitely
  generated.
\end{example} 

\begin{definition}
  A poset $(P,\leq)$ has the \emph{ascending chain condition (acc)} if, for
  every countable sequence ${(x_k)}_{k\in\mathbb{N}}$ with $x_k\leq x_{k+1}$,
  there exists $m$ such that $x_k=x_m$ for all $k\geq m$.

  A group $G$ has \emph{the ascending chain condition for subgroups} if the set
  of subgroups ordered by inclusion has the acc.
\end{definition}

\begin{proposition}
  TFAE
  \begin{enumerate}[listparindent=\parindent,parsep=5pt,label={(\arabic*)}]
    \item $G$ has the acc for subgroups
    \item All subgroups of $G$ are f.g.
  \end{enumerate}
\end{proposition}

\begin{proposition}
  Let $N\unlhd G$. TFAE
  \begin{enumerate}[listparindent=\parindent,parsep=5pt,label={(\arabic*)}]
    \item $G$ has the acc for subgroups
    \item Both $N$ and $G/N$ have the acc for subgroups.
  \end{enumerate}
\end{proposition}

\begin{proposition}
  Every f.g.\ abelian group has the acc for subgroups. In particular, every
  subgroup of a f.g.\ abelian group is also f.g.
\end{proposition}

\begin{definition}
  Let $G$ be a group. We say that an element $a\in G$ is \emph{torsion} if it
  has finite order, and write $G_\tors\subseteq G$ for the subset of torsion
  elements. A group $G$ is \emph{torsion free} if $G_\tors=\{e\}$. A group $G$
  is \emph{torsion} if $G_.\tors=G$
\end{definition}

\begin{proposition}
  If $G$ is an abelian group then $G_\tors$ is a subgroup of $G$.
\end{proposition}

\begin{proposition}
  If $G$ is abelian, then $G/G_\tors$ is torsion free.
\end{proposition}

\begin{proposition}
  Every f.g.\ torsion abelian group is finite.
\end{proposition}

\begin{proposition}[Product Recognition]\label{product_recognition}
  Let $G$ be a group, and suppose $G_1,\ldots,G_n\unlhd G$ are \emph{normal} subgroups such that
  \begin{enumerate}[listparindent=\parindent,parsep=5pt,label={(\arabic*)}]
    \item $G_1\cdots G_n=G$, and
    \item $G_k\cap(G_1\cdots G_{k-1}G_{k+1}\cdots G_n)=\{e\}$ for
    $k=1,\ldots,n$.
  \end{enumerate}
  Then the function
  \[
    \phi:G_1\times\cdots\times G_n\to G
    \qquad
    (g_1,\ldots,g_n)\mapsto g_1\cdots g_n
  \]
  is an isomorphism of groups.
\end{proposition}

\begin{proposition}
  If $G=G_1\times\cdots\times G_n$ and $N_k\unlhd G_k$ for $k=1,\ldots,n$, then $N=N_1\times\cdots\times N_n$ is a normal subgroup of $G$, and there is an isomorphism
  \[
    G/N\cong(G_1/N_1)\times\cdots\times(G_n/N_n).
  \]
\end{proposition}

\begin{theorem}
  Every f.g.\ abelian group $G$ is isomorphic to one of the form
  \[
    G\cong F\times \mathbb{Z}^r,
    \qquad
    |F|<\infty,
    \qquad
    \mathbb{Z}^r=\underbrace{\mathbb{Z}\times\cdots\times\mathbb{Z}}_{\text{$r$ copies}},
    \qquad
    r\geq 0.
  \]
  The factors are unique, in the sense that if $G$ admits two such isomorphisms
  $G\cong F\times\mathbb{Z}^r\cong F'\times\mathbb{Z}^{r'}$, then $F\cong F'$
  and $r=r'$.
\end{theorem}

\begin{theorem}
  Every finite abelian group $G$ is isomorphic to one of the form
  \[
    G\cong\mathbb{Z}/n_1\times\cdots\mathbb{Z}/n_s,
  \]
  where 
  \begin{itemize}
    \item $s\geq 0$,
    each $n_i\geq 2$,
    $n_{i+1}\mid n_i$ for all $i=1,\ldots,s-1$.
  \end{itemize}
  Furthermore, the decomposition is unique up to isomorhpism of the factors.
\end{theorem}

\begin{definition}
  A complete set of invariants for a f.g.\ abelian group
  \[
    G\cong\mathbb{Z}/n_1\times\cdots\times\mathbb{Z}/n_s\times\mathbb{Z}^r
  \]
  are the free rank $r$ and the list $n_1,\ldots,n_s$ of invariant factors. $G$
  is finite iff $r=0$.
\end{definition}

\begin{theorem}[Elementary divisor decomposition]
  For every finite abelian group $G$ of order $n=p_1^{a_1}\cdots p_k^{a_k}$, where the $p_1<\cdots<p_k$ are distinct primes, there is
  \begin{enumerate}[listparindent=\parindent,parsep=5pt,label={(\arabic*)}]
    \item an isomorphism $G\cong A_1\times\cdots\times A_k$, with $|A_i|=p_i^{a_i}$ and $a_i\geq 1$, such that
    \item for each $A_i$, there is an isomorphism
    \[
      A_i\cong\mathbb{Z}/p_i^{b_{i1}}\times\cdots\times\mathbb{Z}/p_i^{b_{is_i}},
    \]
    with $b_{i 1}\geq\cdots\geq b_{is_i}$ and $b_{i 1}+\cdots+b_{is_i}=a_i$.
  \end{enumerate}
  Furthermore, this decomposition is unique, in the sense that if $G$ admits
  isomorphisms $G\cong B_1\times\cdots\times B_\ell$ with $|B_i|=a_i^{a_i}$
  with $q_i$ prime and $a_j\geq 1$, then $k=\ell$, $p_i=q_i$, and $A_i\cong
  B_i$.
\end{theorem}

\begin{definition}
  The decomposition described in (1) above is called the \emph{primary
  decomposition} of $G$. Part (2) is just giving the invariant factor
  decomposition of each $A_i$.

  The list of numbers $p_1^{b_{11}},\ldots p^{b_{ks_k}}$ are the
  \emph{elementary divisors} of the group $G$. The list of elementary divisors
  is a complete isomorphism invariant of a finite abelian group $G$.
\end{definition}

\begin{definition}
  Let $H,K,G$ be groups. We say that $G$ is an \emph{extension} of $K$ by $H$
  if there exists a normal subgroup $H'\unlhd G$ and isomorphisms $H\cong H'$
  and $K\cong G/H'$, equivalently, an exact sequence of groups
  \[
    0\to H\to G\to K\to 0.
  \]
  The extension is \emph{split} if there is aditionally a subgroup $K'\leq G$
  such that the map $K'\to G/H'$ sending $x\mapsto xH'$, equivalently, if there
  is a homomorphism $s:K\to G$ such that $p\circ s=\id_K$ (in which case
  $K'=s(K)$).
\end{definition}

\begin{example}
  Given groups $K$ and $H$, you can always extend $K$ by $H$ via the
  \emph{trivial extension}, defined by
  \[
    G:=H\times K,
    \qquad
    H':=H\times\{e\}..
  \]
  The trivial extension is always split, by $K'=\{e\}\times K$.
\end{example}

\begin{example}
  Let $H=K=C_2$. Then both $G_1=C_2\times C_2$ and $G_2=C_4$ are extensions of $K$ by $H$
  \[
    H':=\{e,a\}\unlhd G_1=C_2\times C_2=\langle a\mid a^2\rangle\times\langle b\mid b^2\rangle=\{e,a,b,ab\},
    \qquad
    G_1/H'=\{\overline{e},\overline{b}\},
  \]
  and
  \[
    H'=\{e,c^2\}\unlhd G_2=C_4=\langle c\mid c^4\rangle=\{e,c,c^2,c^3\},
    \qquad
    G-2/H'=\{\overline{e},\overline{c}\}.
  \]
  The first extension is split, using $K'=\{e,b\}\leq G_1$, but the second
  extension is not split.
\end{example}

\begin{definition}
  The \emph{extension problem} for groups is to classify, for given $H$ and
  $K$, all possible extensions of $K$ by $H$, up to isomorphism.
\end{definition}

\begin{theorem}
  Let $H,K$ be groups, and $\alpha:K\to\Aut(H)$ a homomorphism. Let $G$ be the set $H\times K$, and define a product on $G$ by the rule
  \[
    (h_1,k_1)(h_2,k_2):=(h_1\alpha(k_1)(h_1),k_1k_2).
  \]
  Then we have the following
  \begin{enumerate}[listparindent=\parindent,parsep=5pt,label={(\arabic*)}]
    \item $G$ is a group, with identity element $(e,e)$ and inverses
    $(h,k)^{-1}:=(\alpha(k^{-1})(h^{-1}),k^{-1})$.
    \item The subsets $H'=H\times\{e\}$ and $K'=\{e\}\times K$ are subgroups,
    and there are isomorphisms $H\xrightarrow{\sim}H'$ and
    $K\xrightarrow{\sim}K'$ defined by $h\mapsto(h,e)$ and $k\mapsto(e,k)$
    respectively.
  \end{enumerate}
  We now identify $H$ with $H'$ and $K$ with $K'$ via these isomorphisms in the following.
  \begin{enumerate}[listparindent=\parindent,parsep=5pt,label={\arabic*.}]
    \skipitems{2}
    \item $H\unlhd G$.
    \item $H\cap K=\{e\}$ and $G=HK$.
    \item We have $khk^{-1}=\alpha(k)(h)$ for all $h\in H$ and $k\in K$.
  \end{enumerate}
  We denote this group $G$ by $H\rtimes G$, or by $H\rtimes_\alpha K$ if we
  want to make the action of $K$ on $H$ explicit.
\end{theorem}

\begin{example}
  Let $H=F(a)$ and $K=\langle b\mid b^2\rangle$. Let $\phi:K\to\Aut(H)$ be the
  homomorphism defined by $\phi(b)(a)=a^{-1}$. We obtain a semi-direct product
  $G=H\rtimes K$. If we identify $H$ and $K$ with the obvious subgroups of $G$,
  this means that
  \[
    G=\{a^n\mid n\in\mathbb{Z}\}\coprod \{a^nb\mid n\in\mathbb{Z}\},
    \qquad
    bab^{-1}=a^{-1}.
  \]
  In fact, $G$ is the infinite dihedral group.
\end{example}

\begin{example}
  Let $G\subseteq\Sym(\mathbb{R}^n)$ be the set of all functions $\phi:\mathbb{R}^n\to\mathbb{R}^n$ of the form
  \[
    \phi(x)=Ax+b,
    \qquad\quad
    A\in\GL_n(\mathbb{R}),
    \qquad
    b\in\mathbb{R}^n.
  \]
  This can be shown to be a subgroup. It is a semi-direct product of its subgroups
  \[
    H=\{\phi\mid\phi(x)=x+b,\ b\in\mathbb{R}^n\},
    \qquad
    K=\{\phi\mid\phi(x)=Ax,\ A\in\GL_n(\mathbb{R})\}.
  \]
\end{example}

\begin{definition}
  A \emph{composition series} for a group $G$ is a finite chain of subgroups
  \[
    \{e\}=M_0\leq M_1\leq\cdots\leq M_{r-1}\leq M_r=G,
    \qquad
    r\geq 0,
  \]
  such that
  \begin{enumerate}[listparindent=\parindent,parsep=5pt,label={(\arabic*)}]
    \item $M_{k-1}$ is a normal subgroup of $M_k$, for each $k=1,\ldots,r$, and
    \item the quotient $M_k/M_{k-1}$ is a simple group.
  \end{enumerate}
  The groups $M_1/M_0,M_2/M_1,\ldots,M_r/M_{r-1}$ are called the
  \emph{composition factors} of the composition series.
\end{definition}

\begin{proposition}
  Every finite group has a composition series.
\end{proposition}

\begin{theorem}[Jordan-H\"older]
  Suppose $G$ is a group with a composition series. Then the composition factors of a composition series are unique up to change of permutation. That is, if
  \[
    \{e\}=M_0\leq\cdots\leq M_r=G,
    \qquad
    \{e\}=N_0\leq\cdots\leq N_s=G
  \]
  are two composition series, then $r=s$ and there exists $\sigma\in S_r$ such
  that $M_k/M_{k-1}\cong N_{\sigma(k)}/M_{\sigma(k)-1}$ for all $k=1,\ldots,n$.
\end{theorem}

\begin{definition}
  A group $G$ is \emph{solvable} if it admits a finite chain of subgroups
  \[
    1=G_0\unlhd G_1\unlhd\cdots\unlhd G_s=G,
  \]
  with each $G_k\unlhd G_{k+1}$, such that each quotient $G_k/G_{k-1}$ is
  abelian. In particular, a finite group $G$ is solvable if its composition
  factors are abelian, i.e., all cyclic of prime order.
\end{definition}

\begin{example}
  \begin{itemize}
    \item If $N\unlhd G$ and both $N$ and $G/N$ are solvable, then $G$ is
    solvable.
    \item Abelian groups are solvable
    \item Dihedral groups are solvable, since $C_n\unlhd D_{2n}$ with
    $D_{2n}/C_n\cong C_2$. 
    \item The quarternion group $Q_8$ is solvable: it has a composition series
    \[
      \{\pm 1\}<\langle i\rangle <Q_8.
    \]
    \item $S_4$ is solvable, since $N=\langle(1\ 2)(3\ 4),\ (1\ 3)(2\
    4)\rangle\unlhd S_4$ and $S_4/N\cong S_3\cong D_6$.
  \end{itemize}
\end{example}

\begin{definition}
  Given elements $x,y\in G$, we write
  \[
    [x,y]:=xyx^{-1}y^{-1}\in G
  \]
  for the \emph{commutator} of $x$ and $y$. For subsets $S,T\subseteq G$, we
  write
  \[
    [S,T]:=\langle[x,y],\, x\in S,\,y\in T\rangle
  \]
  for the subgroup generated by such commutators. In particular, the
  \emph{commutator subgroup} of $G$ is the subgroup $[G,G]$ generated by all
  commutators.
\end{definition}

\begin{remark}
  $[G,G]$ is a normal subgroup of $G$. The quotient group $G/[G,G]$ is abelian,
  and is called the \emph{abelianization} of $G$.
\end{remark}

\begin{proposition}
  If $H\unlhd G$, then $G/H$ is abelian iff $[G,G]\leq H$.
\end{proposition}

\begin{definition}
  The \emph{derived series} of a gorup $G$ is the sequence of subgroups $G^{(k)}$ defined by
  \begin{itemize}
    \item $G^{(0)}=G$,
    \item $G^{(1)}=[G,G]$,
    \item $G^{(k)}=[G^{(k-1)},G^{(k-1)}]$, $k\geq 2$.
  \end{itemize}
  We obtain a descending chain of subgroups, each of which is normal in the previous:
  \[
    G=G^{(0)}\geq G^{(1)}\geq G^{(2)}\geq\cdots.
  \]
\end{definition}

\begin{proposition}
  $G$ is solvable iff $G^{(s)}=\{e\}$ for some $s$.
\end{proposition}

\begin{corollary}
  If $G$ is solvable, then so is any subgroup or quotient group of $G$.
\end{corollary}

\begin{definition}
  Given a group $G$, its \emph{upper central series} is defined by
  \begin{itemize}
    \item $Z_0(G)=\{e\}$,
    \item $Z_1(G)=Z(G)$,
    \item $Z_{k+1}(G)$ is the preimage under the quotient map $\pi:G\to
    G/Z_k(G)$ of $Z(G/Z_k(G))$, for all $k\geq 1$.
  \end{itemize}
  We ovtain a possibly infinite sequence of subgroups
  \[
    \{e\}=Z_0(G)\leq Z_1(G)\leq Z_2(G)\leq\cdots\leq G,
  \]
  each of  which is normal in $G$.
\end{definition}

\begin{definition}
  A group $G$ is \emph{nilpotent} if there exists a $c$ such that
  $Z_c(G)=G$. The smallest such $c$ is called the \emph{nilpotence class} of
  $G$.
\end{definition}

\begin{proposition}
  If $G$ is nilpotent, so is any quotient group $G/N$, and the nilpotence class
  of $G$ is $\geq$ the nilpotence class of $G/N$.
\end{proposition}

\begin{proposition}\label{product_of_nilpotent_is_nilpotent}
  $Z_k(G_1\times\cdots\times G_s)=Z_k(G_1)\times\cdots\times Z_k(G_s)$. In
  particular, if $G_1,\ldots,G_s$ are nilpotent, then so is
  $G=G_1\times\cdots\times G_s$.
\end{proposition}

\begin{proposition}\label{p-group_is_nilpotent}
  Let $p$ be a prime and $G$ a $p$-group of order $p^a$, $a\geq 1$. Then $G$ is
  nilpotent, and if $a\geq 2$, it has nilpotence class $\leq a-1$.
\end{proposition}

\begin{theorem}\label{finite_nilpotent_groups_theorem}
  Let $G$ be a finite group with $p_1,\ldots,p_s$ the distinct primes dividing
  its order. Then TFAE.
  \begin{enumerate}[listparindent=\parindent,parsep=5pt,label={(\arabic*)}]
    \item $G$ is nilpotent.
    \item If $H<G$, then $H<N_G(H)$ (i.e., every proper subgroup of $G$ is
    proper in its normalizer, or equivalently, $G$ is the only subgroup which
    is its own normalizer).
    \item $|\Syl_{p_i}(G)|=1$ for all $i=1,\ldots,s$ (or equivalently, $G$ has
    a normal $p_i$-Sylow subgroup for all $i=1,\ldots,s$).
    \item $G\cong P_1\times\cdots P_s$, where $P_i\in\Syl_{p_i}(G)$.
  \end{enumerate}
\end{theorem}

\begin{corollary}
  Any finite abelian group is a product of its Sylow subgroups.
\end{corollary}

\section{Rings \& Modules}

\begin{definition}
  A \emph{ring} is a set $R$ with binary operations $+$ and $\cdot$ satisfying
  \begin{itemize}
    \item $(R,+)$ is an abelian group with unit $0$.
    \item $(R,\cdot)$ is a semigroup.
    \item The product $\cdot$ distributes over $+$: $x\cdot(y+z)=x\cdot
    y+x\cdot z$ and $(y+z)\cdot x=y\cdot x+z\cdot x$.
  \end{itemize}
  If $\cdot$ has a unit element, then $R$ is called a \emph{ring with
  identity}, and the multiplicative unit will be denoted by $1$.
\end{definition}

We will only care about rings with identity, so we will simply write ``ring''
to mean ``ring with identity''. The \emph{trivial ring} is the unique ring with
$1=0$.

\begin{definition}
  Let $R$ be a ring.
  \begin{itemize}
    \item $a\in R$ is a \emph{unit} if there exists $b\in R$ such that $ab=1=ba$. If such a $b$ exists it is obviously unique.

    We write $R^\times$ for the set of units in $R$, which is a group under $\cdot$.
    \item $a\in R$ is an \emph{zero divisor} if $a\neq 0$ and there exists
    $b\in R\setminus\{0\}$ such that either $ab=0$ or $ba=0$.
    \item $a\in R$ is a \emph{non-zero divisor}, or \emph{cancellable}, if
    $a\neq 0$ and it is not a zero-divisor.
  \end{itemize}
\end{definition}

\begin{definition}
  \begin{itemize}
    \item A \emph{division ring} ( or \emph{skew-field}) is a ring with $1\neq
    0$ such that every nonzero element is a unit.
    \item A field is a commutative division ring.
    \item An \emph{integral domain} (or just \emph{domain}) is a commutative
    ring with $1\neq 0$ and no zero divisors.
  \end{itemize}
\end{definition}

\begin{proposition}
  Every finite domain is a field.
\end{proposition}
\begin{proof}
  Since every $a\in R\setminus\{0\}$ is cancellable, the map $x\mapsto ax$ from
  $R\to R$ is injective. Since $|R|<\infty$ it is bijective by the pigeonhole
  principle, so there exists some $b\in R$ such that $ab=1$.
\end{proof}

\begin{definition}
  A \emph{subring} of a ring $R$ is a subset $S\subseteq R$ which is a subgroup
  w.r.t.\ $+$ and is closed under $\cdot$. It's called a \emph{subring with
  identity} if in addition $1\in S$. (Warning: a subring $S\subseteq R$ can
  have an identity element which is not equal to $1$).
\end{definition}

\begin{example}
  The ring $\mathbb{H}$ of \emph{quarternions} is the set $\mathbb{R}^4$ of $4$-tuples of real numbers, where we write ``$a+bi+cj+dk$'' instead of ``$(a,b,c,d)$''. Addition is componentwise, and multiplication is defined using the distributive law and the identities
  \[
    i^2=j^2=k^2=-1
    \qquad
    ij=k=-ji
    \qquad
    jk=i=-kj
    \qquad
    ki=j=-ik.
  \]
  The quarternions form a division ring.
\end{example}

\begin{definition}
  A ring homomorphism is a function preserving addition and multiplication. If
  the rings have identity, then a homomorphism might not preserve the identity,
  although usually we will want them to.
\end{definition}

\begin{definition}
  Let $R$ be a ring and $I\subseteq R$ a subset. We say that $I$ is
  \begin{itemize}
    \item a \emph{left ideal} if $I$ is a subgroup of $(R,+)$ and if
    $rI\subseteq I$ for all $r\in R$.
    \item a \emph{right ideal} if $I$ is a subgroup of $(R,+)$ and if
    $Ir\subseteq I$ for all $r\in R$.
    \item a \emph{two-sided ideal} if $I$ is both a left and a right ideal.
  \end{itemize}
  We will sometimes call two-sided ideals simply \emph{ideals}.

  Note if $R$ is commutative then all three of these notions are the same.

  If $R$ has an identity, then the \emph{unit ideal} of $R$ is the unique ideal
  $I$ containing the identity, in which case $I=R$.
\end{definition}

\begin{theorem}[Second isomorphism theorem for rings]
  Let $A\subseteq R$ be a subring, and $I\subseteq R$ be an ideal. Then
  \begin{enumerate}[listparindent=\parindent,parsep=5pt,label={(\arabic*)}]
    \item $A+I$ is a subring of $R$.
    \item $I$ is an ideal of $A+I$.
    \item $A\cap I$ is an ideal of $A$.
    \item $A/(A\cap I)\cong (A+I)/I$ via $x+(A\cap I)\mapsto x+I$.
  \end{enumerate}
\end{theorem}

\begin{theorem}[Third isomorphism theorem for rings]
  Let $I,J\leq R$ be ideals with $I\subseteq J$. Then
  \begin{enumerate}[listparindent=\parindent,parsep=5pt,label={(\arabic*)}]
    \item $J/I$ is an ideal in $R/I$, and
    \item $R/J\cong (R/I)/(R/J)$ via $x+J\mapsto(x+I)+(J/I)$.
  \end{enumerate}
\end{theorem}

\begin{definition}
  Let $R$ be a commutative ring with identity, and suppose $A,B\leq R$ are
  ideals. Then we say $A$ and $B$ are \emph{comaximal} if $A+B=R$,
  equivalently, if $1=a+b$ for some $a\in A$ and $b\in B$.
\end{definition}

\begin{theorem}[The Chinese Remainder Theorem]
  If $A_1,\ldots,A_n$ are pairwise comaximal ideals in a commutative ring $R$
  with identity, then $A_1\cdots A_n=A_1\cap\cdots\cap A_n$, and
  \[
    R/A_1\cdots A_n\to(R/A_1)\times\cdots\times(R/A_n)
  \]
  sending
  \[
    r+A_1\cdots A_n\mapsto(r+A_1,r+A_2,\ldots,r+A_n)
  \]
  is a well-defined isomorphism.
\end{theorem}

\begin{example}
  Let $a_1,\ldots,a_n$ be pairwise coprime integers, and $a=a_1\cdots
  a_n$. Then
  $\mathbb{Z}/(a)\cong\mathbb{Z}/(a_1)\times\cdots\times\mathbb{Z}/(a_n)$.
\end{example}

\begin{definition}
  A \emph{Euclidean domain} is an integral domain $R$ such that there exists a function $N:R\setminus\{0\}\to\mathbb{Z}_{\geq 0}$ such that
  \begin{itemize}
    \item for any $a,b\in R$ with $b\neq 0$, there exists $q,r\in R$ such that
    \[
      a=qn+r\quad\text{with either }r=0\text{ or }N(r)<N(b).
    \]
  \end{itemize}
\end{definition}

\begin{example}
  The Gaussian integrs $\mathbb{Z}[i]\subseteq\mathbb{C}$ form a Euclidean
  domain.

  Define
  \[
    N(a+bi):=|a+bi|^2=(a+bi)(a-bi)=a^2+b^2.
  \]
  Note that $N(\alpha\beta)=N(\alpha)N(\beta)$. Now let $\alpha=a+bi$ and $\beta=c+di$ in $\mathbb{Z}[i]$, then in $\mathbb{C}$ we have
  \[
    \frac{\alpha}{\beta}
    =r+si
    =\frac{ac-bd}{c^2+d^2}+\frac{ad+bc}{c^2+d^2}i,
    \qquad\text{so}\qquad
    r=\frac{ac-bd}{c^2+d^2}\text{ and }s=\frac{ad+bc}{c^2+d^2}\text{ in }\mathbb{Q}.
  \]
  The number $\alpha/\beta$ is distance at most $\sqrt{2}/2=1/\sqrt{2}$ from
  some element of $\mathbb{Z}[i]$, which are exactly the points in the integer
  lattice inside $\mathbb{C}$, so we may choose $p,q\in\mathbb{Z}$ such that
  $|r-p|,|s-q|\leq 1/2$. Then
  \[
    \left|\frac{\alpha}{\beta}-(p+qi)\right|^2
    \leq|r-p|^2+|s-q|^2
    \leq\frac{1}{2}
  \]
  and thus
  \[
    |\alpha-(p+qi)\beta|^2\leq\frac{|\beta|^2}{2}.
  \]
  Therefore setting $\gamma=\alpha-(p+qi)\beta\in\mathbb{Z}[i]$, we have
  \[
    \alpha=(p+qi)\beta+\gamma,
    \qquad
    |\gamma|^2<|\beta|^2.
  \]
  Thus $\mathbb{Z}[i]$ is a Euclidean domain.
\end{example}

\begin{definition}
  In a domain $R$, a \emph{greatest common divisor (gcd)} of $a,b\in R$ with
  $b\neq 0$ is any element $d\in R$ such that (i) $a,b\in(d)$, and
  $a,b\in(e)\implies d\in(e)$.
\end{definition}

\begin{proposition}\label{euclidean_algorithm}[GCD algorithm in a Euclidean domain]
  Given a Euclidean domain $R$ and elements $a,b\in R$, there is an algorithm
  for computing a gcd of $a$ and $b$, called the \emph{Euclidean algorithm}. The algorithm works by constructing a sequence of elements of $R$ that begins with the two given elements $r_{-2}=a$ and $r_{-1}=b$ and will eventually terminate with $0$:
  \[
    \{r_{-2}=a,r_{-1}=b,r_0,r_1,\ldots,r_{n-1},r_n=0\},
  \]
  where $N(r_{k+1})<N(r_k)$. The element $r_{n-1}$ will then be the
  GCD. Explicitly, assuming $r_{k-2}$ and $r_{k-1}$ have been found, one should
  choose $r_k$ so that
  \[
    r_{k-2}=q_k\cdot r_{k-1}+r_k,\qquad\text{with }N(r_{k-1})>N(r_k)\geq 0.
  \]
\end{proposition}
\begin{proof}
  Note that since
\end{proof}

\begin{definition}
  Let $R$ be a domain. We can clasify elements of $R$ into exactly one of the following types.
  \begin{itemize}
    \item \emph{Zero}. Just $0$.
    \item \emph{Units}. Elements which have a multiplicative inverse.
    \item \emph{Reducible elements}. $r\in R$ which is not $0$ or a unit, such
    that $r=ab$ for some $a,b$ which are not $0$ or units.
    \item \emph{irreducible elements.} $r\in R$ which are not $0$ or a unit or reducible.
  \end{itemize}
\end{definition}

\begin{definition}
  We say $a,b\in R$ are \emph{associate} (or \emph{same up to units}) if there
  exists a unit $u\in R^\times$ such that $b=ua$. Being associate is an
  equivalence relation on $R$.

  We say that $a\mid b$ iff $(a)\subseteq(b)$. Equivalently if there is $c\in
  R$ such that $b=ac$.
\end{definition}

\begin{proposition}
  Let $a,b\in R$ a domain. TFAE.
  \begin{enumerate}[listparindent=\parindent,parsep=5pt,label={(\arabic*)}]
    \item $a$ and $b$ are assocaite.
    \item $a\mid b$ and $b\mid a$.
    \item $(a)=(b)$.
  \end{enumerate}
\end{proposition}

\begin{lemma}
  Let $p\in R$ which is not zero and not a unit. Then $p\in R$ is irreducible
  iff for all $a\in R$, $(p)\subsetneq(a)$ implies $(a)=R$. That is, $p$ is
  irreducible iff $(p)$ is maximal amongst proper \emph{principal} ideals.

  In particular if $R$ is a PID, then $p\in R$ is irreducible iff $p\neq 0$ and
  $(p)$ is maximal.
\end{lemma}

\begin{corollary}
  If $p,q$ are irreducible elements in a domain, then $p\mid q$ iff $p$ and $q$
  are the same up to units.
\end{corollary}

\begin{definition}
  In a domain $R$, an element $p\in R$ is \emph{prime} iff $p\neq 0$ and $(p)$
  is not a prime ideal. THat is, iff $p$ is nonzero and not a unit, and if
  $p\mid ab$ implies either $p\mid a$ or $p\mid b$.
\end{definition}

\begin{proposition}
  In a domain, prime elements are irreducible.
\end{proposition}

\begin{proposition}
  In a PID, an element is prime iff it is irreducible.
\end{proposition}

\begin{definition}
  A \emph{unique factorization domain} (UFD) is a domain such that every
  non-zero non-unit $r\in R$ satisfies
  \begin{enumerate}[listparindent=\parindent,parsep=5pt,label={(\arabic*)}]
    \item $r=p_1\cdots p_n$ for some irreducibles $p_1,\ldots,p_n\in R$, $n\geq 1$, and
    \item this decomposition is unique up to associates, i.e., if $r=p_1\cdots
    p_n=q_1\cdots q_m$ for irreducibles $p_i$, $q_j$, then $m=n$ and there is a
    a permutation $\sigma\in S_n$ such that $q_k=u_kp_{\sigma(k)}$ for some
    unit $u_k$, for $k=1,\ldots,n$.
  \end{enumerate}
\end{definition}

\begin{proposition}
  In a UFD, prime and irreducible are equivalent.
\end{proposition}

\begin{theorem}
  Every PID is a UFD.
\end{theorem}

\begin{definition}
  Let $R$ be an integral domain. Say that $R$ has the \emph{ascending chain condition (acc) for principal ideals} if for any collection $(I_k)_{k\in\mathbb{Z}_{\geq 0}}$ of principal ideals in $R$ such that
  \[
    I_1\subseteq I_2\subseteq I_3\subseteq\cdots,
  \]
  there exists some $n$ such that $I_k=I_n$ for all $k\geq n$.
\end{definition}

\begin{lemma}
  Every PID has the acc for principal ideals.
\end{lemma}

\begin{proposition}
  Let $R$ be a domain. If $R$ has the acc for principal ideals, then every
  nonzero nonunit in $R$ is a finite product of irreducible elements.
\end{proposition}

\begin{proposition}
  Let $R$ be an integral domain. If all irreducible elements in $R$ are prime
  elements, then factorization in irreducibles (when it exists) is unique up to
  units and reordering.
\end{proposition}

\begin{lemma}
  Let $\alpha\in\mathbb{Z}[i]$. If $N(\alpha)\in\mathbb{Z}$ is a prime number,
  then $\alpha$ is irreducible in $\mathbb{Z}[i]$.
\end{lemma}

\begin{proposition}
  If $R$ is commutative with unit, $S\subseteq R$ is a subring (with $1$), and
  $P\leq R$ is a prime ideal, then $S\cap P$ is a prime ideal of $S$.
\end{proposition}

\begin{proposition}
  Let $p\in\mathbb{Z}$ be a prime number, and let $\alpha\in\mathbb{Z}[i]$ be
  an irreducible element. TFAE
  \begin{enumerate}[listparindent=\parindent,parsep=5pt,label={(\arabic*)}]
    \item $\alpha$ is an irreducible divisor of $p$ in $\mathbb{Z}[i]$.
    \item $p\mathbb{Z}=(\alpha)\cap\mathbb{Z}$.
  \end{enumerate}
\end{proposition}

Suppose $\alpha\in\mathbb{Z}[i]$ is irreducible with
$(\alpha)\cap\mathbb{Z}=p\mathbb{Z}$ with $p$ a prime integer. We have
$p=\alpha\beta$ for some $\beta\in\mathbb{Z}[i]$. Taking norms gives
\[
  p^2=N(\alpha)N(\beta).
\]
Since $\alpha$ is not a unit, there are two cases
\begin{itemize}
  \item $N(\alpha)=p^2$, $N(\beta)=1$, so that $\beta$ is a unit and thus $p$
  and $\alpha$ are associate, so $\alpha\in\{\pm p,\pm pi\}$.
  \item $N(\alpha)=p$, $N(\beta)=p$, so that both $\alpha$ and $\beta$ are
  irreducible, and $p=\alpha\beta$ is an irreducible factorization of $p$. Thus
  these are the only two irreducible divisors up to associates, by uniqueness
  of irreducible factorizations.
\end{itemize}
As a conclusion, if $p$ is a prime number, then an element $\alpha=a+bi\in\mathbb{Z}[i]$ is an irreducible divisor of $p$ iff one of the following mutually exclusive cases occurs:
\begin{enumerate}[listparindent=\parindent,parsep=5pt,label={(\arabic*)}]
  \item $\alpha=\pm p$ or $\alpha=\pm pi$, or
  \item $a^2+b^2=p$.
\end{enumerate}
Thus $p$ is prime in $\mathbb{Z}[i]$ iff the equation $a^2+b^2=p$ has an
integer solution $(a,b)\in\mathbb{Z}^2$.

\begin{lemma}[Lagrange]
  Let $p$ be a prime number of the form $p=4m+1$, with $m\in\mathbb{Z}$. Then
  there exists some $n\in\mathbb{Z}$ such that $p\mid (n^2+1)$.
\end{lemma}

\begin{theorem}[Fermat]
  A rational prime $p$ is a sum of two squares iff $p=2$ or $p\equiv 1\bmod 4$.
\end{theorem}

\begin{corollary}
  A prime number $p$ is prime/irreducible in $\mathbb{Z}[i]$ iff $p\neq 2$ and
  $p\not\equiv 1\bmod 4$.
\end{corollary}

\begin{proposition}
  A positive integer $n$ has the form $n=a^2+b^2$ for some $a,b\in\mathbb{Z}$
  iff its prime factorization (in $\mathbb{Z}$) $n=p_1^{k_1}\cdots p_r^{k_r}$
  (primes $p_i$ pairwise disticnct) is such that: if $p_i\equiv -1\mod p$, then
  $k_i$ is even.
\end{proposition}

\begin{corollary}
  A positive integer $n$ has the form $n=a^2+b^2$ for some $a,b\in\mathbb{Z}$
  iff its prime factorization (in $\mathbb{Z}$) $n=p_1^{k_1}\cdots p_r^{k_r}$
  (primes $p_i$ pairwise distinct) is such that: if $p_i\equiv -1\bmod p$, then
  $k_i$ is even.
\end{corollary}

\begin{definition}
  Let $\{a_1,\ldots,a_n\}$ be a finite subset of a domain $R$. Then $d\in R$ is a GCD of the set iff
  \begin{enumerate}[listparindent=\parindent,parsep=5pt,label={(\arabic*)}]
    \item $(a_1,\ldots,a_n)\subseteq(d)$, and
    \item if $(a_1,\ldots,a_n)\subseteq(e)$ for some $e\in R$, then
    $(d)\subseteq (e)$.
  \end{enumerate}
\end{definition}

\begin{proposition}
  If $R$ is a UFD, then every finite subset of $R$ has a GCD.
\end{proposition}

\begin{proposition}
  Let $R$ be a UFD, $\{a_1,\ldots,a_n\}$ a finite set of elements in $R$, and
  $d,c\in R$ with $c\neq 0$. Then $d$ is a GCD of $\{a_1,\ldots,a_n\}$, if and
  only if $dc$ is a GCD of $\{a_1c,\ldots,a_nc\}$.
\end{proposition}

\begin{definition}
  We say a subset $\{a_1,\ldots,a_n\}$ of a domain $R$ is \emph{relatively
  prime} if $1$ is a GCD for the set. If $d$ is a GCD of a subset
  $\{a_1,\ldots,a_n\}$ of a UFD, then $\{a_1/d,\ldots,a_n/d\}$ is a relatively
  prime subset.
\end{definition}

\begin{proposition}
  If $R$ is a UFD, $F=\Frac R$ is the fraction field of $R$, and $c\in
  F^\times=F\setminus\{0\}$, then we can write $c=a/b$ for $a,b\in R$ with
  $\{a,b\}$ relatively prime. Furthermore, any two such exrepssions
  $c=a/b=a'/b'$ differ by a unit: i.e., there exists $u'\in R^\times$  such
  that $a'=ua$ and $b'=ub$.
\end{proposition}

\begin{definition}
  For a domain $S$, we write $\Irred(S)\subseteq S$ for the subset of
  irreducible elements.
\end{definition}

\begin{proposition}
  If $f,g,h\in R[x]$ are such that $f=gh$, then $f\in R\setminus\{0\}$ iff
  $g,h\in R\setminus\{0\}$.
\end{proposition}

\begin{definition}
  Let $f=\sum_{k=0}^{n}c_kx^k\in R[x]$. We say that $f$ is \emph{primitive} if
  the set $\{c_0,\ldots,c_n\}$ of its coefficients is relatively prime. For
  example, every monic polynomial is primitive.
\end{definition}

\begin{proposition}
  Let $R$ be a UFD and let $f\in R[x]$ with $f\neq 0$. Then there exist $a\in
  R$ and $g\in\Prim(R[x])$ such that
  \[
    f=ag.
  \]
  Furthermore, this factorization is unique up-to-units in $R$. That is, if
  \[
    f=ag=a'g',\qquad a,a'\in R,\qquad g,g'\in\Prim(R[x]),
  \]
  then there exists $u\in R^\times$ such that $a'=ua$, $g'=u^{-1}g$.
\end{proposition}

\begin{proposition}
  Let $R$ be a UFD, let $F:=\Frac R$, and let $f\in F[x]$ with $f\neq 0$. Then
  there exists $c\in F^\times$ and $g\in\Prim(R[x])$ such that $f=cg$, and
  furthermore this factorization is unique up-to-units in $R$. That is, if
  \[
    f=cg=c'g'\in F[x],\qquad c,c'\in F,\qquad g,g'\in\Prim(R[x]),
  \]
  then there exists $u\in R^\times$ such that $c'=uc$ and $g'=u^{-1}g$.
\end{proposition}

\begin{proposition}[Gauss' Lemma]
  Let $f,g$ be two primitive polynomials over a UFD. Then $fg$ is
  primitive. I.e., if $R$ is a UFD, then $\Prim(R[x])$ is multiplicatively
  closed.
\end{proposition}

\begin{proposition}
  Let $R$ be a UFD and suppose $f=gh\in R[x]$. Then $f\in\Prim(R[x])$ if
  $g,h\in\Prim(R[x])$.
\end{proposition}

\begin{proposition}
  If $R$ is a UFD, $F:=\Frac R$, and
  \[
    f=gh\in\Prim(R[x]),\qquad g,h\in F[x],
  \]
  there exist
  \[
    c\in F^\times,\qquad, g',h'\in\Prim(R[x])\qquad\text{ such that }\quad g=c^{-1}g',\qquad h=ch',\qquad f=g'h'.
  \]
\end{proposition}

\begin{corollary}
  If $R$ is a UFD and $F:=\Frac(R[x])$, then $f\in\Prim(R[x])$ is irreducible
  iff it is irreducible in $F[x]$.
\end{corollary}

\begin{corollary}
  If $R$ is a UFD, a nonunit $f\in\Prim(R[x])$ admits a factorization
  $f=p_1\cdots p_r$ into primitive irreducibles $p_1,\ldots,p_r$, and this
  factorization is unique up to reordering and units.
\end{corollary}

\begin{theorem}
  Let $R$ be a UFD. Then $R[x]$ is also a UFD. Furthermore, every irreducible $f\in R[x]$ is one of exactly two types.
  \begin{enumerate}[listparindent=\parindent,parsep=5pt,label={(\arabic*)}]
    \item $f\in R$ and $f$ is irreducible in $R$.
    \item $f\in\Prim(R[x])$ and $f$ is irreducible in $F[x]$, where $F:=\Frac
    R$.
  \end{enumerate}
\end{theorem}

\begin{proposition}
  Let $F$ be a field. If $f\in F[x]$ and $a\in F$ is such that $f(a)=0$, then
  $f=(x-a)g$ for some $g\in F[x]$.
\end{proposition}

\begin{corollary}
  Let $F$ be a field. If $f\in F[x]$ with $\deg f\in\{2,3\}$, then $f$ is
  irreducible iff it has a root in $f$.
\end{corollary}

\begin{definition}
  Let $F$ be a field, and $f\in F[x]$. Say that $c\in F$ is a \emph{root of
  multiplicity $m$} if $m\in\mathbb{Z}_{\geq 0}$ is the largest integer such
  that $(x-c)^m\mid f$ in $F[x]$.
\end{definition}

\begin{proposition}
  If $f\in F[x]$ with $\deg f=n$, then $f$ has at most $n$ roots in $F$, even
  if ``counted up to multiplicity''.
\end{proposition}

\begin{proposition}
  Suppose $F$ is the fraction field of a UFD $R$, and consider a polynomial in $R[x]$ of the form
  \[
    f=a_nx^n+a_{n-1}x^{n-1}+\cdots+a_1x+a_0,\qquad a_k\in R,\qquad \deg f=n.
  \]
  If $c\in F$ is a root of $f$, and if $c=r/s$ with $r,s\in R$ is a fraction in lowest terms, then
  \[
    r\mid a_0
    \qquad\text{and}\qquad
    s\mid a_n.
  \]
  In particular, if $f$ is monic, then any roots of $f$ in $F$ are elements
  $c\in R$ which divide $a_0$.

  {\color{red}The numerator divides the constant term, the denominator divides
  the leading term.}
\end{proposition}

\begin{example}
  The polynomial $f=x^3-3x-1\in\mathbb{Z}[x]$ is irreducible in
  $\mathbb{Q}[x]$, since by the above the only possible roots are $\pm 1$, but
  $f(\pm 1)\neq 0$. Because $f$ is monic and thus primitive, it is also
  irreducible in $\mathbb{Z}[x]$.
\end{example}

\begin{proposition}
  Let $R$ be an integral domain, and $I<R$ a proper ideal. Let $f\in R[x]$ be a
  monic polynomial of positive degree. If its image $\overline{f}\in(R/I)[x]$
  is irreducible in $(R/I)[x]$, then $f$ is irreducible in $R[x]$.
\end{proposition}

\begin{example}
  Let $R=\mathbb{Q}[x]$ and $I=(x)$. Consider $f=x^3+y^2+3x^2y+17xy+1\in
  R[y]=\mathbb{Q}[x,y]$. As a polynomial with coefficients in $\mathbb{Q}[x]$,
  this is monic. Note that $(R/I)[y]\cong\mathbb{Q}[y]$, and reducing mod $I$
  amounts to setting $x=0$, and gives $\overline{f}=y^2+1$, which is
  irreducible in $\mathbb{Q}[y]$, so $f$ is irreducible.
\end{example}

\begin{proposition}[Eisenstein's criterion]
  Let $R$ be a domain with prime ideal $P\subseteq R$, and let
  $f=x^n+a_{n-1}x^{n-1}+\cdots+a_1x+a_0\in R[x]$ be a monic polynomial over
  $R$. If $a_0,\ldots,a_{n-1}\in P$ and $a_0\notin P^2$, then $f$ is
  irreducible in $R[x]$.
\end{proposition}

\begin{example}[The cyclotomic polynomial $\Phi_p$]
  Let $R=\mathbb{Z}$ and $P=(p)$ for some prime $p$. Then if
  $f=a_nx^n+\cdots+a_0$ is a monic polynomial in $\mathbb{Z}[x]$ such that
  $p\mid a_k$ for $k=0,\ldots,n-1$, and $p\nmid a_0$, then $f$ is
  irreducible.

  For instance, consider $\Phi_p(x)=\sum_{k=0}^{p-1}x^k\in\mathbb{Z}[x]$. This is a factor of
  \[
    x^p-1=(x-1)\Phi_p(x),
  \]
  so roots of $\Phi_p$ in $\mathbb{C}$ are $\lambda\in\mathbb{C}$ such that
  $\lambda^p=1$ but $\lambda\neq 1$.

  Let
  \[
    f(x)
    =\Phi_p(x+1)
    =\sum_{k=0}^{p-1}{p\choose k+1}x^k
    =x^p+px^{p-1}+\cdots+\frac{p(p-1)}{2}x+p
  \]
  (where the second equality follows by the hockey-stick identity). This has
  the Eisenstein property for $p$, so $f$ is irreducible in $\mathbb{Z}[x]$,
  and thus in $\mathbb{Q}[x]$. (Note: this argument uses the fact that the
  function $\mathbb{Q}[x]\to\mathbb{Q}[x]$ defined by $f(x)\mapsto f(x+1)$ is
  an isomorphism of rings, and thus takes irreducible elements to irreducible
  elements.)
\end{example}

\begin{proposition}
  Let $F$ be a field and $G\leq F^\times$ a finite subgroup of its abelian
  group of units. Then $G$ is a cyclic group.
\end{proposition}

\begin{definition}
  Let $R$ be commutative with unit. We say that $R$ is \emph{Noetherian} if it
  has the ascending chain condition for ideals. That is, if
  $(I_k)_{k\in\mathbb{N}}$ is an increasing sequence of ideals (so
  $I_k\subseteq I_{k+1}$ for all $k\in\mathbb{N}$), then there exists some
  $n>0$ such that $I_k=I_n$ for all $k\geq n$.
\end{definition}

\begin{theorem}
  Let $R$ be commutative with unit. Then $R$ is noetherian iff every ideal in
  $R$ is f.g.
\end{theorem}

\begin{theorem}[Hilbert basis theorem]
  Let $R$ be a Noetherian ring, then $R[x_1,\ldots,x_n]$ is Noetherian.
\end{theorem}

\begin{proposition}
  Let $M$ be a cyclic $R$-module (meaning $M$ has a generating set of size
  $1$). Then there is an isomorphism of $R$-modules $M\cong R/I$ for some left
  ideal $I\leq R$.
\end{proposition}

\begin{proposition}
  Let $N_1,\ldots,N_k\subseteq M$ be submodules, and set $N:=N_1+\cdots+N_k$. Then TFAE.
  \begin{enumerate}[listparindent=\parindent,parsep=5pt,label={(\arabic*)}]
    \item The map $\phi:N_1\oplus\cdots\oplus N_k\to N$ defined by
    $\phi(x_1,\ldots,x_k):=x_1+\cdots+x_k$ is an isomomorphism of modules.
    \item $N_j\cap(N_1+\cdots+N_{j-1}+N_{j+1}+\cdots+N_k)=0$ for all $j=1,\ldots,k$.
    \item Every $x\in N$ can be written \emph{uniquely} in the form
    $x=x_1+\cdots+x_k$ with $x_j\in N_j$.
  \end{enumerate}
\end{proposition}

\begin{definition}
  Let $R$ be a ring with $1$ (but possibly non-commutative). Suppose $M$ is a
  right $R$-module and $N$ is a left $R$-module. Then an \emph{$R$-balanced
  bilinear function} $\beta:M\times N\to A$ is a bilinear function of abelian groups which also satisfies
  \[
    \beta(mr,n)=\beta(m,rn)\text{ for }m\in M,\ n\in N,\ r\in R.
  \]
  Note if $R=\mathbb{Z}$, then any bilinear map is already balanced.

  If $R$ is commutative, then left and right $R$-modules are the same. In this case, if $A$ is also an $R$-module, then a map $\beta:M\times N\to A$ is \emph{$R$-bilinear}, if
  \[
    \beta(mr,n)=\beta(m,rn)=r\beta(m,n)\text{ for }m\in M,\ n\in N,\ r\in R.
  \]
\end{definition}

\begin{definition}
  Let $R$ be a ring with $1$. Let $M$ and $N$ be right and left $R$-modules,
  respectively. Then there exists an abelian group $M\otimes_RN$ equipped with
  a group homomorphism $s:M\times N\to M\otimes_RN$. Moreover, $s$ is the
  universal $R$-bilinear map out of $M\times N$, i.e., it yields a bijection
  between $R$-balanced bilinear map $M\times N\to A$ and group homomorphisms
  $M\otimes_RN\to A$. The abelian group $M\otimes_RM$ is called the
  \emph{tensor product} of $M$ and $N$ over $R$.

  We write $m\otimes n$ for the image of $(m,n)$ under $s$. Elements of this
  form are called \emph{simple tensors}.
\end{definition}

\begin{proposition}
  Let $R$ be commutative with $1$. If $M$ and $N$ are free $R$-modules on bases
  $\{u_1,\ldots,u_n\}$ and $\{v_1,\ldots,v_m\}$ respectively, then
  $M\otimes_RN$ is a free $R$-module on the basis $\{u_i\otimes
  v_j\}_{i=1,\ldots,n,\,j=1,\ldots,m}$.
\end{proposition}

\begin{proposition}
  Let $R$ be commutative with $1$. If $M$ and $N$ are $R$-modules, generated by subsets $S$ and $T$ respectively, and $M'\subseteq M$ and $N'\subseteq N$ are submodules generated by subsets $U\subseteq M'$ and $V\subseteq N'$, then
  \[
    M/M'\otimes_RN/N'\cong(M\otimes_RN)/R\{s\otimes v,\ u\otimes t\mid s\in S,\,t\in T,\,u\in U,\,v\in V\}.
  \]
\end{proposition}

\begin{definition}
  Let $R$ be a domain. An element $x$ in an $R$-module $M$ is \emph{torsion} if
  there exists a nonzero $r\in R$ such that $rx=0$. We say a module $M$ is
  \emph{torsion} if $M_\tors=M$ and is \emph{torsionfree} if $M_\tors=\{0\}$.
\end{definition}

\begin{lemma}
  Let $R$ be an integral domain. The collection $M_\tors\subseteq M$ of torsion
  elements is a submodule. The quotient module $M/M_\tors$ is torsionfree.
\end{lemma}

\begin{proposition}
  Let $R$ be a domain. Given an $R$-submodule $N\subseteq M$, the quotient
  module $M/N$ is torsion iff for all $x\in M$ there exists $c\in
  R\setminus\{0\}$ such that $cx\in N$.
\end{proposition}

\begin{definition}
  Let $R$ be a domain and $M$ an $R$-module. Say that an indexed collection
  $(x_i\in M)_{i\in I}$ is $R$-linearly dependent (or just $R$-dependent) if
  there exists an indexed collection $(r_i\in R)_{i\in I}$ with $0<|\{i\in
  I\mid r_i\neq 0\}|<\infty$ and $\sum_{i}r_ix_i=0$. Otherwise the collection
  is \emph{$R$-linearly independent}, or just $R$-independent.
\end{definition}

\begin{lemma}
  Let $R$ be a domain and $M$ an $R$-module. A subset $S\subseteq M$ is
  $R$-independent iff the submodule $N=RS$ generated by $S$ is free, with $S$ a
  free basis of $N$.
\end{lemma}

\begin{definition}
  Let $R$ be a domain. The collection of $R$-independent subsets $S\subseteq M$
  is ordered by $\subseteq$. Say that an $R$-independent subset $S\subseteq M$
  is \emph{maximally $R$-independent} if it is maximal with respect to this
  ordering, i.e., if whenever $S\subseteq T\subseteq M$ with $T$ an
  $R$-independent subset, then $S=T$.
\end{definition}

\begin{lemma}
  Let $R$ be a domain. An $R$-independent subset $S\subseteq M$ is maximal iff
  $M/N$ is a torsion module where $N=RS$.
\end{lemma}

\begin{proposition}
  Every module over an integral domain admits a maximal $R$-independent subset.
\end{proposition}

\begin{proposition}\label{Interchange lemma}
  Let $R$ be a domain and $M$ an $R$-module. Suppose we have sequences of elements $v_1,\ldots,v_m$, $w_1,\ldots,w_n$ in $M$ such that
  \begin{itemize}
    \item $v_1,\ldots,v_m$ is $R$-independent, and
    \item $M/R\{w_1,\ldots,w_n\}$ is a torsion module.
  \end{itemize}
  Then
  \begin{enumerate}[listparindent=\parindent,parsep=5pt,label={(\arabic*)}]
    \item $m\leq n$, and
    \item after reording $w_1,\ldots,w_n$, we have that
    $M/R\{v_1,\ldots,v_m,w_{m+1},\ldots,w_n$ is a torsion module.
  \end{enumerate}
\end{proposition}

\begin{proposition}
  Let $R$ be a domain and $M$ an $R$-module. Let $S\subseteq M$ be a finite
  subset of size $n$ such that $M/RS$ is torsion. Then there exists a maximal
  $R$-independent subset of size $m\leq n$, and every maximal $R$-independent
  subset of $M$ has size $m$.

  We call this $m$ the \emph{rank} of $M$.
\end{proposition}

\begin{proposition}
  Let $R$ be an integral domain, $M$ an $R$-module with $N\subseteq M$ a
  submodule. If $N$ has finite rank $n$, and $M/N$ has finite rank $m$, then
  $M$ has finite rank $m+n$.

  In particular, if $A$ and $B$ are modules of finite rank, then $\rank(A\oplus
  B)=\rank A+\rank B$.
\end{proposition}

\begin{definition}
  Let $R$ be a ring with $1$ (not necessarily commutative). Given a left $R$-module $M$, the \emph{annihilator} of $M$ is the subset
  \[
    \Ann(M)
    :=\{x\in R\mid xM=0\}
    =\{x\in R\mid xm=0\text{ for all }m\in M\}.
  \]
\end{definition}

\begin{proposition}
  $\Ann(M)$ is a right ideal in $R$.
\end{proposition}

\begin{proposition}
  If $M\cong N$ are isomorphic left $R$-modules, then $\Ann M=\Ann N$.
\end{proposition}

\begin{proposition}
  Let $R$ be a ring and $I,J\subseteq R$ $2$-sided ideals. Then $R/I\cong R/J$
  as left $R$-modules iff $I=J$.
\end{proposition}

\begin{proposition}
  Every f.g.\ module over a PID is isomorphic to a finite direct sum of cyclic
  modules.
\end{proposition}

\begin{theorem}[Modules over a PID: Invariant factor form]
  Let $R$ be a PID and $M$ a f.g.\ $R$-module.
  \begin{itemize}
    \item There exists $t\geq 0$ and a chain of proper ideals $R\supsetneq (a_1)\supseteq\cdots\supseteq(a_t)$ such that
    \[
      M\cong R/(a_1)\oplus\cdots\oplus R/(a_t).
    \]
    \item The number $t$ and the sequence $(a_1),\ldots,(a_t)$ of ideals are
    unique, in the sense that if also $M\cong R/(a_1')\oplus\cdots\oplus
    R/(a_{t'}')$ with $R\supsetneq(a_1')\supseteq\cdots\supseteq(a_{t'}')$,
    then $t=t'$ and $(a_k)=(a_k)'$ for all $k$.
  \end{itemize}
\end{theorem}

\begin{remark}
  Write $t=s+r$ with $0\leq s,r\leq t$ where $(a_1),\ldots,(a_s)\neq(0)$ and $(a_{s+1})=\cdots=(a_{s+r})=(0)$. Then this becomes
  \[
    M\cong R/(a_1)\oplus\cdots\oplus R/(a_s)\oplus R^r,
  \]
  where each $R/(a_1),\ldots,R/(a_s)$ is a torsion cyclic module, and $\rank
  M=r$. This is how the invariant factor composition is usually presented.

  The ideals $(a_1),\ldots,(a_s)$ are called the \emph{invariant factors}, and
  $r=\rank M$.
\end{remark}

\begin{theorem}[Modules over a PID: Elementary divisor form]
  Let $R$ be a PID, and $M$ a f.g.\ $R$-module.
  \begin{itemize}
    \item There exist $r,u\geq 0$, and a sequence of elements $p_1^{k_1},\ldots,p_u^{k_u}\in R$ (not necessarily distinct) with $p_i$ prime and $k_i\geq 1$, such that
    \[
      M\cong R^{r}\oplus R/(p_1^{k_1})\oplus\cdots\oplus R/(p_u^{k_u}).
    \]
    \item The numbers $r$ and $u$ are unique, and the sequence
    $p_1^{k_1},\ldots,p_u^{k_u}$ is unique up to reordering and units, in the
    sense that if also $M\cong R^{r'}\oplus R/(q_1^{\ell_1})\oplus\cdots\oplus
    R/(q_{u'}^{\ell_{u'}})$, then $r=r'$, $u=u'$, and the sequence
    $q_1^{\ell_1},\ldots,q_u^{\ell_u}$ is the same as
    $p_1^{k_1},\ldots,p_u^{k_u}$ up to reordering and units.
  \end{itemize}
\end{theorem}

\begin{remark}
  In the elementary divisor form, we also have $r=\rank M$. The list
  $p_1^{k_1},\ldots,p_u^{k_u}$ are called \emph{elementary divisors}.
\end{remark}

\begin{proposition}
  Let $R$ be a PID, $M$ a free $R$-module of rank $m$, and $N\subseteq M$ a submodule. Then
  \begin{enumerate}[listparindent=\parindent,parsep=5pt,label={(\arabic*)}]
    \item $N$ is a free $R$-module of some rank $n\leq m$, and
    \item There exists
    \begin{itemize}
      \item a free basis $x_1,\ldots,x_m$ of $M$, and
      \item elements $a_1,\ldots,a_n\in R$ with $(a_1)\supseteq\cdots\supseteq
      (a_n)\supsetneq(0)$, such that
      \item $y_1=a_1x_1,\ldots,y_n=a_nx_n$ is a free basis of $N$.
    \end{itemize}
  \end{enumerate}
\end{proposition}

\begin{lemma}
  Let $R$ be a commutative ring, and $M$ an $R$-module. Suppose $I\leq R$ is an
  ideal such that $I\subseteq\Ann M$. That is, $IM=0$, or more concretely,
  $am=0$ for all $a\in I$ and $m\in M$. Then $M$ admits the structure of an
  $R/I$-module, defined so that
  \[
    (r+I)m:=rm.
  \]
  Furthermore, if $M\cong N$ as $R$-modules, and if $IM=0$, then also $IN=0$
  and the isomorphism is also an isomorphism of $R/I$-modules.
\end{lemma}

\begin{proposition}
  Let $R$ be a commutative ring.
  \begin{enumerate}[listparindent=\parindent,parsep=5pt,label={(\arabic*)}]
    \item If $\phi:M\to N$ is an isomorphism of $R$-modules, then $\phi$
    restricts to an isomorphism $IM\to IN$ of submodules. It further induces an
    isomorphism $M/IM\to N/IN$ on quotient modules, which is an isomorphism of
    $R/I$-modules.
    \item If $M=M_1\oplus\cdots\oplus M_n$ is an internal diret sum
    decomposition of an $R$-module, then $IM=IM_1\oplus\cdots\oplus IM_n$, and
    thus $M/IM\cong M/IM_1\oplus\cdots\oplus M/IM_n$ as $R/I$-modules.
    \item If $M$ is a f.g.\ $R$-module, then $M/IM$ is f.g.\ as both an
    $R$-module and an $R/I$-module.
    \item If $M$ is a f.g.\ $R$-module, and $I\leq R$ is a f.g.\ ideal, then
    $IM$ is also a f.g.\ $R$-module.
  \end{enumerate}
\end{proposition}

\begin{definition}
  Let $R$ be a PID and $p\in R$ a prime, and consider a f.g.\ module $M$. Note that $p^{k+1}M=p(p^kM)\subseteq p^kM$. THus we obtain a chain of submodules
  \[
    M=p^0M\supseteq p^1M\supseteq p^2M\supseteq\cdots,
  \]
  each of which is also f.g. We therefore get quotients
  \[
    M/pM,\qquad pM/p^2M,\qquad p^3M/p^2M,\ldots,
  \]
  each of which is a f.g.\ $R/p$-module.

  Note that since $p$ is irreducible, $R/p$ is a field. For $k\geq 1$ define
  \[
    \alpha_{p^k}(M):=\dim_{R/p}p^{k-1}M/p^kM.
  \]
\end{definition}

\begin{proposition}
  \begin{enumerate}[listparindent=\parindent,parsep=5pt,label={(\arabic*)}]
    \item The function $\alpha_{p^k}$ is an isomorphism invariant of f.g.\
    $R$-modules.
    \item If $M\cong M_1\oplus\cdots\oplus M_n$, then
    $\alpha_{p^k}(M)=\alpha_{p^k}(M_1)+\cdots+\alpha_{p^k}(M_n)$.
    \item If $M\cong R/(a)$ for some $a\in R$, then
    \[
      \alpha_{p^k}(M)=\begin{cases}
        1 & \text{if }p^k\mid a, \\
        0 & \text{if }p^k\nmid a.
      \end{cases}
    \]
    In particular, when $a=0$, this says that $\alpha_{p^k}(R)=1$.
  \end{enumerate}
\end{proposition}

\begin{remark}
  As a consequence of the above proposition, if 
  \[
    M\cong R^r\oplus R/(a_1)\oplus\cdots\oplus R/(a_m),\qquad a_k\in R\setminus\{0\},
  \]
  we have
  \[
    \alpha_{p^k}(M)=r+\text{number of $j\in\{1,\ldots,m\}$ such that $p^k\mid a_j$}.
  \]
  Now define
  \[
    \beta_{p^k}(M)=\alpha_{p^k}(M)-\alpha^{p^{k+1}}(M).
  \]
  Then for the above $M$, we have
  \[
    \beta_{p^k}(M)=\text{number of $j\in\{1,\ldots,m\}$ such that $p^k\mid a_j$ and $p^{k+1}\nmid a_j$}.
  \]
  By construction, $\alpha_{p^k}$ and thus $\beta_{p^k}$ are isomorphism invariants, and we have shown that, for any elementary divisor decomposition
  \[
    M\cong R^r\oplus R/(p_1^{k_1})\oplus\cdots\oplus R/(p_u^{k_u}),
  \]
  we have that $\beta_{p^k}(M)=$the number of elementary divisors in
  $p_1^{k_1},\ldots,p^{k_u}_u$ which are the same as $p^k$ up-to-units.
\end{remark}

\begin{definition}
  Recall that given a \emph{linear operator}, i.e., a pair $(V,T:V\to V)$ where
  $V$ is an $F$-vector space and $T$ is an $F$-linear map, we can give $V$ the
  structure of an $R=F[x]$-module, so that
  \[
    fv:=f(T)v,\qquad f\in F[x],\qquad v\in V.
  \]
  We will write $V_T$ for this $F[x]$-module.

  Conversely, every $F[x]$-module $M$ is of the form $V_T$ for some $(V,T)$,
  where $V$ is the underlying $F$-vector space of the module $M$ (so $V=M$ as
  an abelian group), and $T$ is defined by $T(v):=xv$. So $F[x]$-modules are
  really the same as $F$-linear operators.

  There is a further dictionary:
  \begin{itemize}
    \item Submodules of $V_T$ correspond to \emph{$T$-invariant subspaces},
    i.e., vector spaces $W\subseteq V$ such that $T(W)\subseteq W$.
    \item Homomorphisms $\phi:V_T\to W_U$ of $F[x]$-modules correspond to
    linear maps which \emph{interwine} $U$ and $V$, i.e., linear maps
    $\phi:V\to W$ such that $\phi\circ T=U\circ\phi$.
    \item $V_T$ and $V_U$ are isomorphic as $F[x]$-modules iff the linear
    operators $T$ and $U is $ are \emph{similar}, i.e., if there exists a linear
    isomorphism $\phi:V\to V$ such that $U=\phi\circ T\circ\phi^{-1}$.
    \item Given $(V,T)$, the space $V$ is f.d.\ over $F$ if and only if $V_T$
    is f.g.\ and torsion as an $F[x]$-module.
  \end{itemize}

  Given $(V,T)$ f.d., consider the annihilator ideal $\Ann(V_T)=(f)\subseteq
  F[x]$. By the classification theorem, we can write $V_T\cong\bigoplus_{k=1}^m
  R/(f_k)$ for some nonzero $f_k$, and therefore $0\neq f_1\cdots
  f_m\in\Ann(V_T)$, so that $f\neq 0$. We usually assume $f$ is monic, in which
  case we call $f$ the \emph{minimal polynomial} of $T$.
\end{definition}

\begin{proposition}
  Consider $(V,T)$ with $V$ f.d., and $f$ the minimal polynomial of $T$. For any $c\in V$ TFAE.
  \begin{enumerate}[listparindent=\parindent,parsep=5pt,label={(\arabic*)}]
    \item There exists $v\in V$ with $v\neq 0$ such that $Tv=cv$. That is, $c$
    is an eigenvalue of $T$.
    \item $f(c)=0$.
  \end{enumerate}
\end{proposition}

\begin{remark}
  Given any $F[x]$-module decomposition
  \[
    V_T\cong M_1\oplus\cdots\oplus M_m=F[x]/(f_1)\oplus\cdots\oplus F[x]/(f_m),
  \]
  we can give a block matrix representation of $T$ of the form 
  \[
    \left(\begin{array}{c|c|c|c}B_1 & & & \\ \hline & B_2 & & \\ \hline & & \ddots & \\ \hline & & & B_m\end{array}\right)
  \]
  by choosing an $F$-basis $e_1,\ldots,e_n$ of $V$, so that the first batch of
  basis elements are in $M_1$, the second batch in $M_2$, and so on. We'll
  describe some choices for cyclic modules.

  Given $V_T=F[x]/(f)$ with $f=x^k+b_{k-1}x^{k-1}+\cdots+b_1x+b_0$ a monic
  polynomial over $F$, we can use the basis
  \[
    e_1=\overline{1},\quad e_2=\overline{x},\quad\ldots,\quad e_i=\overline{x}^{i-1},\quad\ldots,\quad e_k=\overline{x}^{k-1}.
  \]
  Then the matrix describing the operator $T$ in this basis is the $k\times k$ \emph{companion matrix}
  \[
    C_f
    =\left(\begin{array}{cccccc}0 & 0 & \ldots & \ldots & 0 & -b_0 \\ 1 & 0 & \ldots & \ldots & 0 & -b_1 \\ 0 & 1 & \ldots & \ldots & 0 & -b_2 \\ \vdots & \vdots & \ddots & & \vdots & \vdots \\ \vdots & \vdots & & \ddots & \vdots & \vdots \\ 0 & 0 & \ldots & \ldots & 1 & -b_{k-1}\end{array}\right)
  \]
  A matrix is in \emph{rational canonical form} if it is a diagonal block
  matrix whose non-trivial bloxks are companion matrices
  $C_{f_1},\ldots,C_{f_m}$ for non-constant monic polynomials $f_k$ such that
  $f_1\mid f_2\mid\cdots\mid f_m$.
\end{remark}

\begin{theorem}[Rational canonical form]
  Given an operator $(V,T)$ on a f.d.\ vector space, there exists a basis
  w.r.t.\ which the matrix $A$ of $T$ is in rational canonical
  form. Furthermore, the rational canonical form of the matrix is unique.

  In particular, if the blocks of the rational canonical form of $T$ are the
  companion matrices associated to non-constant monic polynomials $f_1\mid
  f_2\mid\cdots\mid f_m$, then the $f_j$'s are called the \emph{invariant
  factors} of $T$, in the sense that
  \[
    V_T\cong\bigoplus_{j=1}^m F[x]/(f_j)
  \]
  is an invariant factor decomposition of the $F[x]$-module $V_T$.
\end{theorem}

\begin{remark}
  Note that the characteristic polynomial of the companion matrix is
  \[
    \det(xI-C_f)=f(x),
  \]
  and thus if $V_T\cong\bigoplus_{k=1}^mF[x]/(f_k)$ with $f_k$ monic, then the
  characteristic polynomial of $T$ is
  \[
    \det(xI-T)=f_1(x)\cdots f_m(x).
  \]
  If $f$ is the minimal polynomial of $T$, then $f_1\cdots f_m\in\Ann(V_T)=(f)$.
\end{remark}

Putting together the above results, we have the following result

\begin{proposition}
  Let $V$ be an $n$-dimensional $F$-vector space, and let $T:V\to V$ be a
  linear transformation. Then
  \begin{enumerate}[listparindent=\parindent,parsep=5pt,label={(\arabic*)}]
    \item The characteristic polynomial of $T$ is the product of all the
    invariant factors of $T$.
    \item (Cayley-Hamilton) The minimal polynomial of $T$ divides the
    characteristic polynomial of $T$.
    \item The characteristic polynomial of $T$ divides some power of the
    minimal polynomial of $T$. In particular, these polynomials have the same
    roots, not counting multiplicities.
  \end{enumerate}
\end{proposition}

Given the characteristic and minimal polynomials of a $2\times 2$ or $3\times
3$ matrix over $F$, the above proposition is completely enough to determine the
invariant factors of the matrix.

\begin{definition}
  If $V_T=F[x]/(x-c)^k$, then in terms of the basis
  \[
    e_1=(\overline{x}-c)^{k-1},\quad e_2=(\overline{x}-c)^{k-2},\quad\ldots,\quad e_{k-1}=\overline{x}-c,\quad e_k=1,
  \]
  the matrix describing $T$ is the $k\times k$ \emph{Jordan matrix}
  \[
    J_k(c):=\begin{pmatrix}
      c & 1 & 0 & \cdots & \cdots & 0 & 0 \\
      0 & c & 1 & \cdots & \cdots & 0 & 0 \\
      0 & 0 & c & \cdots & \cdots & 0 & 0 \\
      \vdots & \vdots & \vdots & \ddots & & \vdots & \vdots \\
      \vdots & \vdots & \vdots & & \ddots & \vdots & \vdots \\
      0 & 0 & 0 & \cdots & \cdots & c & 1 \\
      0 & 0 & 0 & \cdots & \cdots & 0 & c
    \end{pmatrix}
  \]
  with $c$'s along the diagonal, $1$'s along the first superdiagonal, and $0$'s
  elsewhere.

  Thus for an operator $T$ whose characteristic (or minimal) polynomial is a
  product of linear factors in $F[x]$ (e.g., if $F$ is algebraically closed),
  the elementary divisors of $T$ will all have the form $(x-c_i)^{k_i}$ with
  $c_i\in F$ and $k_i\geq 1$, in which case there exists a basis such that $T$
  is represented in \emph{Jordan canonical form}, i.e., as a diagonal block
  matrix whose blocks are Jordan matrices, and which is unique up to reordering
  the Jordan blocks.
\end{definition}

\section{Galois theory}

\begin{definition}
  Let $F\subseteq K$ be a field extension. Then we define the \emph{degree} of
  the extension $K/F$ to be
  \[
    [K:F]:=\dim_FK.
  \]
  The extension is \emph{finite} if $[K:F]<\infty$.
\end{definition}

\begin{proposition}[Tower law]
  Suppose we have field inclusions $F\subseteq K\subseteq L$. Then
  \[
    [L:F]=[L:K][K:F].
  \]
\end{proposition}

\begin{definition}
  Given fields $K$ and $L$, write $\Emb(K,L)$ for the set of ring homomorphisms
  (field embeddings) $K\hookrightarrow L$.
\end{definition}

\begin{definition}
  If $K/F$ and $L/F$ are field extensions, a \emph{homomorphisms of extensions}
  is a ring map $\phi:K\to L$ such that $\phi|_F=\id_F$.
\end{definition}

\begin{definition}
  Let $F$ be a field. Write $\Irred(F)\subseteq F[x]$ for the set of
  \emph{monic and irreducible} polynomials over $F$.
\end{definition}

\begin{remark}
  Let $f\in\Irred(F)$, and identify $F$ with its image under the canonical map
  $F\hookrightarrow K:=F[x]/(f)$. Then $K$ is a field (since $(f)$ is maximal
  in $F[x]$) and $[K:F]=\deg f$.
\end{remark}

\begin{definition}
  Let $F$ be a field, and define $F(x):=\Frac F[x]$. Then
  $[F(\alpha):F]=\infty$.
\end{definition}

\begin{definition}
  Given a field extension $K/F$ and a subset $S\subseteq K$, we write
  $F(S)\subseteq K$ for the subfield of $K$ generated by $F\cup S$, i.e.,
  $F(S)$ is the intersection of all subfields of $K$ containing $F$ and $S$.
\end{definition}

\begin{definition}
  An extension $K/F$ is \emph{simple} if $K=F(\alpha)$ for some element
  $\alpha\in K$.
\end{definition}

\begin{example}
  $\mathbb{Q}(\sqrt{2},\sqrt{3})=\mathbb{Q}(\sqrt{2}+\sqrt{3})$.
\end{example}
\begin{proof}
  Clearly we have
  $\mathbb{Q}(\sqrt{2},\sqrt{3})\supseteq\mathbb{Q}(\sqrt{2}+\sqrt{3})$ as
  $\mathbb{Q}(\sqrt{2},\sqrt{3})$ is a field containing $\mathbb{Q}$ and
  $\sqrt{2}+\sqrt{3}$. To see the opposite inclusion, we need to show that
  $\sqrt{2},\sqrt{3}\in\mathbb{Q}(\sqrt{2}+\sqrt{3})$. To see this, note
  \begin{align*}
    \frac{5}{2}\cdot\frac{1}{\sqrt{2}+\sqrt{3}}+\frac{1}{2}\left(\sqrt{2}+\sqrt{3}\right)
    &=\frac{5}{2}\cdot\frac{\sqrt{2}-\sqrt{3}}{5}+\frac{1}{2}\left(\sqrt{2}+\sqrt{3}\right) \\
    &=\frac{1}{2}\left(\sqrt{2}-\sqrt{3}+\sqrt{2}+\sqrt{3}\right)=\sqrt{2}.
  \end{align*}
  and
  \begin{align*}
    -\frac{5}{2}\cdot\frac{1}{\sqrt{2}+\sqrt{3}}+\frac{1}{2}\left(\sqrt{2}+\sqrt{3}\right)
    &=-\frac{5}{2}\cdot\frac{\sqrt{2}-\sqrt{3}}{5}+\frac{1}{2}\left(\sqrt{2}+\sqrt{3}\right) \\
    &=\frac{1}{2}\left(-\sqrt{2}+\sqrt{3}+\sqrt{2}+\sqrt{3}\right)=\sqrt{3}.\qedhere
  \end{align*}
\end{proof}

\begin{definition}
  Let $K/F$ be a field extension, and suppose $\alpha\in K$. Consider the
  subfield $F(\alpha)\subseteq K$ generated by $F$ over $\alpha$. Observe that
  we can always evaluate a polynomial $f\in F[x]$ at $\alpha$. There are two
  cases.
  \begin{enumerate}[listparindent=\parindent,parsep=5pt,label={(\arabic*)}]
    \item There exists a nonzero $f\in F[x]$ such that $f(\alpha)=0$. In this
    case we say that $\alpha$ is \emph{algebraic} over $F$.
    \item There does not exist a nonzero $f\in F[x]$ such that
    $f(\alpha)=0$. In this case we say that $\alpha$ is \emph{transcendental}
    over $F$.
  \end{enumerate}
\end{definition}

\begin{proposition}
  Let $\alpha\in K$ be algebraic over $F$. Then there exists a unique
  irreducible monic polynomial $m\in\Irred(F)$ such that
  $m(\alpha)=0$. Furthermore, a polynomial $f\in F[x]$ has a root iff $m\mid f$
  in $F[x]$.

  This $m$ is called the \emph{minimal polynomial} of $\alpha$ over $F$.
\end{proposition}

\begin{proposition}
  If $\alpha\in K$ is transcendental over $F$, then there is a unique isomorphism of $F$-extensions
  \[
    \phi:F(x)\xrightarrow{\sim}F(\alpha),\qquad\text{ such that }\phi(x)=\alpha.
  \]
  As a consequence, $[F(\alpha):F]$ is finite.
\end{proposition}

\begin{proposition}
  Suppose $K/F$ is a field extension with $K=F(\alpha)$ for some $\alpha\in K$. There are two cases:
  \begin{enumerate}[listparindent=\parindent,parsep=5pt,label={(\arabic*)}]
    \item $[K:F]<\infty$. Then $\alpha$ is algebraic over $F$, and there is a unique isomorphism of $F$-extensions of the form
    \[
      \phi:F[x]/(f)\to K,\qquad f=m_{\alpha/F}\in\Irred(F),\qquad \phi(\overline{x})=\alpha.
    \]
    \item $[K:F]=\infty$. Then $\alpha$ is transcendental over $F$, and there is a unique isomorphism of $F$-extensions of the form
    \[
      \phi:F(x)\to K,\qquad \phi(x)=\alpha.
    \]
  \end{enumerate}
\end{proposition}

\begin{definition}
  An extension $K/F$ is \emph{finite} if $[K:F]<\infty$. It is \emph{finitely
  generated} if $K=F(\alpha_1,\ldots,\alpha_n)$ for some finite list of
  elements $\alpha_1,\ldots,\alpha_n\in K$.
\end{definition}

\begin{proposition}
  Let $L/F$ be an extension and $\alpha_1,\ldots,\alpha_n\in L$ a finite list
  of elements. Let $K=F(\alpha_1,\ldots,\alpha_n)$. Then TFAE:
  \begin{enumerate}[listparindent=\parindent,parsep=5pt,label={(\arabic*)}]
    \item $[K:F]<\infty$.
    \item Every element $\beta\in K$ is algebraic over $F$ for all
    $k=1,\ldots,n$.
  \end{enumerate}
  Furthermore, if any of these hold, then $[K:F]\leq d_1\cdots d_n$, where
  $d_j$ is the degree of the minimal polynomial of $\alpha_j$ over $F$.
\end{proposition}

\begin{lemma}
  Let $F\subseteq K\subseteq L$ and $\alpha\in L$ such that $\alpha$ is algebraic over $F$ and $[K:F]<\infty$. Then 
  \[
    [K(\alpha):K]\leq[F(\alpha):F]
    \qquad\text{and}\qquad
    [K(\alpha):F(\alpha)]\leq[K:F].
  \]
\end{lemma}

\begin{definition}
  Given subfields $F\subseteq K,K'\subseteq L$, the \emph{composite extension}
  is the subfield of $L$ generated over $F$ by $K\cup K'$. It is usually
  written $KK'\subseteq L$.

  Clearly if $K=F(\alpha_1,\ldots,\alpha_m)$ and
  $K'=F(\beta_1,\ldots,\beta_n)$, then
  $KK'=F(\alpha_1,\ldots,\alpha_m,\beta_1,\ldots,\beta_n)$.
\end{definition}

\begin{proposition}
  If $K/F$ and $K'/F$ are subextensions of $L/F$ which are finite over $F$, then
  \[
    [KK':K]\leq[K':F],\qquad [KK':K']\leq[K:F],\qquad [KK':F]\leq[K:F][K':F].
  \]
\end{proposition}

\begin{definition}
  We say $K/F$ is \emph{algebraic} if every $\alpha\in K$ is algebraic in $F$.
\end{definition}

\begin{proposition}
  If $L/F$ is an extension and $\alpha,\beta\in L$ are algebraic over $F$, then
  $\alpha+\beta,\alpha\beta,-\alpha,\alpha^{-1}$ are algebraic over $F$.
\end{proposition}
\begin{proof}
  Since $F(\alpha)/F$ and $F(\beta)/F$ are finite extensions, the composite
  extension $F(\alpha,\beta)/F$ is also finite and thus algebraic. Since
  $\alpha+\beta$, $\alpha\beta$, $-\alpha$, and $\alpha^{-1}\in
  F(\alpha,\beta)$, these are algebraic elements.
\end{proof}

\begin{definition}
  An \emph{algebraic number} is an $\alpha\in\mathbb{C}$ which is algebraic
  over $\mathbb{Q}$, i.e., is the root of some nonzero $f\in\mathbb{Q}[x]$.

  Write $\mathbb{Q}^\alg$ for the set of algebraic numbers. Then
  $\mathbb{Q}^{\alg}$ is a subfield of $\mathbb{C}$ by the above proposition.
\end{definition}

\begin{remark}
  Given $r$ distinct primes $p_1,\ldots,p_r$, one can check that
  $[\mathbb{Q}(\sqrt{p_1},\ldots,\sqrt{p_r}):\mathbb{Q}]=2^r$. Thus it follows
  by the tower rule that $\mathbb{Q}^\alg/\mathbb{Q}$ is an infinite extension.
\end{remark}

\begin{proposition}
  If $F\subseteq K\subseteq L$ such that $K/F$ and $L/K$ are algebraic
  extensions, then $L/F$ is an algebraic extension.
\end{proposition}

\begin{definition}
  We say that a field $K$ is \emph{algebraically closed} if every nonconstant $f\in K[x]$ has a root in $K$.

  If $f$ has a root $c\in K$ then $f$ factors as $(x-c)g\in K[x]$. Thus if $K$
  is algebraically closed then every nonzero polynomial over $K$ \emph{splits}
  over $K$, i.e., is a product of degree $1$ polynomials.
\end{definition}

\begin{example}
  The complex numbers $\mathbb{C}$ form an algebraically closed field.
\end{example}

\begin{example}
  The field $\mathbb{Q}^\alg$ is algebraically closed. To see this, suppose
  $f\in\Irred(\mathbb{Q}^\alg)$. This has a root $\alpha\in\mathbb{C}$, and I
  want to show it is in $\mathbb{Q}^\alg$. But we have a sequence of algebraic
  extensions $\mathbb{Q}\subseteq
  \mathbb{Q}^\alg\subseteq\mathbb{Q}^\alg(\alpha)$, and therefore
  $\mathbb{Q}^\alg(\alpha)/\mathbb{Q}$ is algebraic, i.e.,
  $\alpha\in\mathbb{Q}^\alg$.
\end{example}

\begin{proposition}
  A field $K$ is algebraically closed iff for any algebraic extension $L/K$ we
  have $L=K$.
\end{proposition}

\begin{definition}
  An \emph{algebraic closure} is an extension $\overline{F}/F$ which is
  algebraic, and is such that every non-constant polynomial $f\in F[x]$ splits
  over $\overline{F}$, i.e., is a product of degree $1$ factors.
\end{definition}

\begin{proposition}
  Given an extension $K/F$, we have that $K$ is an algebraic closure of $F$ iff
  (i) $K/F$ is algebraic, and (ii) $K$ is algebraically closed. Algebraic
  closures are algebraically closed.
\end{proposition}

\begin{proposition}
  If $K/F$ is an extension and $K$ is algebraically closed, then $K$ contains a
  unique algebraic closure $\overline{F}$ of $F$, which is equal to the subset
  of elements which are algebraic over $F$.
\end{proposition}

\begin{proposition}
  Let $K/F$ be an extension with $\chr(F)\neq 2$ and $[K:F]=2$. Then
  $K=F(\sqrt{d})$ for some $d\in F$ which is not a square in $F$.
\end{proposition}
\begin{proof}
  Pick some $\alpha\in K\setminus F$, then since $[K:F]=2$, we have
  $K=F(\alpha)$. Let $f$ be the mimimal polynomial of $\alpha$ over $F$, so
  $f=x^2+bx+c$ for some $b,c,\in F$. Let $d=b^2-4c$. Then
  \[
    (2\alpha+b)^2
    =4\alpha^2+4b\alpha+b^2
    =4(-b\alpha-c)+4b\alpha+b^2
    =b^2-4c
    =d,
  \]
  so we can set $\sqrt{d}=2\alpha+b\in K$. Clearly $\sqrt{d}\notin F$, since
  otherwise we would have $\alpha=(-b+\sqrt{d})/2\in F$. Thus $K=F(\sqrt{d})$.
\end{proof}

\begin{remark}
  If $\chr F=2$, then if we try to do this it turns out that $\sqrt{d}=b\in
  F$, so it does not generate $K$ over $F$.
\end{remark}

\begin{definition}
  We say that a finite extension $K/F$ is \emph{2-radical} if there exists a
  finite tower of subfields of the form
  \[
    F=K_0\subseteq K_1\subseteq\cdots \subseteq K_r=K,\qquad [K_j:K_{j-1}]=2.
  \]
\end{definition}

\begin{example}
  In each of
  \[
    \mathbb{Q}\subseteq \mathbb{Q}(\sqrt{2})\subseteq \mathbb{Q}(\sqrt[4]{2})\subseteq\mathbb{Q}(\sqrt[8]{2})\subseteq\mathbb{Q}(\\sqrt{[16}]{2})\subseteq\mathbb{Q}(\sqrt[32]{2})\subseteq\cdots
  \]
  and
  \[
    \mathbb{Q}\subseteq\mathbb{Q}(\sqrt{2})\subseteq\mathbb{Q}\left(\sqrt{1+\sqrt{2}}\right)\subseteq\mathbb{Q}\left(\sqrt{1+\sqrt{1+\sqrt{2}}}\right)\subseteq\mathbb{Q}\left(\sqrt{1+\sqrt{1+\sqrt{1+\sqrt{2}}}}\right)\subseteq\cdots,
  \]
  each intermediate extension has degree $2$.
\end{example}

\begin{remark}
  If $K/F$ is $2$-radical then $[K:F]=2^r$ for some $r$, but the converse is
  not true.
\end{remark}

\begin{proposition}
  If $K/F$ and $K'/F$ are finite subextensions of $L/F$ which are $2$-radical,
  then the composite extension $KK'/F$ is $2$-radical.
\end{proposition}
\begin{proof}
  Factor $K/F$ as a sequence of degree $2$ extensions $K_j=K_{j-1}(\alpha_j)$
  with $K_0=F$ and $K_r=K$. Then we have a chain of extensions
  \[
    K'=K'K_0\subseteq K'K_1\subseteq K'K_2\subseteq\cdots\subseteq K'K_n=KK'.
  \]
  Each intermediate extension is a simple extension since
  $K'K_j=K'K_{j-1}(\alpha_j)$, and we know that $[K'K_j:K'K_{j-1}]\leq
  [K_j:K_{j-1}]=2$.
\end{proof}

\begin{definition}
  Given $F\subseteq\mathbb{C}$, let
  \[
    F^\mathrm{2rad}:=\bigcup_{\substack{F\subseteq L\subseteq\mathbb{C}\\ L/F\text{ is $2$-radical}}}L.
  \]
  Thus $\alpha\in F^{\mathrm{2rad}}$ iff there exists a finite $2$-radical
  extension $L/F$ with $\alpha\in L$. We see that $F^\mathrm{2rad}$ is a
  subfield of $\mathbb{C}$, using that if $L,L'$ are $2$-radical extensions
  over $F$ then so is $LL'$.
\end{definition}

\begin{definition}
  A field $K$ is said to be \emph{squareroot closed} if every element of $K$
  has a square root in $K$.
\end{definition}

\begin{proposition}
  For $F\subseteq\mathbb{C}$, the subfield $F^{\mathrm{2rad}}$ is the smallest
  subfield of $\mathbb{C}$ containing $F$ which is squareroot closed.
\end{proposition}

\begin{definition}
  Let $\mathcal{P}$ be a set of points in $\mathbb{C}$, and suppose
  $\mathcal{P}$ contains at least two points. Then given $\alpha\in\mathbb{C}$,
  we say $\alpha$ is \emph{construcible from $\mathcal{P}$} if $\alpha$ can be
  obtained as the intersection of lines and circles drawn as follows:
  \begin{itemize}
    \item you can draw a line between any two distinct points of $\mathcal{P}$,
    and
    \item you can draw a circle with center at a point $a$ of $\mathcal{P}$ and
    radius $r=|b-a|$ where $b$ is some other point belonging to $\mathcal{P}$.
  \end{itemize}
  We say $\alpha$ is just \emph{constructible} if it is constructable from
  $\mathcal{P}=\{0,1\}$.
\end{definition}

\begin{theorem}
  Let $\mathcal{P}\subseteq\mathbb{C}$, and suppose $\mathcal{P}$ contains $0$
  and $1$. Then a point $\alpha\in\mathbb{C}$ is construcible from
  $\mathcal{P}$ iff $\alpha\in F^{\mathrm{2rad}}$, where $F\subseteq\mathbb{C}$
  is the subfield generated by $\mathcal{P}$.
\end{theorem}

Here are some impossibility results which follow from this, using only the fact
that $\alpha\in\mathbb{Q}^{\mathrm{2rad}}$ must have
$[\mathbb{Q}(\alpha):\mathbb{Q}]=2^r$.
\begin{itemize}
  \item \emph{Cannot duplicate the cube}. Given $r$ we want to product
  $r\sqrt[3]{2}$, i.e., to construct $\alpha=\sqrt[3]{2}$. But
  $[\mathbb{Q}(\alpha):\mathbb{Q}]=3$.
  \item Cannot trisect every angle. In particular $\theta=2\pi/3$ cannot be
  trisected. This amounts to showing that $\zeta:=e^{2\pi i/9}$ is not
  constructible.

  We know that $\zeta^9=1$, but $\zeta^3\neq 1$. Since
  $0=\zeta^9-1=(\zeta^3-1)(\zeta^6+\zeta^3+1)$, we see that $\zeta$ is a root
  of $f=x^6+x^3+1\in\mathbb{Q}[x]$, so $[\mathbb{Q}(\zeta):\mathbb{Q}]\leq
  3$. Let $\alpha=\zeta+\zeta^{-1}\in\mathbb{Q}(\zeta)$. Using $f(\zeta)=0$,
  you can show that
  \[
    \alpha^3
    =\zeta^3+3\zeta+3\zeta^{-1}+\zeta^{-3}
    =3\alpha-1.
  \]
  So $\alpha$ is a root of $g=x^3-3x+1\in\mathbb{Q}[x]$. By the rational root
  test this has no root in $\mathbb{Q}$ so. Thus
  $[\mathbb{Q}(\alpha):\mathbb{Q}]=3$. Since
  $[\mathbb{Q}(\zeta):\mathbb{Q}]=[\mathbb{Q}(\zeta):\mathbb{Q}(\alpha)][\mathbb{Q}(\alpha):\mathbb{Q}]$,
  we see that $3$ divides $[\mathbb{Q}(\zeta):\mathbb{Q}]$.
  \item \emph{Cannot square the circle.} That is, given a circle with radius
  $r$, produce a square with side $\sqrt{\pi}r$. But $\sqrt{\pi}$ is not
  constructible. If it were, than it would be algebraic over $\mathbb{Q}$, and
  thus $\pi\in\mathbb{Q}(\sqrt{\pi})$ would be algebraic over $\mathbb{Q}$, but
  by Lindemann's theorem it is not.
  \item \textit{Cannot construct the regular heptagon.} Show that
  $\zeta:=e^{2\pi i/7}\in\mathbb{Q}^{\mathrm{2rad}}$. Its minimal polynomial
  over $\mathbb{Q}$ is $\Phi_7=x^6+\cdots+x+1$, so
  $[\mathbb{Q}(\zeta):\mathbb{Q}]=6$.
\end{itemize}

\begin{definition}
  Let $f\in F[x]$ with $f\neq 0$. A \emph{splitting field} of $f$ is an
  extension $\Sigma/F$ such that
  \begin{itemize}
    \item $f$ \emph{splits} over $\Sigma$, i.e.,
    $f=c(x-\alpha_1)\cdots(x-\alpha_n)$ for some
    $c,\alpha_1,\ldots,\alpha_n\in\Sigma$ with $c\neq 0$, and
    \item $\Sigma$ is generated over $F$ by the roots of $f$, i.e.,
    $\Sigma=F(\alpha_1,\ldots,\alpha_n)$. (Equivalently: the only subfield of
    $\Sigma$ over which $f$ splits is $\Sigma$ itself.)
  \end{itemize}
\end{definition}

\begin{proposition}
  If $L/F$ is an extension and $f\in F[x]$ splits over $L$, then the subfield
  $\Sigma=F(\alpha_1,\ldots,\alpha_n)$ generated by the roots of $f$ in $L$ is
  a splitting field of $f$.
\end{proposition}
\begin{proof}
  Obvious.
\end{proof}

\begin{example}
  If $f=(x^2+1)(x^2-5)\in\mathbb{Q}[x]$, then $\Sigma=\mathbb{Q}(i,\sqrt{5})$
  is a splitting field.
\end{example}

\begin{proposition}
  Let $F$ be a field. Then every nonzero $f\in F[x]$ admits a splitting field.
\end{proposition}

\begin{corollary}
  If $\Sigma/F$ is a splitting field of $f\in F[x]$, then $[\Sigma:F]\leq n!$,
  where $n=\deg f$.
\end{corollary}

\begin{example}[Cyclotomic extensions]
  Let $\zeta\in L$ be a primitive $n^\text{th}$ root of unity, i.e., an element
  of order $n$ in $L^\times$. Then for $F\subseteq L$, the subfield
  $K=F(\zeta)$ is the splitting field of $f=x^n-1$.

  This is because the elements $1,\zeta,\ldots,\zeta^{n-1}$ are pairwise
  distinct (since $|\zeta|=n$), and are all clearly roots of $f$ contained in
  $K$. Thus $f=(x-1)(x-\zeta)\cdots(x-\zeta^{n-1})$ and clearly $K$ is
  generated over $F$ by the roots.

  The degree of the extension $[F(\zeta):F]$ will be less than $n$ since
  $f=(x-1)g$ (unless $n=1$).

  Let $\zeta_{n}:=e^{2\pi i/n}\in\mathbb{C}$. The field $\mathbb{Q}(\zeta_n)$
  is called a \emph{cyclotomic field}. We know that if $n=p$ is prime, then
  $[\mathbb{Q}(\zeta_p):\mathbb{Q}]=p-1$, since $\Phi_p$ is irreducible over
  $\mathbb{Q}$.
\end{example}

\begin{example}
  Let $f=x^p-2\in\mathbb{Q}[x]$ where $p$ is a prime number. Note that $f$ is
  irreducible by Eisenstein's criterion.

  If $\alpha$ is a root of this (e.g., $\sqrt[p]{2}\in\mathbb{R}$), so is
  $\alpha\zeta^k$, where $\zeta$ is some fixed primitive $p^\text{th}$ root of
  unity. That is, the roots of $f$ in $\mathbb{C}$ are
  \[
    \alpha,\ \alpha\zeta,\ \alpha\zeta^2,\ \ldots,\ \alpha\zeta^{p-1}.
  \]
  As these are distinct (since $\zeta^k\neq 1$ if $p\nmid k$), thede are
  distinct roots of $f$, so $\Sigma=\mathbb{Q}(\alpha,\zeta)$ is a splitting
  field of $f$ (note that $\zeta=(\alpha\zeta)\alpha^{-1}$ can be written in
  terms of roots of $f$, so it must be in the splitting field). Since
  $[\Sigma:\mathbb{Q}(\zeta)]\leq[\mathbb{Q}(\alpha):\mathbb{Q}]=p$, we have
  \[
    [\Sigma:\mathbb{Q}]\leq (p-1)p,
  \]
  but it is necessarily divisible by both $p$ and $p-1$, so (since these are
  relatively prime) $[\Sigma:\mathbb{Q}]=p(p-1)$.
\end{example}

\begin{definition}
  Given a polynomial
  \[
    f=a_0+a_1x+\cdots+a_nx^n\in F[x],
  \]
  define its \emph{formal derivative} by the formula
  \[
    Df:=a_1+2a_2x+3a_3x^2+\cdots+na_nx^{n-1}\in F[x].
  \]
  It is straightforward to see that $D(f+g)=Df+Dg$, $D(fg)=(Df)g+f(Dg)$,
  $Dc=0$, and $D(cf)=cD(f)$ if $f,g\in F[x]$ and $c\in F$.
\end{definition}

\begin{definition}
  We say that a nonzero polynomial $f\in F[x]$ is \emph{separable} if $f$ and
  $Df$ are relatively prime in $F[x]$.
\end{definition} 

\begin{exercise}
  Show that if $f=gh\in F[x]$ and $f$ is separable than so are $g$ and $h$.
\end{exercise}
\begin{proof}
  Write $f=\sum_{j=0}^{n}a_jx^j$, then since $f$ and $Df$ are relatively prime,
  there exists $u,v\in F[x]$ such $uf+v(Df)=1$. Then we have
  \[
    (uh+vDh)g+vhDg
    =ugh+v((Dg)h+gDh)
    =ugh+vD(gh)
    =uf+vDf
    =1,
  \]
  so that indeed $g$ and $Dg$ are coprime, so $g$ is separable. Similarly $h$
  is separable, by symmetry.
\end{proof}

\begin{remark}
  If $F\subseteq K$ and $f\in F[x]$, then $f$ is separable over $F$ iff it is
  separable over $K$.

  To see this: If $f$ is separable over $F$, then $1=uf+vDf$ for some $u,v\in
  F[x]\subseteq K[x]$, so $f$ is separable over $K$ as well. On the other hand,
  if $f$ is separable over $K$, then any common divisor $d\in F[x]$ of
  $\{f,Df\}$ is also a common divisor of these in $K[x]$, so $d\in F^\times$,
  and thus $f$ is separable over $F$.
\end{remark}

\begin{exercise}
  Show that if $\phi:F\to K$ is a homomorphism of fields, then $f\in F[x]$ is a
  separable polynomial iff $\phi(f)\in K[x]$ is a separable polynomial.
\end{exercise}

\begin{proposition}
  Let $L/F$ be any extension over which $f\in F[x]$ splits. Then $f$ is
  separable iff $f$ has no multiple roots in $L$, iff $f$ and $Df$ have no
  common roots in $L$.
\end{proposition}

\begin{example}
  The polynomial $f=x^4+2x^2+1\in\mathbb{Q}[x]$ has $Df=4x^3+4x$. It is not
  hard to see (e.g., using the Euclidean algorithm) they have a common factor
  $x^2+1$. Thus $f$ is not separable. In fact, $f=(x^2+1)^2$ over $\mathbb{Q}$,
  and $f=(x-i)^2(x+i)^2$ over $\mathbb{C}$, so all roots are multiple.
\end{example}

\begin{example}
  The polynomial $f=x^n-1$ is separable over $\mathbb{Q}$ since $Df=x^{n-1}$
  and $x\nmid f$. Thus $f$ has $n$ distinct roots over $\mathbb{C}$, as we
  know.
\end{example}

\begin{proposition}
  A nonzero polynomial $f\in F[x]$ is separable iff for some irreducible
  factorization $f=g_1\cdots g_n$ over $F$, we have that (i) each $g_k$ is
  separable, and (ii) there are no repeated factors, i.e., if $i\neq j$ then
  $g_i\nmid g_j$.
\end{proposition}

\begin{proposition}
  Suppose $f\in F[x]$ is irreducible. Then $f$ is separable iff $Df\neq 0$.

  In particular, if $\chr F=0$, all irreducible polynomials over $F$ are
  separable.
\end{proposition}

\begin{example}
  Consider a field $F$ of prime characteristic $p$ and $f:=x^p-a\in F[x]$,
  where $a\in F$. Then $Df=px^{p-1}=0$, meaning $f$ is not separable.
\end{example}

\begin{example}
  Consider the field $F=\mathbb{F}_p(t)$ of rational functions over
  $\mathbb{F}_p$ and let $f=x^p-t$. Then $f$ is irreducible over $F$ but not
  separable.
\end{example}

\begin{proposition}
  Let $F(\alpha)/F$ be a finite extension, where $\alpha$ has minimal
  polynomial $m\in F[x]$. Suppose we are given an embedding of fields
  $\lambda:F \hookrightarrow F'$, and an extension $L/F'$. Let
  $m':=\lambda(m)\in F'[x]$. Then for any root $\beta\in L$ of $m'$, there
  exists a unique embedding $\mu:F(\alpha)\to L$ such that $\mu|_F=\lambda$ and
  $\mu(\alpha)=\beta$.
  % https://q.uiver.app/#q=WzAsNCxbMCwwLCJGKFxcYWxwaGEpIl0sWzEsMCwiTCJdLFsxLDEsIkYnIl0sWzAsMSwiRiJdLFswLDEsIlxcYWxwaGFcXG1hcHN0b1xcYmV0YSIsMCx7InN0eWxlIjp7InRhaWwiOnsibmFtZSI6Im1vbm8ifSwiYm9keSI6eyJuYW1lIjoiZGFzaGVkIn19fV0sWzEsMl0sWzAsM10sWzMsMiwiXFxsYW1iZGEiXSxbMCwxLCJcXG11IiwyLHsic3R5bGUiOnsidGFpbCI6eyJuYW1lIjoibW9ubyJ9LCJib2R5Ijp7Im5hbWUiOiJkYXNoZWQifX19XV0=
  \[\begin{tikzcd}
    {F(\alpha)} & L \\
    F & {F'}
    \arrow["{\alpha\mapsto\beta}", dashed, tail, from=1-1, to=1-2]
    \arrow["\mu"', dashed, tail, from=1-1, to=1-2]
    \arrow[from=1-1, to=2-1]
    \arrow[from=1-2, to=2-2]
    \arrow["\lambda", from=2-1, to=2-2]
  \end{tikzcd}\]
  In fact, there is a bijection
  % https://q.uiver.app/#q=WzAsNCxbMCwwLCJcXGxlZnRcXHtcXGJlZ2lue2FycmF5fXtjfVxcbXU6RihcXGFscGhhKVxcdG8gTFxcXFxcXHRleHR7c3VjaCB0aGF0IH1cXG11fF9GPVxcbGFtYmRhXFxlbmR7YXJyYXl9XFxyaWdodFxcfSJdLFsyLDAsIlxcbGVmdFxce1xcYmVnaW57YXJyYXl9e2N9XFxiZXRhXFxpbiBMXFxcXFxcdGV4dHtzdWNoIHRoYXQgfW0nKFxcYmV0YSk9MFxcZW5ke2FycmF5fVxccmlnaHRcXH0iXSxbMCwxLCJcXG11Il0sWzIsMSwiXFxtdShcXGFscGhhKSJdLFswLDEsIiIsMCx7InN0eWxlIjp7InRhaWwiOnsibmFtZSI6ImFycm93aGVhZCJ9fX1dLFsyLDMsIiIsMCx7InN0eWxlIjp7InRhaWwiOnsibmFtZSI6Im1hcHMgdG8ifX19XV0=
  \[\begin{tikzcd}[row sep=tiny]
    \begin{array}{c} \left\{\begin{array}{c}\mu:F(\alpha)\to L\\\text{such that }\mu|_F=\lambda\end{array}\right\} \end{array} && \begin{array}{c} \left\{\begin{array}{c}\beta\in L\\\text{such that }m'(\beta)=0\end{array}\right\} \end{array} \\
    \mu && {\mu(\alpha)}
    \arrow[tail reversed, from=1-1, to=1-3]
    \arrow[maps to, from=2-1, to=2-3]
  \end{tikzcd}\]
\end{proposition}
\begin{proof}
  Fix the data given in the setup of the proposition, so $F(\alpha)/F$ is a
  finite extension, $m\in F[x]$ is the minimal polynomial of $\alpha$,
  $\lambda:F\to F'$ is an embedding of fields, $L/F'$ is an extension, and
  $\beta\in L$ is a root of $m':=\lambda(m)$. Then we want to construct an
  embedding $\mu:F(\alpha)\to L$ which restricts to $\lambda$ on $F$ and
  satisfies $\mu(\alpha)=\beta$.

  Recall we have an isomorphism $F(\alpha)\cong F[x]/(m)$, so by the universal
  property of the quotient a map $F(\alpha)\to L$ is the data of a map $F[x]\to
  L$ whose kernel contains $(m)$. By the universal property of the polynomial
  ring, a map $F[x]\to L$ is the data of a map $F\to L$ and a chosen element to
  send $x$ to. Thus, there is a unique map $\widetilde{\mu}:F[x]\to L$ which
  restricts to the composition $F \xrightarrow{\lambda}F'\hookrightarrow L$ on
  $F$ and sends $x\mapsto\beta$. Moreover, clearly $m\in\ker\widetilde{\mu}$,
  as $\widetilde{\mu}(m)=\lambda(m)(\beta)=m'(\beta)=0$. Hence by the universal
  property of the quotient, $\widetilde{\mu}$ factors through a map
  $\mu:F(\alpha)\to L$ which restricts to $\lambda$ on $F$ and sends $\alpha$
  to $\beta$, and $\mu$ is the unique such map $F(\alpha)\to L$ with these
  properties.
\end{proof}

It is worth writing out the above proposition in the case $\lambda=\id_F$.

\begin{corollary}
  Let $F(\alpha)/F$ be a finite extension, where $\alpha$ has minimal
  polynomial $m\in F[x]$. Suppose we are given an extension $L/F$. Then for any
  root $\beta\in L$ of $m$, there exists a unique embedding
  $F(\alpha)\hookrightarrow L$ such that $\mu|_F=\id_F$ and
  $\mu(\alpha)=\beta$.
  % https://q.uiver.app/#q=WzAsMyxbMCwwLCJGKFxcYWxwaGEpIl0sWzIsMCwiTCJdLFsxLDEsIkYiXSxbMCwxLCJcXGFscGhhXFxtYXBzdG9cXGJldGEiLDAseyJzdHlsZSI6eyJ0YWlsIjp7Im5hbWUiOiJtb25vIn0sImJvZHkiOnsibmFtZSI6ImRhc2hlZCJ9fX1dLFswLDIsIiIsMix7InN0eWxlIjp7ImhlYWQiOnsibmFtZSI6Im5vbmUifX19XSxbMiwxLCIiLDIseyJzdHlsZSI6eyJoZWFkIjp7Im5hbWUiOiJub25lIn19fV0sWzAsMSwiXFxtdSIsMix7InN0eWxlIjp7InRhaWwiOnsibmFtZSI6Im1vbm8ifSwiYm9keSI6eyJuYW1lIjoiZGFzaGVkIn19fV1d
  \[\begin{tikzcd}
    {F(\alpha)} && L \\
    & F
    \arrow["{\alpha\mapsto\beta}", dashed, tail, from=1-1, to=1-3]
    \arrow["\mu"', dashed, tail, from=1-1, to=1-3]
    \arrow[no head, from=1-1, to=2-2]
    \arrow[no head, from=2-2, to=1-3]
  \end{tikzcd}\]
  That is, there is a bijection
  % https://q.uiver.app/#q=WzAsNCxbMCwwLCJcXGxlZnRcXHtcXGJlZ2lue2FycmF5fXtjfVxcbXU6RihcXGFscGhhKVxcdG8gTFxcXFxcXHRleHR7c3VjaCB0aGF0IH1cXG11fF9GPVxcaWRfRlxcZW5ke2FycmF5fVxccmlnaHRcXH0iXSxbMiwwLCJcXGxlZnRcXHtcXGJlZ2lue2FycmF5fXtjfVxcYmV0YVxcaW4gTFxcXFxcXHRleHR7c3VjaCB0aGF0IH1tKFxcYmV0YSk9MFxcZW5ke2FycmF5fVxccmlnaHRcXH0iXSxbMCwxLCJcXG11Il0sWzIsMSwiXFxtdShcXGFscGhhKSJdLFswLDEsIiIsMCx7InN0eWxlIjp7InRhaWwiOnsibmFtZSI6ImFycm93aGVhZCJ9fX1dLFsyLDMsIiIsMCx7InN0eWxlIjp7InRhaWwiOnsibmFtZSI6Im1hcHMgdG8ifX19XV0=
  \[\begin{tikzcd}
    \begin{array}{c} \left\{\begin{array}{c}\mu:F(\alpha)\to L\\\text{such that }\mu|_F=\id_F\end{array}\right\} \end{array} && \begin{array}{c} \left\{\begin{array}{c}\beta\in L\\\text{such that }m(\beta)=0\end{array}\right\} \end{array} \\
    \mu && {\mu(\alpha)}
    \arrow[tail reversed, from=1-1, to=1-3]
    \arrow[maps to, from=2-1, to=2-3]
  \end{tikzcd}\]
\end{corollary}

\begin{example}
  Let $f:=x^3-2\in\mathbb{Q}[x]$, and let $\alpha:=\sqrt[3]{2}$. Then there are
  \emph{three} distinct embeddings $\mu:\mathbb{Q}(\alpha)\to\mathbb{C}$,
  corresponding to the roots $\alpha$, $\alpha\omega$, and $\alpha\omega^2$,
  where $\omega=e^{2\pi i/3}$.
\end{example}

The above proposition has the following consequence.

\begin{remark}
  Given a finite extension $K/F$, we have a recipe for constructing
  homomorphisms of extensions $K\to L$ over $F$: Write $K/F$ as a composite of
  simple extensions
  \[
    F=K_0\subseteq K_1\subseteq K_2\subseteq\cdots \subseteq K_n,
    \qquad
    K_j=K_{j-1}(\alpha_j),
  \]
  and \emph{inductively} construct homomorphisms
  $\phi_j:F(\alpha_1,\ldots,\alpha_n)\to L$ extending $\phi_{j-1}$. At each
  step there is one choice: $\phi(\alpha_j)\in L$ can be any root of
  $m_{\alpha_j/K_{j-1}}$.
\end{remark}

\begin{proposition}
  Consider
  \begin{itemize}
    \item an isomorphism of fields $\lambda:F \xrightarrow{\sim}F'$,
    \item a nonzero polynomial $f\in F[x]$,
    \item a splitting field $\Sigma/F$ of $f$,
    \item an extension $L/F'$ over which $f':=\lambda(f)\in F'[x]$ splits.
  \end{itemize}
  Then there exists a homomorphism $\phi:\Sigma\to L$ such that
  $\phi|_F=\lambda$. The image $\phi(\Sigma)$ of $\phi$ is a splitting field of
  $f'$ over $F'$.
\end{proposition}
\begin{proof}
  We use induction on $\deg f$. If $f$ is constant then $\Sigma=F$ and we take
  $\phi=\lambda$. So suppose $\deg f\geq 1$, so that $\deg f'=\deg f\geq 1$.

  Let $\alpha_1$ be some root of $f$ in $\Sigma$, and let
  $m=m_{\alpha_1/F}\in\Irred(F)$ be its minimal polynomial. Then $f=mg$ for
  some $g\in F[x]$. Under $\lambda:F[x]\to F'[x]$ we get a factorization
  $f'=m'g'$ with $m'=\lambda(m)$. By the hypothesis $f'$ splits over $L$, so we
  choose a root $\beta_1\in L$ of $m'$. Then by the previous proposition there
  exists a unique field embedding $\phi_1:F(\alpha_1)\to F'(\beta_1)$ such that
  $\phi_1|_F=\lambda$ and $\phi_1(\alpha_1)=\beta_1$. It is straightforward to
  see that $\phi_1$ is in fact an isomorphism (the same proposition can be used
  ton construct a map $F'(\beta_1)\to F(\alpha_1)$ which restricts to
  $\lambda^{-1}$ on $F'$ and sends $\beta_1$ to $\alpha_1$, and it is
  straightforward to check that this map and $\phi_1$ are inverses).

  We can factor $f=(x-\alpha_1)h$ over $F(\alpha_1)$. Now note that we are in
  the same situation: We have
  \begin{itemize}
    \item an isomorphism of fields $\phi_1:F(\alpha_1)\to F'(\beta_1)$
    \item a nonezro polynomial $h\in F(\alpha_1)[x]$
    \item a splitting field $\Sigma/F(\alpha_1)$ of $h$, and
    \item an extension $L/F'(\alpha_1)$ over which $\phi_1(h)$ splits.
  \end{itemize}
  Since $\deg h<\deg f$, induction applies to produce the desired homomorphism
  $\phi$.
\end{proof}

\begin{corollary}
  Let $\Sigma/F$ and $\Sigma'/F$ be two splitting fields for the same nonzero
  polynomial $f\in F[x]$. Then $\Sigma$ and $\Sigma'$ are isomorphic as
  extensions of $F$.
\end{corollary}
\begin{proof}
  By the previous proposition there exists a homomorphism $\phi:\Sigma\to
  \Sigma'$ such that $\phi|_F=\id_F$, and the image $\phi(\Sigma)$ of $\phi$ is
  a splitting field of $f$ over $F$. Since $\phi(\Sigma)\subseteq \Sigma'$, $f$
  splits over $\phi(\Sigma)$, and $f$ splits over $\Sigma'$, it follows by the
  definition of a splitting field that $\phi(\Sigma)=\Sigma'$, so $\phi$ is an
  isomorphism over $F$, as desired.
\end{proof}

\begin{definition}
  Suppose $G\leq\Aut(K)$ is a group of automorphisms of a field $K$. Then the
  set $K^G:=\{\alpha\in K\mid g(\alpha)=\alpha\ \forall g\in G\}$ is a subfield
  of $K$, called the \emph{fixed field} of the action of the group $G$.
\end{definition}

\begin{proposition}
  Let $f\in F[x]$, and let $R:=\{\alpha\in K\mid f(\alpha)=0\}$ be the set of
  roots of $f$ in some extension $K/F$. Then any $\phi\in\Aut(K/F)$ restricts
  to a permutation of the set $R$, and this defines a group homomorphism
  \[
    \iota:\Aut(K/F)\to\Sym(R).
  \]
  Furthermore, if $K=F(R)$, then $\iota$ is injective, so $\Aut(K/F)$ is
  isomorphic to a subgroup of $\Sym(R)$.
\end{proposition}
\begin{proof}
  Let $\phi\in\Aut(K/F)$, and let $\alpha\in R$. Then we wish to show that $\phi(\alpha)\in R$. This is clear as
  \[
    f(\phi(\alpha))
    =\phi(f(\alpha))
    =\phi(0)
    =0,
  \]
  where the first equality follows because $\phi$ fixes $F$, which the
  coefficients of $f$ belong to.

  Now, suppose $K=F(R)$ and $\phi\in\ker\iota$, so $\phi(\alpha)=\alpha$ for
  all $\alpha\in R$, i.e., $\phi$ fixes all roots of $f$. Then $R\subseteq K^G$
  and $F\subseteq K^G$, where $G=\langle\phi\rangle\leq\Aut(K/F)$ is the cyclic
  subgroup generated by $\phi$. Since $K^G$ is a subsextension of $K=F(R)$
  containing both $F$ and $R$, we must have that $K^G=K$, so $\phi=\id$.
\end{proof}

We can use the techniques we have developed to compute the automorphism groups
of some field extensions.

\begin{example}
  Consider $g=x^3-2\in\mathbb{Q}[x]$. This factors
  \[
    g=(x-\alpha_1)(x-\alpha_2)(x-\alpha_3)=(x-\alpha)(x-\alpha\omega)(x-\alpha\omega^2),
  \]
  where $\alpha=\sqrt[3]{2}$ and $\omega=e^{2\pi i/3}$. Thus
  $\Sigma=\mathbb{Q}(\alpha_1,\alpha_2,\alpha_3)\subseteq\mathbb{C}$ is the
  splitting field.

  There are \emph{six} distinct embeddings $\Sigma\to\mathbb{C}$. One of them
  is the ``obvious'' inclusion, but there are others. We can construct them in
  two steps, and we will show that the image of any such embedding is $\Sigma$
  itself.

  Suppose we are constructing a map $\phi:\Sigma\to\mathbb{C}$. To start, we
  will construct its restriction $\phi_1:\mathbb{Q}(\alpha)\to\mathbb{C}$. From
  an earlier proposition, the data of such a map is a choice of a root of $g$,
  so there are three such maps $\phi_1:\mathbb{Q}(\alpha)\to\mathbb{C}$ which
  restrict to the identity on $\mathbb{Q}$, and they are uniquely determined by
  which of $\alpha$, $\omega\alpha$, and $\omega^2\alpha$ that $\phi_1$ sends
  $\alpha$ to.

  Over $\mathbb{Q}(\alpha)$ we have
  \[
    g=(x-\alpha)g_1,
    \qquad
    g_1=x^2+\alpha x+\alpha^2,
  \]
  so that the roots of $g_1$ in $\mathbb{C}$ are
  $\{\omega\alpha,\omega^2\alpha\}$. These are not real numbers, so they are
  not in $\mathbb{Q}(\alpha)\subseteq\mathbb{R}$. Thus $g_1$ is irreducible
  over $\mathbb{Q}(\alpha)$. If $\phi_1(\alpha)=\omega^k\alpha$, then
  $\phi_1(g_1)=x^2+\omega^k\alpha x+\omega^{2k}\alpha^2$, whose roots are
  $\{\alpha,\omega\alpha,\omega^2\alpha\}\setminus\{\omega^k\alpha\}$. Then the
  data of a map $\phi_2:\mathbb{Q}(\alpha,\omega\alpha)\to\mathbb{C}$ extending
  $\phi_1$ is precisely a choice of root belonging to
  $\{\alpha,\omega\alpha,\omega^2\alpha\}\setminus\{\phi_1(\alpha)\}$ to send
  $\omega\alpha$ to --- there are two such choices.

  Now, note that $\mathbb{Q}(\alpha,\omega\alpha)=\Sigma$, as clearly
  $\mathbb{Q}(\alpha,\omega\alpha)\subseteq\Sigma$, and
  $\omega=\omega\alpha/\alpha\in\mathbb{Q}(\alpha,\omega\alpha)$, so that
  $\omega^2\alpha=\omega\cdot\omega\alpha\in\mathbb{Q}(\alpha,\omega\alpha)$. 
  Hence $\phi_2=\phi$, and $\phi(\omega^2\alpha)$ must be the remaining root of
  $g$ other than $\phi(\alpha)$ and $\phi(\omega\alpha)$. Moreover, we have
  \[
    \phi(\Sigma)
    =\phi(\mathbb{Q}(\alpha_1,\alpha_2,\alpha_3))
    =\mathbb{Q}(\phi(\alpha_1),\phi(\alpha_2),\phi(\alpha_3))
    =\mathbb{Q}(\alpha_1,\alpha_2,\alpha_3)
    =\Sigma,
  \]
  so we've shown there are $3\cdot 2=6$ ways to construct an embedding
  $\Sigma\to\mathbb{C}$ over $\mathbb{Q}$, and the image of each of these
  embeddings is in fact $\Sigma$ itself. Hence $G:=\Aut(\Sigma/\mathbb{Q})$ has
  size $6$, and examnining the possible formulas for $\phi\in G$, we see that
  $G\cong S_3$, as we've shown for every $\sigma\in S_3$ there exists a unique
  $\phi\in G$ such that $\phi(\alpha_k)=\alpha_{\sigma(k)}$.
\end{example}

\begin{example}
  Consider $g=(x^2-2)(x^2-3)\in\mathbb{Q}$, with roots $\pm\sqrt{2}$ and
  $\pm\sqrt{3}$. The splitting field is
  $\Sigma=\mathbb{Q}(\sqrt{2},\sqrt{3})$. A similar argument to the previous
  argument yields that an isomorphism $\phi:\Sigma\to\Sigma$ are uniquely
  determined by a choice of $\phi(\sqrt{2})\in\{\pm\sqrt{2}\}$ and
  $\phi(\sqrt{3})\in\{\pm\sqrt{3}\}$ (since $x^3-2$ remains irreducible over
  $\mathbb{Q}(\sqrt{2})[x]$ as $\sqrt{3}\notin\mathbb{Q}(\sqrt{2})$).

  It can be seen that $G=\Aut(\Sigma/\mathbb{Q})\cong C_2\times C_2$.
\end{example}

\begin{example}
  Consider $g=x^3+x^2-2x-1\in\mathbb{Q}[x]$. We see $g\in\Irred(\mathbb{Q})$ by
  the rational root test, and since $g$ is separable (it is irreducible in
  characteristic zero) it has three distinct roots
  $\alpha_1,\alpha_2,\alpha_3\in\mathbb{C}$. Picking a root at random, we get
  three choices of homomorphisms $\phi:\mathbb{Q}(\alpha_1)\to\mathbb{C}$,
  determined by $\phi(\alpha_1)\in\{\alpha_1,\alpha_2,\alpha_3\}$.

  What may not be obvious is that $g$ already \emph{splits} over
  $\mathbb{Q}(\alpha_1)$, in fact, it turns out that the roots of this
  polynomial are
  \[
    \alpha_1=\zeta+\zeta^{-1},
    \qquad
    \alpha_2=\zeta^2+\zeta^{-2},
    \qquad
    \alpha_3=\zeta^3+\zeta^{-3},
  \]
  where $\zeta=e^{2\pi i/7}$ (these are actually all real numbers, all you need
  to check is that $\zeta^7=1$ and $\zeta\neq 1$). Using this you can check
  that: $\alpha_2=\alpha_1^2-2$, $\alpha_3=\alpha_2^2-2$. Thus
  $\Sigma=\mathbb{Q}(\alpha_1)$ is already a splitting field of $g$, and a
  homomorphism $\Sigma\to\Sigma$ is strictly determined by where it sends
  $\alpha_1$. Hence $[\Sigma:\mathbb{Q}]=3$ and $G=\Aut(\Sigma/\mathbb{Q})\cong
  C_3$.
\end{example}

\begin{example}
  $x^4-2\in\mathbb{Q}[x]$. Here the roots are $\{\pm\alpha,\pm i\alpha\}$,
  where $\alpha=\sqrt[4]{2}$. In this case, we can show that
  $[\Sigma:\mathbb{Q}]=8$ and $G=\Aut(\Sigma/\mathbb{Q})\cong D_8$. This can be seen using the chain of extensions
  \[
    \mathbb{Q}\subseteq\mathbb{Q}(\alpha)\subseteq\mathbb{Q}(\alpha,i),
    \qquad
    [\mathbb{Q}(\alpha):\mathbb{Q}]=4,
    \qquad
    [\mathbb{Q}(\alpha,i):\mathbb{Q}(\alpha)]=2.
  \]
  (Note that $\mathbb{Q}(\alpha)\subseteq\mathbb{R}$ so
  $i\notin\mathbb{Q}(\alpha)$).
\end{example}

\begin{definition}
  An algebraic extension $L/F$ is \emph{normal} if every $f\in\Irred(F)$ which
  has a root in $L$ splits in $L$.
\end{definition}

\begin{example}
  $\mathbb{Q}^\alg/F$ for any subfield $F\subseteq\mathbb{Q}^\alg$ is a normal
  extension, since it is an algebraic extension and all polynomials over
  $\mathbb{Q}^\alg$ (and hence over $F$) split in $\mathbb{Q}^\alg$.
\end{example}

\begin{example}
  Every degree $2$ extension is normal.
\end{example}
\begin{proof}
  Let $L/F$ be normal with $[L:F]=2$, and choose $f\in\Irred(F)$ and suppose
  $f$ has a root $\alpha\in L$. If $\alpha\in F$ then since $f$ is irreducible
  and monic over $F$ we must have $f=x-\alpha$, so $f$ splits over $F$ (and
  therefore $L$) as desired. If $\alpha\in L\setminus F$, then by a degree
  argument we must have that $L=F(\alpha)$.%
  \footnote{
    We have $[F(\alpha):F]>1$ since $F(\alpha)\neq F$, so by the tower rule it
    follows $2=[L:F(\alpha)][F(\alpha):F]>[L:F(\alpha)]$, meaning
    $[L:F(\alpha)]=1$, so $L=F(\alpha)$.
  }
  Then since $f$ is the minimal polynomial of $\alpha$, we have
  $L=F(\alpha)\cong F[x]/(f)$ has degree $2$ over $F$, so $\deg f=2$. Then by
  the division algorithm, since $\alpha\in L$ is a root of $f$, $f$ factors as
  $f=(x-\alpha)g$ for some $g\in L[x]$. But since $\deg f=2$, we must have
  $\deg g=1$, so $f$ splits as desired.
\end{proof}

\begin{example}
  $\mathbb{Q}(\sqrt[3]{2}/\mathbb{Q}$ is not normal, since $f=x^3-2$ does not
  split over $\mathbb{Q}$.
\end{example}

\begin{theorem}
  A finite extension $L/F$ is normal iff it is a splitting field for some
  polynomial $f\in F[x]$.
\end{theorem}

\begin{proof}[Proof part 1: Finite normal extensions are splitting fields:]
  If $L/F$ is a finite normal extension then
  \[
    L=F(\alpha_1,\ldots,\alpha_n)
  \]
  for some finite list of elements $\alpha_1,\ldots,\alpha_n$ with minimal
  polynomials $m_1,\ldots,m_n\in F[x]$. Let $f=\prod_{k=1}^{n}m_k\in F[x]$ be
  their product. Normality of $L/F$ says that each $m_k$ splits over $L$, and
  thus $f$ does as well. Since $L/F$ is generated by the $\alpha_k$'s, it is
  clear that $L$ is a splitting field of $f$.
\end{proof}

We get the second part as a special case of a more general claim.

\begin{lemma}
  Suppose $F\subseteq L\subseteq M$, where $L$ is a slitting field of some
  $f\in F[x]$. If $\alpha,\beta\in M$ are roots of the same irreducible
  polynomial $g\in\Irred(F)$, then $[L(\alpha):L]=[L(\beta):L]$.
\end{lemma}
\begin{proof}
  By the tower rule,
  \[
    [L(\alpha):L]
    =\frac{[L(\alpha):F]}{[L:F]}
    =\frac{[L(\alpha):F(\alpha)][F(\alpha):F]}{[L:F]}
  \]
  Thus it suffices to show that $[L(\alpha):F(\alpha)]=[L(\beta):F(\beta)]$ and
  $[F(\alpha):F]=[F(\beta):F]$.

  To see the latter, note that by an above proposition, since $g$ is the
  minimal polynomial of $\alpha$ and $\beta$, there is an isomorphism of
  $F$-extensions $\phi:F(\alpha)\to F(\beta)$ sending $\alpha\mapsto\beta$ (both are
  isomorphic to $F[x]/(g)$), so it is true that $[F(\alpha):F]=[F(\beta):F]$.
  
  To see the former, note that $\phi(f)=f$, since $f\in F[x]$. Moreover, both
  $L(\alpha)/F(\alpha)$ and $L(\beta)/F(\beta)$, being generated over the
  ground fields by roots of $f$, are splitting fields of $f$. Thus by previous
  theory $\phi$ extends to an isomorphism $L(\alpha)\to L(\beta)$, so that
  $[L(\alpha):F(\alpha)]=[L(\beta):F(\beta)]$, as desired.
\end{proof}

\begin{proof}[Proof of part 2: splitting fields are normal extensions]
  Suppose $L/F$ is a splitting field of $f\in F[x]$, and $g\in\Irred(F)$ is
  some irreducible polynomial with root $\alpha\in L$. Form a splitting field
  $\Sigma/L$ of the polynomial $g\in F[x]\subseteq L[x]$. If $\beta$ is any root of $g$ in $\Sigma$, the previous lemma says
  \[
    [L(\alpha):L]=[L(\beta):L].
  \]
  But $\alpha\in L$ so these are $1$, so $\beta\in L$. Thus all roots of $g$
  are in $L$, so $g$ splits over $L$ as desired.
\end{proof}

\begin{corollary}
  A splitting field $\Sigma/F$ is a splitting field for \emph{any}
  $f\in\Irred(F)$ which has a root in $\Sigma$.
\end{corollary}

We can generalize this to infinite extensions.

\begin{theorem}
  An algebraic extension $L/F$ is normal iff it is a splitting field for a set
  $S\subseteq F[x]\setminus\{0\}$ of polynomials.
\end{theorem}
\begin{proof}
  $\implies$ Let $S$ be the set of minimal polynomials of all elements in
  $L/F$, then $L=F(S)$ and every $m\in S$ splits over $L$ since the extension
  is normal.

  $\impliedby$ Suppose $L/F$ is a splitting field of a set of polynomials
  $S\subseteq F[x]\setminus\{0\}$, so $L$ is generated over $F$ by the roots of
  all $f\in L$. Given $\alpha\in L$ and $g\in\Irred(F)$ with $g(\alpha)=0$, we
  see that $\alpha$ must be contained in a subfield generated by a finite set
  of roots, so $\alpha\in F(R_f)\subseteq L$, where $f=f_1\cdots f_k$ for some
  finite list $f_1,\ldots,f_k\in S$. Since $F(R_f)/F$ is a splitting field of
  $f$, it is normal so $g$ splits over $F(R_f)$, and hence over $L$.
\end{proof}

\begin{proposition}
  Consider fields $F\subseteq K\subseteq L$. If $L/F$ is normal then $L/K$ is
  normal.
\end{proposition}
\begin{proof}
  First note that if $L/F$ is algebraic then so is $L/K$. Now suppose
  $f\in\Irred F$ with a root $\alpha\in L$. Then since $K/F$ is algebraic there
  is a minimal polynomial $g=m_{\alpha/F}\in F[x]$ of $\alpha$ over $F$. Since
  $g$ has $\alpha$ as a root, over $K$ we must have $f\mid g$. Since $L/F$ is
  normal we have that $g$ splits over $\Sigma$, and therefore its factor $f$
  splits over $\Sigma$.
\end{proof}

\begin{remark}
  It is \emph{not} true that $L/F$ being normal implies $K/F$ is normal. For
  instance, $F=\mathbb{Q}$, $K=\mathbb{Q}(\alpha)$,
  $L=\mathbb{Q}(\alpha,\omega)$ with $\alpha=\sqrt[3]{2}$ and $\omega=e^{2\pi
  i/3}$.
\end{remark}

\begin{remark}
  It is not true that $L/K$ and $K/F$ normal imply $L/F$ is normal For
  instance, $F=\mathbb{Q}$, $K=\mathbb{Q}(\sqrt{2})$, and
  $L=\mathbb{Q}(\sqrt[4]{2})$. Both $L/K$ and $K/F$ are degree $2$ and hence
  normal, but $L/F$ is not normal since the minimal polynomial $x^4-2$ of
  $\sqrt[4]{2}$ does not split over $L\subseteq\mathbb{R}$.
\end{remark}

\section{Exercises}

\begin{lemma}\label{p-group_has_nontrivial_center}
  Let $G$ be a $p$-group for some prime $p$. Then $p$ divides $|Z(G)|$, and in
  particular $Z(G)$ is nontrivial.
\end{lemma}
\begin{proof}
  Since $G$ is a $p$-group, we may write $|G|=p^a$ for some $a\in\mathbb{N}$,
  i.e., $a\geq 1$. Then by the class equation, we have that
  \[
    |G|=|Z(G)|+\sum_{j=1}^{r}[G:C_G(g_j)],
  \]
  where $g_1,\ldots,g_r$ are representatives of the conjugacy classes of $G$,
  and each $[G:C_G(g_j)]$ is $>1$ and divides $|G|$, say
  $[G:C_G(g_j)]=p^{m_j}$, where $1<m_j$. Thus, we have
  \[
    p^a=|Z(G)|+\sum_{j=1}^{r}p^{m_j}
    \implies |Z(G)|=p^a-\sum_{j=1}^{r}p^{m_j},
  \]
  and the RHS is clearly divisible by $p$, so that $p$ divides $|Z(G)|$ as
  well, yielding the desired result.
\end{proof}

\begin{lemma}\label{product_of_distinct_primes_is_cyclic}
  Let $p_1,\ldots,p_r$ be distinct primes, and $m_1,\ldots,m_r$ be positive
  integers. Set $m:=p_1^{m_1}\cdots p_r^{m_r}$. Then
  \[
    \mathbb{Z}/m\cong\bigoplus_{j=1}^r\mathbb{Z}/p_j^{m_j}.
  \]
\end{lemma}
\begin{proof}
  Let
  \[
    G:=\bigoplus_{j=1}^r\mathbb{Z}/p_j^{m_j},
  \]
  and for $j=1,\ldots,r$, let $a_j\in G$ be a generator of the $j^\text{th}$
  summand, so that $|a_j|=p_j^{m_j}$. Let $x:=a_1\cdots a_r$, so that
  \[
    |x|
    =\lcm(a_1,\ldots,a_r)
    =\lcm(p_1^{m_1},\ldots,p_r^{m_r}).
  \]
  Since each of the $p_j$'s are distinct primes, it follows that
  $|x|=p_1^{m_1}\cdots p_r^{m_r}=m$. Thus $G$ has an element of order $m=|G|$,
  so $G$ is cyclic of order $m$, as desired.
\end{proof}

\begin{lemma}\label{orb_same_size_iff_stab_same_size}
  Let $G$ be a finite group acting on a finite set $X$, and suppose $x,y\in
  X$. Then $|\Orb(x)|=|\Orb(y)|\iff |\Stab(x)|=|\Stab(y)|$.
\end{lemma}
\begin{proof}
  If $|\Orb(x)|=|\Orb(y)|$, then by the Orbit/Stabilizer Theorem we have
  \[
    |\Stab(x)|
    =|G|[G:\Stab(x)]
    =|G||\Orb(x)|
    =|G||\Orb(y)|
    =|G|[G:\Stab(y)]
    =|\Stab(y)|.
  \]
  On the other hand, if $|\Stab(x)|=|\Stab(y)|$, we have
  \[
    |\Orb(x)|
    =[G:\Stab(x)]
    =|G|/|\Stab(x)|
    =|G|/|\Stab(y)|
    =[G:\Stab(y)]
    =|\Orb(y)|.\qedhere
  \]
\end{proof}

\begin{lemma}[Burnside's Lemma]\label{burnside}
  Let $G$ be a finite group acting on a finite set $X$. Then 
  \[
    |X/G|=\frac{1}{|G|}\sum_{g\in G}|X^g|,
  \]
  where $X/G$ denotes the collection of $G$-orbits in $X$, and given $g\in G$,
  $X^g:=\{x\in X\mid g\cdot x=x\}$. 
\end{lemma}
\begin{proof}
  First of all, note that
  \[
    \sum_{g\in G}|X^g|
    =\left|\left\{(g,x)\in G\times X\mid g\cdot x=x\right\}\right|
    =\sum_{x\in X}|\Stab(x)|,
  \]
  By the Orbit/Stabilizer theorem, we have $|\Stab(x)|=|G|/|\Orb(x)|$, so that
  \[
    \frac{1}{|G|}\sum_{g\in G}|X^g|
    =\frac{1}{|G|}\sum_{x\in X}\frac{|G|}{|\Orb(x)|}
    =\sum_{x\in X}\frac{1}{|\Orb(x)|}.
  \]
  Finally, writing $X$ as the disjoint union of its orbits in $X/G$, we have
  \[
    \frac{1}{|G|}\sum_{g\in G}|X^g|
    =\sum_{A\in X/G}\sum_{x\in A}\frac{1}{|A|}
    =\sum_{A\in X/G}1
    =|X/G|.\qedhere
  \]
\end{proof}

\begin{lemma}\label{aut_of_prod_of_coprimes}
  let $P$ and $Q$ be finite groups of coprime order. Then $\Aut(P\times
  Q)\cong\Aut(P)\times\Aut(Q)$.
\end{lemma}
\begin{proof}
  There is a canonical map
  \[
    \Aut(P)\times\Aut(Q)\to\Aut(P\times Q)
  \]
  sending a pair $(\sigma,\tau)$ to the automorphism $\sigma\times\tau$ defined
  by $(\sigma\times)\tau(p,q)=(\sigma(p),\tau(q))$. It is straightforward to
  verify that $\sigma\times\tau$ is an automorphism of $P\times Q$ and that
  this assignment is an injective homomorphism. It remains to show the
  assignment is surjective.

  Now, let $x\in P$, and write $\eta(x,e)=(p,q)$, where $p\in P$ and $q\in Q$. Then since $\eta$ is a homomorphism, we have
  \[
    (e,e)
    =\eta(e,e)
    =\eta((x,e)^{|x|})
    =(p^{|x|},q^{|x|}),
  \]
  so that $q^{|x|}=e$. Thus $|q|$ divides $|x|$, say $|x|=n|q|$. By Lagrange's,
  $|x|=n|q|$ divides $|P|$ and $|q|$ divides $|Q|$, so $|q|$ is a common factor
  of $|P|$ and $|Q|$. Yet $|P|$ and $|Q|$ are coprime, so it follows that
  $|q|=1$, which means $q=e$. Thus we've shown that
  $\eta(P\times\{e\})\subseteq P\times\{e\}$. A similar argument yields that
  $\eta(\{e\}\times Q)\subseteq\{e\}\times Q$. Now let $\sigma$ and $\tau$
  denote the compositions which fit into the following diagram
  % https://q.uiver.app/#q=WzAsNixbMSwwLCJQXFx0aW1lcyBRIl0sWzEsMSwiUFxcdGltZXMgUSJdLFswLDEsIlAiXSxbMiwxLCJRIl0sWzAsMCwiUCJdLFsyLDAsIlEiXSxbMCwxLCJcXGV0YSIsMl0sWzEsMiwiIiwwLHsic3R5bGUiOnsiaGVhZCI6eyJuYW1lIjoiZXBpIn19fV0sWzEsMywiIiwwLHsic3R5bGUiOnsiaGVhZCI6eyJuYW1lIjoiZXBpIn19fV0sWzQsMCwiIiwwLHsic3R5bGUiOnsidGFpbCI6eyJuYW1lIjoiaG9vayIsInNpZGUiOiJ0b3AifX19XSxbNSwwLCIiLDAseyJzdHlsZSI6eyJ0YWlsIjp7Im5hbWUiOiJob29rIiwic2lkZSI6ImJvdHRvbSJ9fX1dLFs0LDIsIlxcc2lnbWEiLDIseyJzdHlsZSI6eyJib2R5Ijp7Im5hbWUiOiJkYXNoZWQifX19XSxbNSwzLCJcXHRhdSIsMCx7InN0eWxlIjp7ImJvZHkiOnsibmFtZSI6ImRhc2hlZCJ9fX1dXQ==
  \[\begin{tikzcd}
    P & {P\times Q} & Q \\
    P & {P\times Q} & Q
    \arrow[hook, from=1-1, to=1-2]
    \arrow["\sigma"', dashed, from=1-1, to=2-1]
    \arrow["\eta"', from=1-2, to=2-2]
    \arrow[hook', from=1-3, to=1-2]
    \arrow["\tau", dashed, from=1-3, to=2-3]
    \arrow[two heads, from=2-2, to=2-1]
    \arrow[two heads, from=2-2, to=2-3]
  \end{tikzcd}\]
  where the top arrows denote the identifications $P\cong P\times\{e\}$ and
  $Q\cong\{e\}\times Q$. Then given $p\in P$ and $q\in Q$, it follows that
  \[
    \eta(p,q)=\eta(p,e)\eta(e,q)=(\sigma(p),e)(e,\tau(q))=(\sigma(p),\tau(q)),
  \]
  where the middle equality is where we used the fact that
  $\eta(P\times\{e\})\subseteq P\times\{e\}$ and $\eta(\{e\}\times
  Q)\subseteq\{e\}\times Q$. Thus we've shown that $\eta=\sigma\times\tau$. It
  is straightforward to see that $\eta$ is not injective (resp.\ surjective)
  unless $\sigma$ and $\tau$ are, so we have shown the desired result.
\end{proof}

\begin{lemma}\label{p-Sylow_subgroups_of_product}
  Suppose $G$ and $H$ are finite groups and $p$ a prime dividing $|G|$ but not
  $|H|$. Then there is a bijection
  \[
    \Syl_p(G)\xrightarrow{\sim}\Syl_p(G\times H)
    \qquad\text{given by}\qquad
    K\mapsto K\times\{e\}.
  \]
  In particular $n_p(G)=n_p(G\times H)$.
\end{lemma}
\begin{proof}
  Let $K\in\Syl_p(G)$, and identify $K$ with $K\times\{e\}\leq G\times
  H$. Since $p$ does not divide $|H|$, $K$ is also a $p$-Sylow subgroup of
  $G\times H$. Thus by Sylow 2, every $p$-Sylow subgroup of $G\times H$ is
  conjugate to $K$. Clearly any conjugate of $K\times\{e\}$ by an element of
  $G\times H$ lands in $G\times\{e\}$, so every element of $\Syl_p(G\times H)$
  is of the form $L\times\{e\}$ for a unique $L\in\Syl_p(G)$, as desired.
\end{proof}

\begin{lemma}\label{p-Sylow_subgroups_of_GxZpn}
  Suppose $G$ is a finite group, $p$ is a prime number, and $n$ is a positive integer. Then there is a bijection
  \[
    \Syl_p(G)\to\Syl_p(G\times\mathbb{Z}/p^n)
    \qquad\text{given by}\qquad
    K\mapsto K\times\mathbb{Z}/p^n.
  \]
  In particular $n_p(G)=n_p(G\times\mathbb{Z}/p^n)$.
\end{lemma}
\begin{proof}
  Clearly if $K$ is a $p$-Sylow subgroup of $G$ then $K\times\mathbb{Z}/p^n$ is
  a $p$-Sylow subgroup of $H:=G\times\mathbb{Z}/p^n$. Thus by Sylow 2 every
  $p$-Sylow subgroup of $H$ is a conjugate of $K\times\mathbb{Z}/p^n$, and
  clearly any conjugate of $K\times\mathbb{Z}/p^n$ is of the form
  $L\times\mathbb{Z}/p^n$ for some subgroup $L\leq G$ satisfying $|L|=|K|$
  (since conjugating $A\times B$ by $(a,b)$ is the same as first conjugating
  $A$ by $a$ and $B$ by $b$ and then taking their product).
\end{proof}

\begin{lemma}\label{units_of_Z/p^k_are_cyclic}
  The multiplicative group of units $(\mathbb{Z}/p^k)^\times$ is
  cyclic. Moreover, if $k\geq 2$, given a generator $n$ of
  $(\mathbb{Z}/p^{k-1})^\times$, there exists some $m\in\mathbb{Z}^{\geq 0}$
  such that $n+p^{k-1}m$ is a generator of $(\mathbb{Z}/p^k)^\times$.
\end{lemma}
\begin{proof}
  This result is outside the scope of a standard algebra class, and requires
  Hensel's lemma. However, if you are asked to prove that
  $(\mathbb{Z}/p^k)^\times$ is cyclic for some specific prime $p$ and integer
  $k\geq 2$, it can be useful to know the statement.
\end{proof}

\begin{lemma}\label{Sylow_subgroups_are_normal_implies_prod_of_Sylows}
  Let $G$ be a finite group such that $n_p(G)=1$ for each prime $p$ dividing
  $|G|$. Then $G$ is isomorphic to a product of its Sylow subgroups, i.e., $G$
  is a product of $p$-groups.
\end{lemma}
\begin{proof}
  Write $|G|=p_1^{n_1}\cdots p_k^{n_k}$ (where the $p_i$'s are distinct primes
  and the $n_i$'s are positive integers), so that for $i=1,\ldots,k$ $G$ admits
  a unique subgroup $H_i$ of order $p_i^{n_i}$ (which is normal by Sylow
  2). Then we wish to show that
  \begin{equation}\label{eq2}
    G\cong H_1\times\cdots\times H_k.
  \end{equation}
  For $i=1,\ldots,k$, define
  \[
    K_i:=H_1H_2\cdots H_{i-1}H_{i+1}\cdots H_k,
  \]
  i.e. $K_i$ is the product of all the $H_j$'s for $j\neq i$. Then since each
  $H_i$ is normal, in order for \autoref{eq2} to hold, by the product
  recognition theorem it suffices to show that
  \begin{itemize}
    \item $H:=H_1H_2\cdots H_k=G$, and 
    \item $H_i\cap K_i=\{e\}$ for $i=1,\ldots, k$.
  \end{itemize}
  To see the former, note that by the second isomorphism theorem $H_i$ is a
  subgroup of $H$ for each $i$, so that in particular $|H_i|=p_i^{n_i}$ divides
  $|H|$ for each $i$. Since the $p_i$'s are distinct primes, it follows that
  $|H|=p_1^{n_1}\cdots p_k^{n_k}=|G|$, so that $H=G$, as desired.

  Now, fix some $i\in\{1,\ldots,k\}$, and note that $H_i\cap K_i$ is a subgroup of both $H_i$ and $K_i$, so by Lagrange's the order of $H_i\cap K_i$ divides both $|H_i|$ and $|K_i|$. Note that
  \[
    |K_i|
    \leq\prod_{\substack{j=1,\ldots,k\\j\neq i}}|H_j|
    =\prod_{\substack{j=1,\ldots,k\\j\neq i}}p_j^{n_j},
  \]
  but also for $i\neq j$, $H_j$ is a subgroup of $K_i$, so that $|H_j|=p_j^{n_j}$ must divide the order of $K_i$. Again since the $p_j$'s are distinct primes, it follows that
  \[
    |K_i|\geq\prod_{\substack{j=1,\ldots,k\\j\neq i}}p_j^{n_j},
  \]
  so $|K_i|=\prod_{\substack{j=1,\ldots,k\\j\neq i}}p_j^{n_j}$. Thus since
  $|H_i\cap K_i|$ has to divide both $\prod_{\substack{j=1,\ldots,k\\j\neq
  i}}p_j^{n_j}$ and $p_i^{n_i}$, which have no common factors, it follows that
  $|H_i\cap K_i|=1$, so that $H_i\cap K_i=\{e\}$, as desired.
\end{proof}

\begin{enumerate}[listparindent=\parindent,parsep=5pt,label={\arabic*.}]
  \item (May 2022 Q1) \begin{enumerate}[listparindent=\parindent,parsep=5pt,label={(\alph*)}]
    \item Let $H$ be a subgroup of a group $G$. Then $G$ acts on the set
    $G/H=\{gH\mid g\in G\}$ by left multiplication. This action naturally
    determines a homomorphism $\alpha:G\to S(G/H)$, where $S(X)$ is the group
    of permutations on a set $X$. Prove that the kernel of $\alpha$ is
    contained in $H$.
    \begin{proof}
      If $H=G$ we are done, so suppose $H$ is a proper subgroup of $G$. Then it
      suffices to show that if $x\in G\setminus H$, then
      $x\notin\ker\alpha$. This is clear, as if $x\notin H$, then $xH\neq H$,
      so that in particular $\alpha(x)(eH)=xH\neq eH$, meaning $\alpha(x)$ is
      not trivial, so $x\notin\ker\alpha$.
    \end{proof}
    \item Let $L$ be a subgroup of a finite group $K$ such that $[K:L]=p$,
    where $p$ is the smallest prime that divides the order $|K|$ of $K$. Prove
    that $L$ is normal in $K$. Hint: Use part (a).
    \begin{proof}
      This is \autoref{index_subgroup_normal_prime}.
    \end{proof}
    \item Describe all finite groups of order $p^2$, where $p$ is a prime, up
    to isomorphism. Prove your answer.

    \medskip

    We claim that there are two finite groups of order $p^2$: $\mathbb{Z}/p^2$
    and $\mathbb{Z}/p\oplus\mathbb{Z}/p$.
    \begin{proof}
      Since $|G|=p^2$, $|G|$ is abelian
      (\autoref{p^2_order_group_is_abelian}). Now, by the classification
      theorem for f.g.\ abelian groups, we can write
      \[
        G\cong\bigoplus_{i=1}^r\mathbb{Z}/p_i^{m_i}
      \]
      for some unique collection of primes $p_1,\ldots,p_r$ (not necessarily
      distinct) and positive integers $m_1,\ldots,m_r$. Given any such
      decomposition, we must have $p_1^{m_1}\cdots p_r^{m_r}=|G|=p^2$. Then the
      desired result follows.
    \end{proof}
    \item\label{May_2022_1(d)} Describe all finite groups of order $425=25\cdot 17$ up to
    isomorphism. Prove your answer.

    \medskip

    There are two:
    \[
      \mathbb{Z}/17\oplus\mathbb{Z}/5\oplus\mathbb{Z}/5
      \qquad\text{and}\qquad
      \mathbb{Z}/17\oplus\mathbb{Z}/25.
    \]
    \begin{proof}
      Let $G$ be a group of order $425$. By the third Sylow theorem,
      $n_{17}\mid 25$ and $n_{17}\equiv 1\bmod 17$, so
      $n_{17}\in\{1,5,25\}\cap\{1,18,35,\ldots\}=\{1\}$. Similarly, $n_5\mid
      17$ and $n_5\equiv 1\bmod 5$, so that
      $n_5\in\{1,17\}\cap\{1,6,11,16,21,\ldots\}=\{1\}$. Thus $G$ contains
      precisely one subgroup $P$ of order $17$ and one subgroup $Q$ of order
      $25$, and they are both normal by the second Sylow theorem. Moreover,
      $P\cap Q$ is a subgroup of both $P$ and $Q$, and $|P\cap Q|$ must divide
      both $17$ and $25$, which are coprime, so we must have $|P\cap Q|=1$,
      meaning $P\cap Q=\{e\}$. Finally, we have that $PQ$ is a subgroup of $G$
      (since $Q$ is normal) by the second isomorphism theorem, and $P$ and $Q$
      are both subgroups of $PQ$, so that $17$ and $25$ must both divide the
      order of $PQ$. Moreover, since $PQ\subseteq G$, we have
      $|PQ|\leq|G|=25\cdot 17$. It follows that $PQ=G$. Thus since $P,Q$ are
      normal, $P\cap Q=\{e\}$, and $PQ=G$, we have that $G=P\times Q$, by the
      product recognition theorem (\autoref{product_recognition}).

      Now, since $|P|=17$ is prime, $P$ is cyclic of order $17$. Moreover,
      since $|Q|=25=5^2$, we showed above that either
      $Q=\mathbb{Z}/5\oplus\mathbb{Z}/5$ or $Q=\mathbb{Z}/25$. Thus we are
      done.
    \end{proof}
  \end{enumerate}

  \item (May 2022 Q4) \begin{enumerate}[listparindent=\parindent,parsep=5pt,label={(\alph*)}]
    \item Let $G$ be a finite subgroup of the multiplicative group $K^*$ of a
    field $K$. Prove that $G$ is cyclic.
    \begin{proof}
      First of all, we claim that each Sylow subgroup of $G$ is cyclic. To that
      end, let $P$ be a $p$-Sylow subgroup of $G$ (where $p$ is some prime
      dividing the order of $G$), and let $a\in P$ have maximal order, say
      $|a|=m$, so $m=p^n$ for some positive integer $n$. Then
      $\{1,a,a^2,\ldots,a^{m-1}\}$ are $m$ distinct roots of the polynomial
      $f:=x^m-1\in K[x]$, which is of degree $m$, so they are the only roots of
      $f$. Now, let $b\in P$. Then since $P$ is a $p$-group, $b$ has order
      $p^k$ for some $k\in\mathbb{Z}_{\geq 0}$. Moreover, by assumption $k\leq
      n$, so that $b^m=b^{p^n}=(b^{p^k})^{p^{n-k}}=1$. Thus $b$ is a root of
      $f$, meaning $b\in\{1,a,a^2,\ldots,a^{m-1}\}$. Our choice of $b\in P$ was
      arbitrary, and we showed $b\in\langle a\rangle$, so $P=\langle a\rangle$,
      as desired.

      Now, since $G$ is a finite abelian group, it can be written as a product
      of its Sylow subgroups, each of which we've shown is cyclic. Thus, $G$ can be written as
      \[
        G=\bigoplus_{i=1}^r\mathbb{Z}/p_i^{m_i},
      \]
      where each of the $p_i$'s are distinct primes (since $G$ is abelian,
      given a fixed prime $p$ dividing $G$, each $p$-Sylow subgroup of $G$ is
      normal, so by the second Sylow theorem $n_p=1$), so by
      \autoref{product_of_distinct_primes_is_cyclic}, $G$ is cyclic, as
      desired.
    \end{proof}
    \item Let $k=\mathbb{Z}/p\mathbb{Z}$ be the finite field of order $p$, $p$
    a prime. Let $K/k$ be a finite field extension of degree $m$. Prove that
    the elements of $K$ are the roots of the polynomial $X^{p^m}-X$ over $k$.
    \begin{proof}
      \color{red}TODO.
    \end{proof}
    \item Prove that every irreducible polynomial $f(x)\in k[x]$ is separable.
    \begin{proof}
      \color{red}TODO.
    \end{proof}
  \end{enumerate}

  \item (August 2021 Q1) Let $G$ be a non-trivial finite group acting on a
  finite set $X$. We assume that for all $g\in G\setminus\{e\}$ there exists a
  unique $x\in X$ such that $g\cdot x=x$.
  \begin{enumerate}[listparindent=\parindent,parsep=5pt,label={(\alph*)}]
    \item Let $Y=\{x\in X\mid G_x\neq\{e\}\}$, where $G_x$ denotes the
    stabilizer of $x$. Show that $Y$ is stable under the action of $G$.
    \begin{proof}
      Let $y\in Y$ and $g\in G$. Then by
      \autoref{orb_same_size_iff_stab_same_size}, since $\Orb(g\cdot
      y)=\Orb(y)$ (by definition), it follows that $|\Stab(g\cdot
      y)|=|\Stab(y)|\geq 2$, so that $g\cdot y$ has a nontrivial stabilizer, as
      desired.
    \end{proof}
    \item Let $y_1,y_2,\ldots,y_n$ be a set of orbit representatives of $Y/G$ (with $|Y/G|=n$), and let $m_i=|G_{y_i}|$. Show that
    \[
      1-\frac{1}{|G|}=\sum_{i=1}^{n}\left(1-\frac{1}{m_i}\right).
    \]
    \begin{proof}
      Note that $|G|/m_i=|\Orb(y_i)|$ by the Orbit/Stabilizer theorem. Thus
      \[
        |G|\sum_{i=1}^{n}\left(1-\frac{1}{m_i}\right)
        =n|G|-\sum_{i=1}^{n}\frac{|G|}{m_i}
        =n|G|-\sum_{i=1}^{n}|\Orb(y_i)|
        =n|G|-|Y|.
      \]
      Thus, it suffices to show that
      \[
        |G|-1=n|G|-|Y|.
      \]
      This follows by Burnside's Lemma (\autoref{burnside}), as
      \begin{align*}
        n|G|-|Y|
        &=|Y/G||G|-|Y| \\
        &=\sum_{g\in G}|Y^g|-|Y|  & (Y^g:=\{y\in Y\mid g\cdot y=y\}) \\
        &=|Y^e|+\sum_{g\in G\setminus\{e\}}|Y^g|-|Y| \\
        &\overset{(\ast)}=|Y|+|G\setminus\{e\}|-|Y| \\
        &=|G|-1,
      \end{align*}
      where $(\ast)$ denotes where we used the assumption that $|Y^g|=1$ for
      all $g\in G\setminus\{e\}$.
    \end{proof}
    \item\label{Aug_2021_Q1(c)} Show that $X$ has (at least) a fixed point under the action of $G$.
    \begin{proof}
      By part (ii), we have
      \[
        |G|-1=n|G|-|Y|
      \]
      which yields
      \begin{equation}\label{eq1}
        |Y|=(n-1)|G|+1.
      \end{equation}
      We claim that $Y$ has at least $n-1$ orbits of size $|G|$. Assuming this
      were true, since $Y$ has $n$ orbits and $|Y|=(n-1)|G|+1$, it would follow
      that the remaining orbit of $Y$ must have order $1$, so that the action
      of $G$ fixes a point of $Y$, and therefore a point of $X$, as desired.

      Now, to see the claim, note that the order of each orbit of $Y$ divides
      $|G|$, so if there were two orbits of size $<|G|$, the sum of their
      orders would be at most $|G|$, which would yield
      \[
        |Y|\leq|G|+(n-2)|G|=(n-1)|G|<(n-1)|G|+1,
      \]
      a contradiction of \autoref{eq1}, as desired.
    \end{proof}
  \end{enumerate}

  \item (January 2021 Q1) Let $G$ be a group of order $2057$.
  \begin{enumerate}[listparindent=\parindent,parsep=5pt,label={(\alph*)}]
    \item Snow that $G\simeq P\times Q$, where $P$ is a group of order $17$ and
    $Q$ is a group of order $121$. Determine all groups of order $2057$ up to
    isomorphism.

    \medskip

    There are two.
    \[
      \mathbb{Z}/17\oplus\mathbb{Z}/121
      \qquad\text{and}\qquad
      \mathbb{Z}/17\oplus\mathbb{Z}/11\oplus\mathbb{Z}/11.
    \]
    \begin{proof}
      Observe that $2057=17\cdot 121=17\cdot 11^2$, and use the exact same
      argument given in \hyperref[May_2022_1(d)]{May 2022, Q1(d)}.
    \end{proof}
    \item Show that $\Aut(G)\simeq\Aut(P)\times\Aut(Q)$.
    \begin{proof}
      This is \autoref{aut_of_prod_of_coprimes}.
    \end{proof}
    \item Show that if $Q$ is cyclic, then so is $\Aut(Q)$. What is the order
    of $\Aut(Q)$ in this case?
    \begin{proof}
      This is proven in class --- if $G\cong\langle a\mid a^n\rangle$, then
      there is an isomorphism of monoids
      \[
        \mathbb{Z}/n\mathbb{Z}\xrightarrow{\sim}\End(G)
        \qquad\text{given by}\qquad
        [k]\mapsto(a^m\mapsto a^{mk})
      \]
      (this is easily proven via the universal property of free groups). Thus
      there is an isomorphism of groups
      \[
        (\mathbb{Z}/n\mathbb{Z})^\times\cong\Aut(G).
      \]
      We know that $|(\mathbb{Z}/n\mathbb{Z})^\times|=\phi(n)$, where $\phi(n)$
      is the number of positive integers less than or equal to $n$ that are
      coprime to $n$.

      It is straightforward to see that $\phi(p^k)=p^k-p^{k-1}=p^{k-1}(p-1)$ if
      $p$ is prime: given $1\leq m<p^k$, the only way to have $\gcd(p^k,m)>1$
      is if $m$ is a multiple of $p$, that is,
      $m\in\{p,2p,3p,\ldots,p^{k-1}p=p^k\}$, and there are $p^{k-1}$ such
      multiples not greater than $p^k$. Therefore, the other $p^k-p^{k-1}$
      numbers are all relatively prime to $p^k$. Thus if
      $Q\cong\mathbb{Z}/121=\mathbb{Z}/11^2$, we have that
      $|\Aut(Q)|=|(\mathbb{Z}/11^2)^\times|=11^2-11=110$.

      Now, it remains to show that $(\mathbb{Z}/11^2)^\times$ is cyclic. There
      is an easy way and a hard way to do this. The easy way is to observe that
      $|(\mathbb{Z}/11^2)^\times|=110=2\cdot 5\cdot 11$, so by the
      classification theorem for finite abelian groups, we must have
      \[
        (\mathbb{Z}/11^2)^\times
        \cong\mathbb{Z}/2\oplus\mathbb{Z}/5\oplus\mathbb{Z}/11,
      \]
      and $2$, $5$, and $11$ are distinct primes, so $(\mathbb{Z}/11^2)^\times$
      is cyclic.

      The hard way is to find an element of $(\mathbb{Z}/11^2)^\times$ of order
      $110$. By \autoref{units_of_Z/p^k_are_cyclic} it suffices to first find a
      generator $n$ of $(\mathbb{Z}/11)^\times$, in which case there is
      guaranteed to exist some $m\geq 0$ such that $n+11m$ is a generator of
      $(\mathbb{Z}/11^2)^\times$. This requires one to guess and check via some
      arduous arithmetic. There are some tricks one can do to make it
      manageable, however.

      First of all, we'll take $n=2$, since $\gcd(2,11)=1$, so that $2$
      generates $(\mathbb{Z}/11)^\times$. Now the aforementioned lemma
      guarantees the existince of some $m\geq 0$ such that $2+11m$ generates
      $(\mathbb{Z}/11^2)^\times$. We'll start by checking $m=0$. So we need to
      check that $2$ has multiplicative order $110$ in $\mathbb{Z}/11^2$. Since
      $110=2\cdot 5\cdot 11$, it suffices to check that $2^k\not\equiv 1\bmod
      11^2$ for $k=2\cdot 5=10$, $k=2\cdot 11=22$, or $k=5\cdot 11=55$. This
      requires a string of computations by hand:
      \begin{itemize}
        \item $2^{10}=1024\equiv 56\bmod 121$.
        \item $2^{22}=(2^{10})^2\cdot 2^2\equiv(56)^2\cdot 4\bmod 121$.
        \item $(56)^2=3136\equiv 111\bmod 121$.
        \item $2^{22}\equiv 111\cdot 4=444\equiv 81\bmod 121$.
        \item $2^{55}=(2^{10})^5\cdot 2^5\equiv(56)^5\cdot 32\equiv(111)^2\cdot 56\cdot 32\bmod 121$
        \item $(111)^2=12321\equiv 100\bmod 121$.
        \item $56\cdot 32=1792\equiv 98\bmod 121$.
        \item $2^{55}\equiv 100\cdot 98=9800\equiv 120\bmod 121$.
      \end{itemize}
      Thus, we will have shown that $2$, as an element of
      $(\mathbb{Z}/11^2)^\times$, has order $110$, so that
      $(\mathbb{Z}/11^2)^\times$ is cyclic, as desired.
    \end{proof}
    \item If $Q$ is not cyclic, find an isomorphic description of $\Aut(Q)$ and
    compute its order.
    \begin{proof}
      If $Q$ is not cyclic, then
      $Q\cong\mathbb{Z}/11\oplus\mathbb{Z}/11=\mathbb{F}_{11}^2$, so that
      $\Aut(Q)=\GL_2(\mathbb{F}_{11})$. The group $\GL_n(\mathbb{F}_p)$ has
      order $(p^n-1)(p^n-p)(p^n-p^2)\cdots(p^n-p^{n-1})$. (The first row $u_1$
      of the matrix can be anything but the $0$-vector, so there are $p^n-1$
      possibilites for the first row. The second row can be anything but a
      multiple of the first row, giving $p^n-p$ possibilites. For any choice
      $u_1,u_2$ of the first rows, the third row can be anything but a linear
      combination of $u_1$ and $u_2$. The number of linear combinations
      $a_1u_1+a_2u_2$ is just the number of choices for the pair $(a_1,a_2)$,
      and there are $p^2$ of these. It follows that there are $p^n-p^2$ for the
      third row. And so on.) Thus $\Aut(Q)$ has order
      $(11^2-1)(11^2-11)=120\cdot 110=13200$.
    \end{proof}
  \end{enumerate}

  \item (August 2020 Q1) \begin{enumerate}[listparindent=\parindent,parsep=5pt,label={(\alph*)}]
    \item A finite group $G$ is called \emph{cool} if $G$ has precisely four
    Sylow subgroups (over all primes $p$). The order $|G|$ of a cool group is
    called a \emph{cool} number. For example, $S_3$ is a cool group and $6$ is
    a cool number. Describe the set of all cool numbers. Hint: Use prime
    factorization in your description.

    \medskip

    We claim there are two types of cool numbers:
    \begin{itemize}
      \item \textbf{Type I.} Numbers of the form $p^nq^mr^ks^\ell$, where
      $p,q,r,s$ are distinct prime numbers, and $n,m,k,\ell$ are positive
      integers. I.e., numbers with exactly four distinct prime factors.
      \item \textbf{Type II.} Numbers of the form $2^n 3^m$, where $n$ and $m$ are any positive integers.
    \end{itemize}
    \begin{proof}
      To start, we will show any Type I or II number is cool. First, let
      $p,q,r,s$ be distinct prime numbers, and $n,m,k,\ell$ be positive
      integers, and consider the group
      \[
        G=\mathbb{Z}/p^n\oplus\mathbb{Z}/q^m\oplus\mathbb{Z}/r^k\oplus\mathbb{Z}/s^\ell.
      \]
      Because $G$ is abelian, every subgroup of $G$ is normal. Thus we have
      $n_p=n_q=n_r=n_s=1$ by the second Sylow theorem, so that $G$ has 4 Sylow
      subgroups as desired.

      Now, let $n$ and $m$ be positive integers and consider the group
      \[
        G=S_3\times \mathbb{Z}/2^{n-1}\times\mathbb{Z}/3^{m-1}.
      \]
      Clearly $|G|=6\cdot 2^{n-1}\cdot 3^{m-1}=2^n3^m$. Now we claim that
      $n_2(G)=3$ and $n_3(G)=1$. To see this, note first that $n_3(S_3)=1$ and
      $n_2(S_3)=3$. Then $n_3(S_3\times\mathbb{Z}/2^{n-1})=1$ by
      \autoref{p-Sylow_subgroups_of_product}, since $3$ does not divide
      $|\mathbb{Z}/2^{n-1}|$, and
      $n_2(S_3\times\mathbb{Z}/2^{n-1})=n_2(S_3)=3$, by
      \autoref{p-Sylow_subgroups_of_GxZpn}. A simular argument yields that
      $n_3(G)=1$ and $n_2(G)=3$, as desired.

      Now, let $G$ be a group. Then we claim that in order for $G$ to be cool,
      its order must be Type I or II as defined above. If $|G|$ has more than
      four distinct prime factors, then $G$ has more than four Sylow subgroups
      by Sylow 1, so $G$ isn't cool. We showed above that any number with
      precisely four distinct prime factors is cool. Clearly the trivial group
      is not cool. Thus, it suffices to consider the cases that $|G|$ has one,
      two, or three prime factors. In what follows, let $p$, $q$, and $r$ be
      distinct primes, and let $n$, $m$, and $k$ be positive integers.

      \textbf{Case 1.} $|G|=p^n$. By the third Sylow theorem, we have $n_p\mid
      1$, which implies $n_p=1\neq 4$, so no $p$-group is cool.

      \textbf{Case 2.} $|G|=p^nq^m$. In order for $G$ to be cool, we must have
      $n_p+n_q=4$, so suppose this holds. Then we claim $\{p,q\}=\{2,3\}$. If
      $n_p=n_q=2$, then by Sylow $3$ we'd have $n_p=2\equiv 1\bmod p$, i.e.,
      $1\equiv 0\bmod p$, but $1$ is not a multiple of any prime, so we can't
      have $n_p=n_q=2$.

      Now, suppose $\{n_p,n_q\}=\{1,3\}$, say WLOG $n_p=3$ and $n_q=1$. Then by
      Sylow 3, we have $n_p=3\equiv 1\bmod p$, i.e., $2\equiv 0\bmod p$, which
      is only possible if $p=2$. We'd also have $n_p=3\mid q^m$, which is only
      possible if $q=3$. Hence $|G|=2^n3^m$, so $|G|$ is Type II, as desired.

      \textbf{Case 3.} $|G|=p^nq^mr^\ell$. Again, if $G$ is cool, then we can
      assume WLOG that $n_p=2$ and $n_q=n_r=1$. Then by Sylow 3, $n_p=2\equiv
      1\bmod p$, i.e., $1\equiv 0\bmod p$, an impossibility since $p\neq
      1$. Thus if $G$ has $3$ prime factors then it is lame.
    \end{proof}
    \item For each cool number $n$ that you found in part (a), determine
    whether every group of order $n$ is nilpotent.

    \medskip

    Every Type I cool group is nilpotent, and no Type II cool group is
    nilpotent.

    \begin{proof}
      Now, let $G$ be a Type I cool group, so that $|G|$ has four distinct
      prime factors and $G$ has four Sylow subgroups. Then it follows by
      \autoref{Sylow_subgroups_are_normal_implies_prod_of_Sylows} that $G$ is a
      product of its Sylow subgroups. Thus $G$ is nilpotent by
      \autoref{finite_nilpotent_groups_theorem}, as desired.

      Now, we claim that no Type II cool group is nilpotent. Indeed, we showed
      above that any Type II cool group satisfies $n_2=3$, which means $G$
      cannot be nilpotent by \autoref{finite_nilpotent_groups_theorem}, as any
      finite nilpotent group has exactly one $p$-Sylow subgroup for each prime
      $p$ dividing its order.
    \end{proof}
    \item For each cool number $n$ that you found in part (a), determine
    whether every cool group of order $n$ is solvable.

    Every cool group is solvable.

    \begin{proof}
      Recall every nilpotent group is solvable, so every Type I cool group is
      solvable. By Burnside's Theorem (proven in Dummit \& Foote Section 19.2),
      every group of order $p^aq^b$ for $p$ and $q$ distinct primes and $a$ and
      $b$ positive integers is solvable, so Type II cool groups are
      solvable.

      An argument that does not require Burnside's theorem: Let $G$ be a Type
      II cool group, so that $|G|=2^n3^m$ for some positive integers $n$ and
      $m$, $n_2(G)=3$, and $n_3(G)=1$. Let $P$ be the unique $3$-Sylow subgroup
      of $G$, which is normal by Sylow 2. Note that $G/P$ has order $2^n$,
      which is a prime power, so $G/P$ is solvable. Moreover $P$ is solvable,
      because $P$ also has prime power order $3^m$. Thus since $G/P$ and $P$
      are solvable, $G$ must be solvable as well.
    \end{proof}
  \end{enumerate}

  \item (August 2020 Q2) Suppose a finite group $G$ acts on a set $A$ so that
  for every nontrivial $g\in G$ there exists a unique fixed point (i.e., there
  is exactly one $a\in A$, depending on $g$, such that $g(a)=a$). Prove that
  this fixed point is the same for all $g\in G$.
  \begin{proof}
    This is \hyperref[Aug_2021_Q1(c)]{August 2021, Q1(c)}.
  \end{proof}

  \item (May 2022 Q2) Make $\mathbb{C}^3$ into a $\mathbb{C}[x]$-module by $f(x)v=f(A)v$, where $v\in\mathbb{C}^3$ and 
  \[
    A=\begin{pmatrix}
      5 & 3 & 0 \\
      0 & 5 & 0 \\
      0 & 3 & 3
    \end{pmatrix}.
  \]
  Find polynomials $p_i(x)$ and exponents $e_i$ such that
  $\mathbb{C}^3\cong\bigoplus_i\mathbb{C}[x]/(p_i^{e_i})$ as
  $\mathbb{C}[x]$-modules. Justify your answer.
  \begin{proof}
    We claim that
    \[
      \mathbb{C}^3\cong\mathbb{C}[x]/((x-5)^2)\oplus\mathbb{C}[x]/(x-3).
    \]
    Recall that with its $\mathbb{C}[x]$-module structure given by $A$,
    $\mathbb{C}^3$ is isomorphic to $\bigoplus_{j=1}^n\mathbb{C}[x]/(f_j)$, where
    $f_1,f_2,\ldots,f_k$ are invariant factors of the matrix $A$ satisfying:
    \begin{itemize}
      \item $f_1\mid f_2\mid\cdots\mid f_k$,
      \item $f_1f_2\cdots f_k$ is the characteristic polynomial $c_A$ of $A$, and
      \item $f_k=m_A$ is the minimal polynomial of $A$.
    \end{itemize}
    We have that the characteristic polynomial of $A$ is given by
    \[
      \det(xI-A)
      =\det \begin{pmatrix}
        x-5 & -3 & 0 \\
        0 & x-5 & 0 \\
        0 & -3 & x-3
      \end{pmatrix}
      =(x-5)^2(x-3).
    \]
    The above conditions give that the minimal polynomial of $A$ is either
    $(x-5)^2(x-3)$ or $(x-5)(x-3)$ (since the minimal polynomial $m_A$ must
    divide $c_A=(x-5)(x-3)^2$, and the other invariant factors, which multiply to
    give $c_A/m_A$, must each divide $m_A$). One can directly check that
    $m_A(x)\neq(x-5)(x-3)$, as
    \[
      (A-5I)(A-3I)=\begin{pmatrix}
        0 & 3 & 0 \\
        0 & 0 & 0 \\
        0 & 3 & -2
      \end{pmatrix}\begin{pmatrix}
        2 & 3 & 0 \\
        0 & 2 & 0 \\
        0 & 3 & 0
      \end{pmatrix}
      =\begin{pmatrix}
        0 & 6 & 0 \\
        0 & 0 & 0 \\
        0 & 0 & 0
      \end{pmatrix}
      \neq 0,
    \]
    so the minimal polynomial must be $m_A=(x-5)^2(x-3)$. Thus $m_A=c_A$ is the
    only invariant factor of $A$, so that with its $\mathbb{C}[x]$-module
    structure given by $A$, we have
    \[
      \mathbb{C}^3\cong\mathbb{C}[x]/((x-5)^2(x-3)).
    \]
    Now, note that $x-5$ and $x-3$ are non-associate primes, so that $(x-5)^2$
    and $(x-3)$ are coprime. Thus by the Chinese remainder theorem, we further
    have that
    \[
      \mathbb{C}^3\cong\mathbb{C}[x]/((x-5)^2)\oplus\mathbb{C}[x]/(x-3),
    \]
    as desired.
  \end{proof}

  \item (May 2022 Q3) Completely factor the following polynomials over the
  given fields (or prove they are irreducible).
  \begin{enumerate}[listparindent=\parindent,parsep=5pt,label={(\alph*)}]
    \item $x^3+x+2\in\mathbb{Z}_3[x]$.

    \[
      f(x):=x^3+x+2\equiv(x^2+2x+2)(x-2)\bmod 3.
    \]

    \begin{proof}
      One can check by hand that $f(x)=x^3+x+2$ has a root, namely $f(2)=0$, so
      $x-2$ must divide $f$. Then doing polynomial long division yields
      \[
        (x^2+2x+2)(x-2)=x^3+x+2\bmod 3.
      \]
      Then one can check $x^2+2x+2$ has no roots in $\mathbb{Z}_3$, and it is
      quadratic, so it is irreducible. Thus the above is the irreducible
      factorization of $f$.
    \end{proof}
    \item $x^4+x^3+x+3\in\mathbb{Z}_5[x]$.

    \medskip

    The polynomial is irreducible.

    \begin{proof}
      There might be an easier proof, but this is all I can think of.

      Let $f(x)=x^4+x^3+x+3$. One can directly check that $f(j)\not\equiv
      0\bmod 5$ for $j=0,1,2,3$, so if $f$ factors over $\mathbb{F}_5$, it must
      do so as a product of quadratics. Suppose it did, so there exists $a,b,c,d\in\mathbb{F}_5$ such that
      \begin{align*}
        x^4+x^3+x+3
        &=(x^2+ax+b)(x^2+cx+d) \\
        &=x^4+(a+c)x^3+(b+d+ac)x^2+(ad+bc)x+bd,
      \end{align*}
      so that
      \[
        a+c=1,\qquad b+d+ac=0,\qquad ad+bc=1,\qquad\text{and}\qquad bd=3.
      \]
      Substituting $c=1-a$ and $d=3/b$ in the middle two equations yields the
      system
      \[
        0=b+\frac{3}{b}+a(1-a)
        \qquad\text{and}\qquad
        1=\frac{3a}{b}+b(1-a).
      \]
      Multiplying the equations by $b$ yields
      \begin{equation}\label{eq3}
        0=b^2+3+ab-a^2b
        \qquad\text{and}\qquad
        0=3a+b^2-ab^2-b.
      \end{equation}
      Now, it suffices to show that there does not exist any
      $a,b\in\mathbb{F}_5$ satisfying both of these equations. To show this, we
      split into cases:

      \textbf{Case 1.} If $a=0$, then the equations become $0=b^2+3$ and
      $0=b^2-b$. Assuming $b^2-b=0$, the first equation becomes $b+3=0$, so
      $b=-3\equiv 2$. But then we'd have $b^2-b=2^2-2=2\not\equiv 0\bmod 5$, a
      contradiction of the fact that $b^2-b=0$ to begin with.

      \textbf{Case 2.} If $a=1$, then the equations become $0=b^2+3+b-b=b^2+3$
      and $0=3+b^2-b^2-b=3-b$. The second equation yields $b=3$, but then the
      first equation is unsatisfied, as $b^2+3=9+3=12\not\equiv 0$. Thus it
      cannot hold that $a=1$.

      \textbf{Case 3.} If $a=2$, then the equations become $0=b^2+3+2b-4b\equiv
      b^2+3b+3$ and $0=6+b^2-2b^2-b\equiv -b^2-b+1$. The second equation yields
      $b^2=1-b$, so the first equation the becomes $0=1-b+3b+3=2b+4$, so that
      $b=-2\equiv 3$. But then the second equation does not hold, as we'd have
      $-b^2-b+1=-9-3+1=-11\not\equiv 0$.

      \textbf{Case 4.} If $a=3$, then the equations become $0=b^2+3+3b-9b\equiv
      b^2-b+3$ and $0=9+b^2-3b^2-b\equiv 3b^2-b-1$. The first equation gives
      $b^2=b-3$, which causes the second equation to become $0=3(b-3)-b-1\equiv
      2b$, so that we must have $b=0$.

      \textbf{Case 5.} Finally if $a=4$, then the equations become
      $0=b^2+3+4b-16b\equiv b^2-2b-2$ and $0=12+b^2-4b^2-b\equiv 2b^2-b+2$. The
      first equation yields $b^2=2b+2$, so that the second equation becomes
      $0=2(2b+2)-b+2\equiv 3b+1$, so that $b=1/3\equiv 2$. But then the first
      equation no longer holds, as we'd have $b^2-2b-2=4-4-2=-2\not\equiv 0$.

      Thus there are no $a,b\in\mathbb{F}_5$ satisfying \autoref{eq3}, so it
      cannot have been true that $f$ factored in the first place.
    \end{proof}
    \item\label{May_2022_Q3c} $x^4+x^3+x^2+6x+1\in\mathbb{Q}[x]$.

    \medskip

    The polynomial is irreducible.

    \begin{proof}
      Since $\mathbb{Z}$ is a UFD (it is in fact a Euclidean domain) with
      $\Frac\mathbb{Z}=\mathbb{Q}$, in order to show $f$ is irreducible in
      $\mathbb{Q}[x]$ it suffices to show it is irreducible in $\mathbb{Z}[x]$,
      as $f$ has coefficients in $\mathbb{Z}$. To that end, one can check via a
      straightforward computation that
      \[
        f(x+1)=x^4+5x^3+10x^2+15x+10.
      \]
      It follows by Eisenstein's with $p=5$ that $f(x+1)$ is irreducible over
      $\mathbb{Z}$, and thus over $\mathbb{Q}$. Since $f(x)\mapsto f(x+1)$ is a
      ring automorphism of $\mathbb{Q}[x]$ (with inverse $f(x)\mapsto f(x-1)$),
      it follows that $f$ is irreducible as well, as desired.
    \end{proof}
  \end{enumerate}

  \item (August 2021 Q2)\begin{enumerate}[listparindent=\parindent,parsep=5pt,label={(\alph*)}]
    \item Show that $x^6+69x^5-511x+363$ is irreducible over the integers.
    \begin{proof}
      This is a hard one. Write $f$ for the polynomial in question. Modulo $3$,
      it's easy to factor, as $f(x)\equiv x^6-x\bmod 3$. Since both $0$ and
      $1\equiv -2$ are roots of $f$ in $\mathbb{F}_3$, it follows that
      $f(x)=x(x+2)g$, where $g$ is a degree $4$ polynomial. Performing
      polynomial long division yields that $g=x^4+x^3+x^2+x+1$. One can check
      that $g$ has no roots in $\mathbb{F}_3$, so if $g$ were to factor it
      would do so as a product of quadratics, say
      \begin{align*}
        g
        &=x^4+x^3+x^2+x+1 \\
        &=(x^2+ax+b)(x^2+cx+d) \\
        &=x^4+(a+c)x^3+(ac+b+d)x^2+(ad+bc)x+bd,
      \end{align*}
      so that
      \[
        a+c=1,\qquad ac+b+d=1,\qquad ad+bc=1,\qquad\text{and}\qquad bd=1.
      \]
      Substituting $d=1/b$ and $c=1-a$ in the middle two equations yields
      \[
        a(1-a)+b+\frac{1}{b}=1,\qquad\text{and}\qquad \frac{a}{b}+b(1-a)=1.
      \]
      Multiplying both equations by $b$ yields
      \begin{equation}\label{eq4}
        ab-a^2b+b^2+1-b=0 \qquad\text{and}\qquad a+b^2-ab^2-b=0.
      \end{equation}
      Thus in order to show $g$ is irreducible, it suffices to show that there
      does not exist $a,b\in\mathbb{F}_3$ which satisfy \autoref{eq4}. To see
      this, suppose for the sake of a contradiction that there existed
      $a,b\in\mathbb{F}_3$ such that \autoref{eq4} holds.

      \textbf{Case 1.} If $a=0$, then the equations become $b^2+1-b=0$ and
      $b^2-b=0$. The second equation yields $b^2=b$, so the first equation
      becomes $b+1-b=0$, i.e., $1=0$, a contradiction.

      \textbf{Case 2.} If $a=1$, then the equations becomes
      $0=b-b+b^2+1-b=b^2+1-b$ and $0=1+b^2-b^2-b=1-b$. The second equation
      yields $b=1$, so then the first equation becomes $0=b^2+1-b=1+1-1=1$, a
      contradiction.

      \textbf{Case 3.} If $a=2$, then the equations become
      $0=2b-4b^2+b^2+1-b\equiv b+1\pmod 3$ and $0=2+b^2-2b^2b-b\equiv
      2-b^2-b\pmod 3$. The first equation yields that $b=-1$, so the second
      equation becomes $0=2-b^2-b=2-1+1=2$, and $2\not\equiv 0\pmod 3$, so we
      reach a contradiction.

      Thus $f$ has an irreducible factorization over $\mathbb{F}_3$ given by
      \[
        f(x)\equiv x(x+2)(x^4+x^3+x^2+x+1)\bmod 3.
      \]
      Thus, if $f$ factors over $\mathbb{Z}$, it must factor as a product of
      irreducible polynomials of degree $1,1,4$, or $1,5$, or $2,4$ (since any
      factorization of $f$ over $\mathbb{Z}$ descends to a factorization over
      $\mathbb{F}_3$).

      Now, consider $f$ over $\mathbb{F}_5$. Taken mod $5$, $f$ is given by
      $x^6-x^5-x+3$. One can check directly that $f$ does not have any roots in
      $\mathbb{F}_5$, so it has no linear factors. Thus $f$ has no linear
      factors over $\mathbb{Z}$. Now, suppose for the sake of a contradiction
      that $f$ factors over $\mathbb{Z}$, so by what we've shown it factors as
      an irreducible quadratic polynomial $p$ times an irreducible quartic
      polynomial $q$. Moreover, by what we have shown above, we must futher
      have
      \[
        p\equiv x(x+2)=x^2+2x\bmod 3.
      \]
      A similar argument to one given above for $\mathbb{F}_3$ yields that $f$
      factors irreducibly as
      \[
        f\equiv x(x+2)(x^4+x^3+9x^2+4x+3)\bmod 11
      \]
      over $\mathbb{F}_{11}$. Thus it follows that
      \[
        p\equiv x^2+2x\bmod 11
        \qquad\text{and}\qquad
        q\equiv x^4+x^3+9x^2+4x+3\bmod 11
      \]
      Now, write $a$ for the constant term of $p$ and $b$ for the constant term
      of $q$, so that $ab=363=3\cdot 11\cdot 11$. By what we've shown above, we
      know that $3$ and $11$ both divide $a$, and $b\equiv 3\bmod 11$. Thus
      $a\in\{\pm 33,\pm 363\}$ and $b\in\{\pm 1,\pm 11\}$. Yet none of $1$,
      $-1$, $11$, or $-11$ are equivalent to $3$ mod $11$. Hene we reach a
      contradiction, $f$ could not have factored in the first place.
    \end{proof}
    \item Show that $x^4+5x+1$ is irreducible over the rationals.
    \begin{proof}
      By the rational root test, any rational root of $f(x):=x^4+5x+1$ must
      divide $1$, and it is straightforward to check that $1$ and $-1$ are not
      roots of $f$. Hence, if $f$ factored, it would do so as a product of
      quadratics, say
      \begin{align*}
        f(x)
        &=x^4+5x+1 \\
        &=(x^2+ax+b)(x^2+cx+d) \\
        &=x^4+(a+c)x^3+(ac+b+d)x^2+(ad+bc)x+bd,
      \end{align*}
      for some $a,b,c,d\in\mathbb{Q}$, so that
      \[
        0=a+c,\qquad 0=ac+b+d,\qquad 5=ad+bc,\qquad\text{and}\qquad 1=bd.
      \]
      Substituting $c=-a$ and $d=1/b$ in the middle two equations further
      yields
      \[
        0=-a^2+b+\frac{1}{b}\qquad\text{and}\qquad 5=\frac{a}{b}-ab.
      \]
      Multiplying both equations by $b$ yields
      \begin{equation}\label{eq5}
        0=-a^2b+b^2+1 \qquad\text{and}\qquad 0=a-ab^2-5b.
      \end{equation}
      Hence it suffices to show that there does not exist $a,b\in\mathbb{Z}$
      which satisfy \autoref{eq5}. Supposing there did exist such a pair, note
      that the second equation yields
      \[
        0=a(1-b^2)-5b\implies a=\frac{5b}{1-b^2},
      \]
      so that the first equation becomes
      \[
        0=-\left(\frac{5b}{1-b^2}\right)^2b+b^2+1=-\frac{25b^3}{b^4-2b^2+1}+b^2+1.
      \]
      Multiplying by $b^4-2b^2+1$ and simplifying yields
      \[
        0=b^{6}-b^{4}-25b^{3}-b^{2}+1.
      \]
      By the rational root test, any rational solution $b$ to this equation
      must be an integer dividing $1$, yet one can directly check that neither
      $1$ nor $-1$ satisfy the above equation. Thus there does not exist any
      $a,b\in\mathbb{Q}$ satisfying \autoref{eq5}, meaning $f$ is irreducible
      over $\mathbb{Q}$, as desired.
    \end{proof}
    \item Show that $x^4+x^3+x^2+6x+1$ is irreducible over the rationals.
    \begin{proof}
      This is \hyperref[May_2022_Q3c]{May 2022, Q3(c)}.
    \end{proof}

    \item Calculate the number of distinct, irreducible polynomials over
    $\mathbb{Z}_5$ that have the form
    \[
      f(x)=x^2+ax+b,\qquad\text{or}\qquad g(x)=x^3+\alpha x^2+\beta x+\gamma
      \qquad a,b,\alpha,\beta,\gamma\in\mathbb{Z}_5.
    \]

    We will prove more generally that in a finite field of order $n$, there
    are:
    \begin{itemize}
      \item $\frac{n^2-n}{2}$ monic, irreducible, quadratic polynomials, and
      \item $\frac{n^3-n}{3}$ monic, irreducible, cubic polynomials.
    \end{itemize}

    \begin{proof}
      Let $F$ be a finite field of order $n$. First of all, note there are
      $n^2$ monic quadratic polynomials, and $n^3$ monic cubic polynomials.

      Now, we'd like to count the number of reducible, monic, and quadratic
      polynomials in $F[x]$. Given such a polynomial $f$, in order for it to
      factor, it must factor as a product of monic linear polynomials, say as
      $f(x)=(x-a)(x-b)$ for some $a,b\in F$. There are $n\choose 2$ such
      polynomials with $a$ and $b$ distinct, and $n$ such polynomials with
      $a=b$, giving a total of
      \[
        n^2-\left({n\choose 2}+n\right)=\frac{n^2-n}{2}
      \]
      irreducible, monic, and quadratic polynomials in $F[x]$.

      Now, we wish to count the number of reducible, monic, and cubic
      polynomials in $F[x]$. Given $f\in F[x]$ monic and cubic, for it to
      factor it must have a linear factor. It follows there are two distinct
      types
      \begin{enumerate}[listparindent=\parindent,parsep=5pt,label={(\alph*)}]
        \item $f(x)=(x-a)(x-b)(x-c)$ with $a,b,c\in F$, and
        \item $f(x)=(x-a)g(x)$, where $a\in F$ and $g$ is an irreducible,
        monic, quadratic polynomial.
      \end{enumerate}
      There are three subcases for a polynomial of type (a), based on how many
      distinct roots it has. There are $n$ ways to choose a type (a) polynomial
      when $a=b=c$. There are ${n\choose 2}\cdot 2$ ways to choose a type (a)
      polynomial with two distinct roots (first pick the two roots from $F$,
      then choose which root to double). Finally, there are $n\choose 3$ ways
      to choose a type (a) polynomial with $3$ distinct roots. Hence, there are
      \[
        n+2\cdot{n\choose 2}+{n\choose 3}
      \]
      polynomials of type (a). To count the number of type (b) polynomials,
      observe that there are $n$ ways to choose a root $a\in F$, and we know
      there are $\frac{n(n-1)}{2}$ irreducible quadratic and monic polynomials
      over $F$, so there are
      \[
        n\cdot\frac{n(n-1)}{2}=\frac{n^2(n-1)}{2}
      \]
      polynomials of type (b). Thus there are
      \begin{align*}
        n^3-\left(n+2{n\choose 2}+{n\choose 3}\right)&-\frac{n^2(n-1)}{2} \\
        &=n^3-n-n(n-1)-\frac{n(n-1)(n-2)}{6}-\frac{n^3-n^2}{2} \\
        &=\frac{n^3-n}{3}
      \end{align*}
      irreducible, monic, and cubic polynomials over $F$, as desired.
    \end{proof}
  \end{enumerate}

  \item (August 2021 Q3) Find [all] possible Jordan canonical forms
  of an $8\times 8$ matrix $M$ over the field $\mathbb{F}_5$ with five elements
  if it is known that the characteristic polynomial of $M$ is $(x^2+1)^4$ and
  the minimal polynomial of $M$ is $(x^2+1)^2(x+2)$.
  \begin{solution}
    First of all, note that $x^2+1\equiv(x+2)(x+3)\bmod 5$, so that the
    characteristic polynomial of $M$ is given by $c(x)=(x+2)^4(x+3)^4$ and the
    minimal polynomial is given by $m(x)=(x+2)^3(x+3)^2$. Now, we'd like to
    find the invariant factors $f_1,f_2,\ldots,f_k\in \mathbb{F}_5[x]$ of
    $M$. Recall the following facts about the invariant factors:
    \begin{itemize}
      \item $f_j$ is nonconstant and monic for $j=1,\ldots,k$.
      \item $f_j\mid f_{j+1}$ for $1\leq j<k$.
      \item $f_k=m$.
      \item $f_1\cdots f_k=c$.
    \end{itemize}
    Putting these facts together, we have that the following is an exhaustive
    list of possibilities for the invariant factors of $M$:
    \begin{enumerate}[listparindent=\parindent,parsep=5pt,label={(\arabic*)}]
      \item $f_1=(x+2)(x+3)^2$, $f_2=m$.
      \item $f_1=(x+3)$, $f_2=(x+2)(x+3)$, $f_3=m$.
    \end{enumerate}
    Now, in order to find the Jordan canonical form of $M$, we need to find the
    elementary divisors of $M$. These are the prime powers dividing the
    invariant factors (counted individually, for each invariant factor). Thus,
    the list of elementary divisors of $M$ are given by either
    \begin{enumerate}[listparindent=\parindent,parsep=5pt,label={(\arabic*)}]
      \item $(x+2),(x+3)^2,(x+2)^3,(x+3)^2$, or
      \item $(x+3),(x+2),(x+3),(x+2)^3,(x+3)^2$.
    \end{enumerate}
    Thus, the Jordan canonical form of $M$ is an $8\times 8$ diagonal block matrix, where either:
    \begin{enumerate}[listparindent=\parindent,parsep=5pt,label={(\arabic*)}]
      \item $M$ has $4$ Jordan blocks: a $2$-Jordan block of size $1$, two
      $3$-Jordan blocks of size $2$, and a $2$-Jordan block of size $3$.
      \item $M$ has $5$ Jordan blocks: a $2$-Jordan block of size $1$, two
      $3$-Jordan blocks of size $1$, a $3$-Jordan block of size $2$, and a
      $2$-Jordan block of size 3.
    \end{enumerate}
    Recall given some $a\in\mathbb{F}_5$, an $a$-Jordan block of size $k$ is a
    $k\times k$ matrix with $a$'s along the diagonal, $1$'s on the first
    superdiagonal, and $0$'s elsewhere. For example, if $M$ is of the first
    type (i.e., if $M$ has four elementary divisors), then the matrix
    \[
      \left(
        \begin{array}{c|cc|cc|ccc}
          2 & 0 & 0 & 0 & 0 & 0 & 0 & 0 \\ \hline
          0 & 3 & 1 & 0 & 0 & 0 & 0 & 0 \\
          0 & 0 & 3 & 0 & 0 & 0 & 0 & 0 \\ \hline
          0 & 0 & 0 & 3 & 1 & 0 & 0 & 0 \\
          0 & 0 & 0 & 0 & 3 & 0 & 0 & 0 \\ \hline
          0 & 0 & 0 & 0 & 0 & 2 & 1 & 0 \\
          0 & 0 & 0 & 0 & 0 & 0 & 2 & 1 \\
          0 & 0 & 0 & 0 & 0 & 0 & 0 & 2 \\
        \end{array}
      \right)
    \]
    is a Jordan canonical form for $M$ (the Jordan blocks have been outlined).
  \end{solution}

  \item (January 2021, Q2)
  \begin{enumerate}[listparindent=\parindent,parsep=5pt,label={(\alph*)}]
    \item Let $R$ be the ring of $3\times 3$ matrices over $\mathbb{Q}$, and
    let $S$ denote the ring of $2\times 2$ matrices over $\mathbb{Q}$. Is there
    a surjective ring homomorphism $\phi:R\to S$? Justify your answer.

    \medskip

    No.

    \begin{proof}
      We claim that $R$ has a cube root of $2$, but $S$ does not. Supposing
      this were true, suppose for the sake of a contradiction that there
      existed a surjective ring map $\phi:R\to S$. Let $A\in R$ such that
      $A^3=2$. Then we'd have
      \[
        \phi(A)^3=\phi(A^3)=\phi(2)=\phi(1+1)=\phi(1)+\phi(1)=1+1=2,
      \]
      a contradiction of the fact that $S$ does not admit a cube root of $2$.

      Now, it remains to prove the claims. First, to see that $R$ has a cube root of
      $2$, consider the $\mathbb{Q}[x]$-module $M:=\mathbb{Q}[x]/(x^3-2)$,
      which has dimension $3$ as a $\mathbb{Q}$-vector space
      ($\{\overline{1},\overline{x},\overline{x}^2\}$ is a basis). There is a
      $\mathbb{Q}$-linear map $M\to M$ given by $Tm:=x\cdot m$. Note that
      $T^3=2$, as given $m\in M$, we have $T^3(m)=\overline{x}^3m=2m$. Hence
      $T$ is an endomorphism of a $3$-dimensional $\mathbb{Q}$-vector space
      which satisfies $T^3=2I$. Then fixing a basis for $M$ yields a matrix $A$
      satisfying $A^3=2I$, as desired.

      To see that $S$ does not have a cube root of $2$, suppose for the sake of
      a contradiction that it did, so there exists some $A\in S$ with
      $A^3=2I$. Let $f(x):=x^3-2$, and define $m(x)$ to be the minimal
      polynomial of $A$, which is of degree at most $2$ because $A$ is a
      $2\times 2$ matrix (by Cayley-Hamilton). We know that $m(A)=f(A)=0$, so
      that $m$ divides $f$. Since $\deg m<\deg f$, it follows that $f$ is
      reducible. But this is absurd: by the rational root test any rational
      root of $f$ has to divide $2$, and one can easily check that $\pm 1,\pm
      2$ are not roots of $f$, so that $f$ is irreducible because it is a cubic
      with no roots. Thus we obtain a contradiction, $S$ could not have had a
      cube root of $2$ in the first place.
    \end{proof}
    \item Compute $\gcd(17+i,24+2i)$ in the ring $\mathbb{Z}[i]$.
    \begin{solution}
      Recall $\mathbb{Z}[i]$ is a Euclidean domain with norm function given by
      $N(\alpha):=|\alpha|^2=\Re\alpha^2+\Im\alpha^2$. Thus we may apply the
      Euclidean algorithm (\autoref{euclidean_algorithm}) in order to compute
      the gcd. Set $r_{-2}=24+2i$ and $r_{-1}=17+i$. Assuming we've defined
      $r_{k-2}$ and $r_{k-1}$ for some $k\geq 0$, since $\mathbb{Z}[i]$ is a
      Euclidean domain, there exists some $q_k,r_k\in\mathbb{Z}[i]$ with
      $|r_k|^2<|r_{k-1}|^2$ and
      \[
        r_{k-2}=r_{k-1}q_k+r_k.
      \]
      Continuing this process inductively yields a sequence of Gaussian integers
      \[
        \{r_{-2},r_{-1},r_0,r_1,\ldots,r_{n-1},r_n=0\}.
      \]
      Then $r_{n-1}$ will be a GCD of $r_{-2}$ and $r_{-1}$. Now we just carry out the algorithm:
      \begin{align*}
        24+2i=(17+i)(2)-10 && |-10|^2=(10)^2<|17+i|^2=(17)^2+1 \\
        17+i=(-10)(-2)+(-3+i) && |-3+i|^2=3^2+1^2=10<100=|-10|^2 \\
        -10=(-3+i)(3+i)+0 && |0|^2<|-10|^2.
      \end{align*}
      Thus we can take $r_0=-10$ (with $q_0=2$), $r_1=-3+i$ (with $q_1=-2$),
      and $r_2=0$ (with $q_2=3+i$). Hence $r_1=-3+i$ is a GCD of $17+i$ and
      $24+2i$.
    \end{solution}
  \end{enumerate}

  \item (January 2021, Q3) Suppose $A$ is a $9\times 9$ matrix over the field
  $\mathbb{F}_5$ with $5$ elements such that the characteristic polynomial of
  $A$ is $(x-1)^2(x-3)^4(x^3-1)$ and the minimal polynomial of $A$ is
  $(x-1)(x-3)^3(x^3-1)$. Compute the following:
  \begin{enumerate}[listparindent=\parindent,parsep=5pt,label={(\alph*)}]
    \item The possible Jordan canonical form (or forms) of $A$ over a suitable
    extension of $\mathbb{F}_5$;
    \begin{solution}
      First of all, note that $x^3-1$ factors as $(x-1)(x^2+x+1)$. It is
      straightforward to see that $x^2+x+1$ is irreducible in $\mathbb{F}_5$,
      as it is quadratic and has no roots in $\mathbb{F}_5$. Thus we must pass
      to a splitting field of $x^2+x+1$, which is
      $\mathbb{F}_{25}\cong\mathbb{F}_5(\alpha)$, where $\alpha$ is a root of
      $x^2+x+1$. It is straightforward to see that $x^2+x+1$ factors as
      $(x-\alpha)(x+\alpha+1)$ over $\mathbb{F}_5(\alpha)$.

      Thus the characteristic polynomial of $A$ is 
      \[
        c(x)=(x-1)^3(x-3)^4(x-\alpha)(x+\alpha+1)
      \]
      and the minimal polynomial is
      \[
        m(x)=(x-1)^2(x-3)^3(x-\alpha)(x+\alpha+1)
      \]
      First we find the invariant factors
      $f_1,f_2,\ldots,f_k\in\mathbb{F}_5(\alpha)[x]$ of $A$. Using facts we
      know about invariant factors, one can check $A$ has precisely two
      invariant factors, namely $f_1=(x-1)(x-3)$ and $f_2=m$. Then the
      elementary divisors of $A$ are the prime powers dividing the invariant
      factors (counted individually, for each invariant factor), so they are
      given by
      \[
        x-1,\ x-3,\ (x-1)^2,\ (x-3)^3,\ x-\alpha,\ x+(\alpha+1)
      \]
      Hence the Jordan canonical form of $A$ has six Jordan blocks: a
      $1$-Jordan block of size $1$, a $3$-Jordan block of size $1$, a
      $1$-Jordan block of size $2$, a $3$-Jordan block of size $3$, a
      $\alpha$-Jordan block of size $1$, and a $-(\alpha+1)$-Jordan block of
      size $1$. Thus there are $6! = 720$ Jordan canonical forms for $A$, given
      by different possible arrangements of the six distinct Jordan blocks. For
      example, one such matrix is given by
      \[
        \left(
          \begin{array}{c|c|cc|ccc|c|c}
            1 & 0 & 0 & 0 & 0 & 0 & 0 & 0 & 0 \\ \hline
            0 & 3 & 0 & 0 & 0 & 0 & 0 & 0 & 0 \\ \hline
            0 & 0 & 1 & 1 & 0 & 0 & 0 & 0 & 0 \\
            0 & 0 & 0 & 1 & 0 & 0 & 0 & 0 & 0 \\ \hline
            0 & 0 & 0 & 0 & 3 & 1 & 0 & 0 & 0 \\
            0 & 0 & 0 & 0 & 0 & 3 & 1 & 0 & 0 \\
            0 & 0 & 0 & 0 & 0 & 0 & 3 & 0 & 0 \\ \hline
            0 & 0 & 0 & 0 & 0 & 0 & 0 & \alpha & 0 \\ \hline
            0 & 0 & 0 & 0 & 0 & 0 & 0 & 0 & -(\alpha+1) \\
          \end{array}
        \right)
      \]
    \end{solution}
    \item The possible rational canonical form (or forms) of $A$.
    \begin{solution}
      Above we found the invariant factors of $A$ were
      \[
        f_1=(x-1)(x-3)\equiv x^2-4x-2\bmod 5
      \]
      and
      \begin{align*}
        f_2
        &=(x-1)(x-3)^3(x^3-1) \\
        &\equiv x^7-4x^5-3x^3-x^2-x-2\bmod 5,
      \end{align*}
      so the rational canonical form is the block matrix with blocks the
      companion matrix of $f_1$ and the companion matrix of $f_2$:
      \[
        \left(
          \begin{array}{cc|ccccccc}
            0 & 2 & 0 & 0 & 0 & 0 & 0 & 0 & 0 \\ 
            1 & 4 & 0 & 0 & 0 & 0 & 0 & 0 & 0 \\ \hline
            0 & 0 & 0 & 0 & 0 & 0 & 0 & 0 & 2 \\
            0 & 0 & 1 & 0 & 0 & 0 & 0 & 0 & 1 \\ 
            0 & 0 & 0 & 1 & 0 & 0 & 0 & 0 & 1 \\
            0 & 0 & 0 & 0 & 1 & 0 & 0 & 0 & 3 \\
            0 & 0 & 0 & 0 & 0 & 1 & 0 & 0 & 0 \\ 
            0 & 0 & 0 & 0 & 0 & 0 & 1 & 0 & 4 \\
            0 & 0 & 0 & 0 & 0 & 0 & 0 & 1 & 0 \\
          \end{array}
        \right)
      \]
    \end{solution}
  \end{enumerate}

  \item (August 2020, Q3)
  \begin{enumerate}[listparindent=\parindent,parsep=5pt,label={(\alph*)}]
    \item Compute, if possible, $\gcd(2+8i,17-17i)$ in the ring $\mathbb{Z}[i]$
    of Gaussian integers.
    \begin{solution}
      Perform the Euclidean algorithm for computing a gcd in a domain (which
      $\mathbb{Z}[i]$ is with norm function
      $N(\alpha)=|\alpha|^2=(\Re\alpha)^2+(\Im\alpha)^2$).
      \begin{align*}
        17-17i
        &=(2+8i)(-2-3i)+(-3+5i) \\
        2+8i
        &=(-3+5i)(1-i)+0,
      \end{align*}
      so that $\gcd(2+8i,17-17i)=-3+5i$.
    \end{solution}
    \item Determine whether the following polynomials are reducible or
    irreducible in given rings
    \item[(b1)] $x^4+x^2+1$ in $\mathbb{Z}_2[x]$, where $\mathbb{Z}_2$ is the
    field with $2$ elements;
    \begin{solution}
      Note that
      \[
        x^4+x^2+1\equiv (x^2+x+1)^2\bmod 2,
      \]
      so the polynomial is reducible (this is an irreducible factorization, as
      $x^2+x+1$ is quadratic with no roots in $\mathbb{F}_2$).
    \end{solution}
    \item[(b2)] $x^4+5x^3+10x^2+15x+5$ in $R[x]$, where $R=\mathbb{Z}[i]$;
    \begin{solution}
      Consider the element $p=2+i\in R$. It is prime/irreducible in $R$ because
      $N(p)=2^2+i^2=5$ is prime. Moreover, $p(2-i)=5$, and $p$ does not divide
      $(2-i)$ (because $up\neq 2-i$ for $u\in\{\pm 1,\pm i\}$), so $p$ divides
      $5$ exactly once. Hence $p$ divides the non-leading coefficients of the
      monic polynomial $x^4+5x^3+10x^2+15x+5$, and $p^2$ does not divide the
      constant term, so by Eisenstein's criterion the polynomial is
      irreducible.
    \end{solution}
    \item[(b3)] $2x^4+4x^3+8x^2+12x+20$ in $\mathbb{Z}[x]$.
    \begin{solution}
      This polynomial is reducible, it factors as
      \[
        2(x^4+2x^3+4x^2+6x+10).
      \]
      (this is an irreducible factorization by Eisenstein's with $p=2$).
    \end{solution}
  \end{enumerate}

  \item (August 2020, Q4)
  \begin{enumerate}[listparindent=\parindent,parsep=5pt,label={(\alph*)}]
    \item Let $A$ be an $n\times n$ complex matrix and let $f$ and $g$ be the
    characteristic and minimal polynomials of $A$, resp. Suppose that
    $f(x)=g(x)(x-i)$ and $g(x)^2=f(x)(x^2+1)$. Determine all possible Jordan
    canonical forms of $A$.
    \begin{solution}
      \[
        g(x)^2
        =f(x)(x^2+1)
        =f(x)(x+i)(x-i)
        =g(x)(x-i)^2(x+i),
      \]
      and $\mathbb{C}[x]$ is a domain, so it follows that $g(x)=(x-i)^2(x+i)$
      and $f(x)=(x-i)^3(x+i)^2$. Now, we'd like to find the invariant factors
      $f_1,f_2,\ldots,f_k\in \mathbb{C}[x]$ of $A$. Recall the following facts
      about the invariant factors:
      \begin{itemize}
        \item $f_j$ is nonconstant and monic for $j=1,\ldots,k$.
        \item $f_j\mid f_{j+1}$ for $1\leq j<k$.
        \item $f_k=m$.
        \item $f_1\cdots f_k=c$.
      \end{itemize}
      Then it follows that the invariant factors of $A$ are given by
      $f_1(x)=(x-i)(x+i)$ and $f_2(x)=g(x)=(x-i)^2(x+i)$. Thus the elementary
      divisors are given by:
      \[
        x-i,\quad x+i, \quad(x-i)^2, \quad\text{and}\quad x+i.
      \]
      Hence the Jordan canonical form of $A$ is any $5\times 5$ block diagonal
      matrix with the following diagonal blocks: Two $i$-Jordan blocks of size
      $1$, An $i$-Jordan block of size $2$, and a $-i$-Jordan block of size
      $1$. For example, the following matrix is a Jordan canonical form for
      $A$:
      \[
        \left(
          \begin{array}{c|c|c|cc}
            i & 0 & 0 & 0 & 0 \\ \hline
            0 & i & 0 & 0 & 0 \\ \hline
            0 & 0 & -i & 0 & 0 \\ \hline
            0 & 0 & 0 & i & 1 \\
            0 & 0 & 0 & 0 & i
          \end{array}
        \right).\qedhere
      \]
    \end{solution}
    \item Let $\mathbb{F}$ be a field of characteristic $p>0$ and $p\neq 3$. If
    $\alpha$ is a root of the polynomial $f(x)=x^p-x+3$, in an extension of the
    field $\mathbb{F}$, show that $f(x)$ has $p$ distinct roots in the field
    $\mathbb{F}(\alpha)$.
    \begin{proof}
      \color{red}TODO
    \end{proof}
  \end{enumerate}

  \item (January 2020 Q1) Let $G$ be a finite group of order $100$.
  \begin{enumerate}[listparindent=\parindent,parsep=5pt,label={(\alph*)}]
    \item Show that $G$ is solvable. (Feel free to use that groups of order
    $p^2$ are abelian for $p$ a prime number).
    \begin{proof}
      By Sylow 3, we have that $n_5(G)\equiv 1\bmod 5$ and $n_5(G)\mid 4$. It
      follows that $n_5(G)=1$, so there exists a unique subgroup $H\leq G$ of
      order $25$, and by Sylow 2 $H$ is normal in $G$. Then we get a subnormal sequence of subgroups
      \[
        0\unlhd H\unlhd G.
      \]
      Since $|H|=5^2$ and $5$ is prime, we have that $H/0\cong H$ is abelian,
      as desired. Moreover, the quotient $G/H$ has order $|G|/|H|=2^2$, and $2$
      is prime, so $G/H$ is abelian as well. Thus we have directly shown $G$ is
      simple.
    \end{proof}

    \item Show, by giving a counterexample, that $G$ need not be nilpotent.
    \begin{proof}
      Consider the group $G=D_{10}\times\mathbb{Z}/2\times\mathbb{Z}/5$ (recall
      $D_{10}$ has presentation $\langle r,s\mid r^5,s^2,rsrs\rangle$). The
      group $G$ has order $10\cdot 2\cdot 5=100$. Note that $n_2(D_{10})>1$:
      $s$ generates a subgroup of order $2$ which is not normal: the relation
      $rsrs=1$ implies $sr^{-1}=rs$, so that $rsr^{-1}s^{-1}=rrss^{-1}=r^2$,
      which is not the identity, so $r$ does not commute with $s$, meaning
      $r\langle s\rangle r^{-1}\neq\langle s\rangle$. Then $K:=\langle
      s\rangle\times\mathbb{Z}/2\times\mathbb{Z}/5$ is a non-normal $2$-Sylow
      subgroup of $G$, because
      \[
        (r,e,e)K(r,e,e)^{-1}=r\langle s\rangle r^{-1}\times\mathbb{Z}/2\times\{e\}\neq K.
      \]
      Thus $G$ is not nilpotent.
    \end{proof}
  \end{enumerate}

  \item (January 2020 Q2) Decide which of the following sets are ideals of the
  ring $\mathbb{Z}[x]$. Provide justification.
  \begin{enumerate}[listparindent=\parindent,parsep=5pt,label={(\alph*)}]
    \item The set of all polynomials whose coefficient of $x^2$ is a multiple
    of $3$.

    \medskip

    This is not an ideal.
    \begin{proof}
      Consider $f=3x^2+x$, which clearly belongs to the collection $S$ of
      polynomials in question. Then $xf(x)=3x^3+x^2\notin S$, since $3\nmid
      1$. Hence $\mathbb{Z}[x]S\not\subseteq S$, so $S$ is not an ideal.
    \end{proof}
    \item $\mathbb{Z}[x^2]$, the set of all polynomials in which only even
    powers of $x$ appear.

    \medskip

    This is not an ideal
    \begin{proof}
      $x^2\in S$, but $x\cdot x^2=x^3\notin S$, so $\mathbb{Z}[x]S\not\subseteq
      S$.
    \end{proof}
    \item The set of polynomials whose coefficients sum to zero.

    \medskip

    This is an ideal
    \begin{proof}
      Clearly if $f,g\in S$ then $-f,f+g\in S$, so $S$ is an abelian group. Now
      we need to show $\mathbb{Z}[x]S\subseteq S$. To see this, let
      $f=\sum_{j=0}^{n}a_jx^j\in\mathbb{Z}[x]$ and $g=\sum_{i=0}^{m}b_ix^i\in
      S$. Then we have
      \[
        fg=\sum_{j=0}^{n}\sum_{i=0}^{m}a_jb_ix^{i+j},
      \]
      so that the sum of the coefficients of $fg$ is
      \[
        \sum_{j=0}^{n}\sum_{i=0}^{m}a_jb_i
        =\sum_{j=0}^{n}a_j\left(\sum_{i=0}^{m}b_i\right)
        =\sum_{j=0}^{n}a_j\cdot 0
        =0,
      \]
      where the middle equality follows because $g=\sum_{i=0}^{m}b_ix^i\in S$,
      so that $\sum_{i=0}^{m}b_i=0$.
    \end{proof}
  \end{enumerate}

  \item (January 2020 Q3) Find the possible Jordan canonical forms of $7\times
  7$ matrices $M$ with entries in $\mathbb{C}$ satisfying the following
  criteria:
  \begin{itemize}
    \item the characteristic polynomial of $M$ is $(z-3)^4(z-5)^3$,
    \item the minimal polynomial of $M$ is $(z-3)^2(z-5)^2$, and
    \item the $\mathbb{C}$-vector space dimension of the nullspace of
    $3\cdot\Id- M$ is $2$.
  \end{itemize}
  \begin{solution}
    First we'd like to find the invariant factors $f_1,f_2,\ldots,f_k\in
    \mathbb{C}[x]$ of $M$. Recall the following facts about the invariant
    factors:
    \begin{itemize}
      \item $f_j$ is nonconstant and monic for $j=1,\ldots,k$.
      \item $f_j\mid f_{j+1}$ for $1\leq j<k$.
      \item $f_k$ is the minimal polynomial of $A$.
      \item $f_1\cdots f_k$ is the characteristic polynomial of $A$.
    \end{itemize}
    Using the first two bullet points and these facts, we get that the the invariant divisors are either
    \begin{itemize}
      \item $f_1=(z-3)^2(z-5)$, $f_2=m:=(z-3)^2(z-5)^2$, or
      \item $f_1=z-3$, $f_2=(z-3)(z-5)$, and $f_3=m=(z-3)^2(z-5)^2$.
    \end{itemize}
    Hence the elementary divisors of $M$ are either
    \begin{itemize}
      \item $(z-3)^2$, $z-5$, $(z-3)^2$, and $(z-5)^2$, or
      \item $z-3$, $z-3$, $z-5$, $(z-3)^2$, and $(z-5)^2$.
    \end{itemize}
    It is straightforward to see that given some $\alpha\in\mathbb{C}$, the
    dimension of $\ker(\alpha I-M)$ is the number of times some power of
    $z-\alpha$ appears as an elementary divisor of $M$. Hence since
    $\dim\ker(3I-M)=2$, powers of $z-\alpha$ must appear as elementary divisors
    of $M$ exactly twice. Thus the elementary divisors of $M$ must be 
    \[
      (z-3)^2,\quad z-5,\quad (z-3)^2,\quad\text{and}\quad (z-5)^2.
    \]
    Thus a matrix is a Jordan canonical form for $M$ if it is block diagonal,
    with: two $3$-Jordan blocks of size $2$, a $5$-Jordan block of size $1$,
    and a $5$-Jordan block of size $2$. For example, the following matrix is a
    Jordan canonical form for $M$:
    \[
      \left(
        \begin{array}{c|cc|cc|cc}
          5 & 0 & 0 & 0 & 0 & 0 & 0 \\ \hline
          0 & 3 & 1 & 0 & 0 & 0 & 0 \\
          0 & 0 & 3 & 0 & 0 & 0 & 0 \\ \hline
          0 & 0 & 0 & 5 & 1 & 0 & 0 \\
          0 & 0 & 0 & 0 & 5 & 0 & 0 \\ \hline
          0 & 0 & 0 & 0 & 0 & 3 & 1 \\
          0 & 0 & 0 & 0 & 0 & 0 & 3
        \end{array}
      \right).\qedhere
      \]
  \end{solution}

  \item (January 2020 Q4) Determine if the following polynomials are irreducible over $\mathbb{Z}$.
  \begin{enumerate}[listparindent=\parindent,parsep=5pt,label={(\alph*)}]
    \item $x^3-5x-1$.

    \medskip

    The polynomial is irreducible
    \begin{proof}
      Since the polynomial $f=x^3-5x-1$ is monic, by Gauss' Lemma it suffices
      to show that the polynomial is irreducible over $\mathbb{Q}$. By the
      rational root theorem, any rational root of the polynomial is an integer
      dividing $-1$, i.e., if $f(x):=x^3-5x-1$ has a root in $\mathbb{Q}$ it is
      $1$ or $-1$. Yet one can check that $f(1),f(-1)\neq 0$, so $f$ has no
      roots. Thus $f$ is a cubic over the field $\mathbb{Q}$ with no roots, so
      $f$ is irreducible over $\mathbb{Q}$, as desired.
    \end{proof}
    \item $x^4+10x^2+5$.

    \medskip

    This polynomial is irreducible.
    \begin{proof}
      This is Eisenstein's with $p=5$.
    \end{proof}
  \end{enumerate}
\end{enumerate}

\end{document}
% hi Isaiah ! - Aden
% wassup. you're gettin vodka rn (it's 5:51 on Thursday July 18). ok. i dont have much else to say. wish I did tho. hmmmmmm
